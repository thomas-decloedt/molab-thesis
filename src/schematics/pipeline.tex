\begin{figure}[ht]
	\centering
	\begin{tikzpicture}[scale=0.5, transform shape]
		\tikzset{node distance = 30pt and 45pt}

		\node[input_output]              (Req) {Request};
		\node[block, right=of Req]       (DE)  {Domain Expert};
		\node[block, right=of DE]        (S2S) {Translator};
		\node[block, right=of S2S]       (L)   {Linker};
		\node[block, right=of L]         (E)   {Execution};
		\node[block, right=of E]         (RH)  {Result Handling};
		\node[input_output, right=of RH] (Res) {Response};

		\draw[->] (Req) -- (DE)  node[descr] {Question};
		\draw[->] (DE)  -- (S2S) node[descr] {\glsentryshort{squall}};
		\draw[->] (S2S) -- (L)   node[descr] {\glsentryshort{sparql}};
		\draw[->] (L)   -- (E)   node[descr] {Query};
		\draw[->] (E)   -- (RH)  node[descr] {Results};
		\draw[->] (RH)  -- (Res) node[descr] {};

		\begin{scope}[on background layer]
			\draw[->, densely dotted, thin]
				(Req.east) -- ++ (20pt,0) |- ([yshift=-11pt]RH.west)
				node[descr, below] {Config};
		\end{scope}

		\node[draw, dashed, fit=(DE) (S2S) (L), label={Text-to-\gls{sparql}}] (T2S) {};

	\end{tikzpicture}
	\caption{Pipeline;
		The request config determines the target \gls{kg} and determines the returned response,
		e.g., visualization.
	}
	\label{fig:pipeline}
\end{figure}

