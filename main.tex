% Note: remove `openany` for printed version
\documentclass[12pt,a4paper,openany,dutch,english]{extbook}
\usepackage[a4paper,includeheadfoot,margin=2.50cm]{geometry}


% By default, LaTeX tries to stretch whitespace between paragraphs on a page in order to reduce whitespace at the end of the page. This sometimes gives ugly results. The following command disables that stretching.
\raggedbottom % Don't reduce whitespace at the end of a page.

\renewcommand{\baselinestretch}{1.2}  % stretch horizontal space between everything by 20%


\usepackage[normalem]{ulem} % Underlined numbers in S (siunitx) columns
\usepackage[hyphens]{url} % Break line on hyphens in long urls
\usepackage{graphicx}
\graphicspath{{images/}}
\usepackage{pdfpages}
\usepackage{enumitem}
\usepackage{float}
\usepackage{caption}
\usepackage{subcaption}
\usepackage[toc,page]{appendix}
\usepackage{fontspec}
\usepackage{yhmath}

% Don't indent table of contents, list of figures, and list of tables
\usepackage{tocloft}
\setlength{\cftsecindent}{0pt}    % Remove indent for \section in Table of Contents
\setlength{\cftsubsecindent}{0pt} % Remove indent for \subsection in Table of Contents
\setlength{\cftfigindent}{0pt}    % remove indentation from figures in List of Figures
\setlength{\cfttabindent}{0pt}    % remove indentation from tables in List of Tables

\usepackage{parskip} % Add space between two paragraphs and don't indent the first line of the paragraph

%
% UGent style guide — use UGent Panno Text if available, else Arial (recommended for external distribution)
%
\IfFileExists{fonts/UGentPannoText-Normal.ttf}{%
  \setmainfont[
    Path=fonts/,
    BoldFont      =UGentPannoText-SemiBold.ttf,
    ItalicFont    =UGentPannoText-Normal.ttf,
    ItalicFeatures={FakeSlant=0.3},
    BoldItalicFont=UGentPannoText-SemiBold.ttf,
    BoldItalicFeatures={FakeSlant=0.3},
  ]{UGentPannoText-Normal.ttf}
}{%
  \setmainfont{Arial}
}
\urlstyle{same} % Also use the default font for URLs


% If you want left justified text, uncomment the line below.
%\usepackage[document]{ragged2e} % Left justify all text

% Style Chapter titles so they have the chapter number in grey.
\usepackage{color}
\definecolor{chaptergrey}{rgb}{0.5,0.5,0.5}
\usepackage[explicit, pagestyles]{titlesec}
\titleformat{\chapter}[display]{\bfseries}{\color{chaptergrey}\fontfamily{lmr}\fontsize{80pt}{100pt}\selectfont\thechapter}{0pt}{\Huge #1}
\titlespacing*{\chapter}{0pt}{-80pt}{30pt}


% Header showing chapter number and title and footer showing page number
\newpagestyle{fancy}{%
  \sethead{} % left
          {} % center
          {\Large\thechapter~~\chaptertitle} %right
  \setfoot{} % left
          {\thepage} % center
          {} %right
  \setheadrule{0pt}
}
\pagestyle{fancy}

% Header showing chapter title and footer showing page number
\newpagestyle{numberless}{%
  \sethead{} % left
          {} % center
          {\Large\chaptertitle} %right
  \setfoot{} % left
          {\thepage} % center
          {} %right
  \setheadrule{0pt}
}

% We use the package `minted` for modern code highlighting.
\usepackage[newfloat,chapter]{minted}
%\SetupFloatingEnvironment{listing}{name=Codefragment, listname=Lijst van codefragmenten} % lang:dutch
\SetupFloatingEnvironment{listing}{name=Listing, listname=List of Listings} % lang:english
\usemintedstyle{pastie} % for other highlighting color schemes, see https://www.overleaf.com/learn/latex/Code_Highlighting_with_minted#Reference_guide
\setminted{
    breaklines=true,
	 fontsize=\footnotesize,
}
\AtBeginEnvironment{listing}{\renewcommand{\colorbox}[3][]{#3}}

\PassOptionsToPackage{hyphens}{url}
\usepackage[hidelinks]{hyperref}
\usepackage{url}

%
% Defines \checkmark to draw a checkmark
%
\usepackage{tikz}
\def\checkmark{\tikz\fill[scale=0.4](0,.35) -- (.25,0) -- (1,.7) -- (.25,.15) -- cycle;}

%
% For tables
%
\usepackage{booktabs}
\usepackage{array}
\usepackage{ragged2e}  % for '\RaggedRight' macro (allows hyphenation)
\newcolumntype{L}[1]{>{\raggedright\let\newline\\\arraybackslash\hspace{0pt}}m{#1}}
\newcolumntype{C}[1]{>{\centering\let\newline\\\arraybackslash\hspace{0pt}}m{#1}}
\newcolumntype{R}[1]{>{\raggedleft\let\newline\\\arraybackslash\hspace{0pt}}m{#1}}

%
% Support for splitting Dutch words correctly
%
\usepackage{polyglossia}
%\setdefaultlanguage[babelshorthands=true]{dutch} % lang:dutch
\setmainlanguage{english}                       % lang:english

% Manually specify additional hypnations for words
%
% Translated strings. If these aren't set, the English words are used.
%

% \addto\captionsenglish{\renewcommand{\contentsname}{Inhoudsopgave}}   % lang:dutch

%\usepackage[numbers]{natbib}       % For bibliography; use numeric citations
%\bibliographystyle{IEEEtranDOI}
\usepackage[backend=biber,style=ieee]{biblatex}
\usepackage[nottoc]{tocbibind}     % Put Bibliography in ToC
\addbibresource{references.bib}
\AtEveryBibitem{
    \clearfield{urlyear}
    \clearfield{urlmonth}
}
% If DOI present, don't include eprint (e.g., arxiv id)
\AtEveryBibitem{%
  \ifboolexpr{
    test {\iffieldundef{doi}}
    or
    test {\iffieldundef{eprint}}
  }
  {} % Keep both if no DOI or eprint
  {%
    \clearfield{eprint} % Remove eprint if DOI exists
  }
}

% Fix error "Package hyperref Warning: The anchor of a bookmark and its parent's must not be the same. Added a new anchor on ..."
\newcommand{\sectionbreak}{\phantomsection}

% \renewcommand\appendixtocname{Bijlagen}                     % lang:dutch
% \renewcommand\appendixpagename{Bijlagen}                    % lang:dutch


\usepackage[toc,acronym]{glossaries}  % for list of acronyms
\makeglossaries                       % start internal list of acronyms

\usepackage{siunitx}

\sisetup{
	output-decimal-marker = {.}, % Use dot for decimals
	group-separator = { },       % Use space for grouping
	group-minimum-digits = 3,    % Group digits in thousands
	tight-spacing,
	detect-all,
	range-phrase = --,
	range-units = single,
}

%
% Set the title and your name
%
%%%%%%%%%%%%%%%%%%%%%%%%%%%%%%%%%%%%%%%%%%%%%%%%%%%%%%%%%%%%%%%%%%%%%%
%
% Add the specific info for your thesis
%
%%%%%%%%%%%%%%%%%%%%%%%%%%%%%%%%%%%%%%%%%%%%%%%%%%%%%%%%%%%%%%%%%%%%%%

\title{Mo-Lab: An Interactive Model Reporting and Analysis in OpenCAESER through Natural Language Processing and SPARQL Integration}
\author{Thomas Decloedt}







%%%%%%%%%%%%%%%%%%%%%%%%%%%%%%%%%%%%%%%%%%%%%%%%%%%%%
% Add all the acronyms you use in your thesis here. %
% These will be added to the List of Acronyms       %
%%%%%%%%%%%%%%%%%%%%%%%%%%%%%%%%%%%%%%%%%%%%%%%%%%%%%

\newacronym{abox}{ABox}{Assertional Box}
\newacronym{ai}{AI}{Artificial Intelligence}
\newacronym{ann}{ANN}{Artificial Neural Network}
\newacronym{api}{API}{Application Programming Interface}
\newacronym{aws}{AWS}{Amazon Web Services}
\newacronym{bgp}{BGP}{Basic Graph Pattern}
\newacronym{bleu}{BLEU}{Bilingual Evaluation Understudy}
\newacronym{bnode}{BNode}{Blank Node}
\newacronym{cad}{CAD}{Computer-Aided Design}
\newacronym{cf}{CF}{Coherency Factor}
\newacronym{cgp}{CGP}{Complex Graph Pattern}
\newacronym{cnl}{CNL}{Controlled Natural Language}
\newacronym{da}{DA}{Domain Adaptation}
\newacronym{dpt}{DPT}{Demonstration Per Template}
\newacronym{fsl}{FSL}{Few-Shot Learning}
\newacronym{gat}{GAT}{Graph Attention Network}
\newacronym{ged}{GED}{Graph Edit Distance}
\newacronym{gnn}{GNN}{Graph Neural Network}
\newacronym{gpm}{GPM}{Graph Pattern Matching}
\newacronym{gp}{GP}{Graph Pattern}
\newacronym{icl}{ICL}{In-Context Learning}
\newacronym{ide}{IDE}{Integrated Development Environment}
\newacronym{ir}{IR}{Information Retrieval}
\newacronym{jit}{JIT}{Just-In-Time}
\newacronym{kbqa}{KBQA}{Knowledge Base Question Answering}
\newacronym{kb}{KB}{Knowledge Base}
\newacronym{kgqan}{KGQAn}{Knowledge Graph Question Answering platform}
\newacronym{kgqa}{KGQA}{Knowledge Graph Question Answering}
\newacronym{kg}{KG}{Knowledge Graph}
\newacronym{llm}{LLM}{Large Language Model}
\newacronym{lm}{LM}{Language Model}
\newacronym{lora}{LoRA}{Low-Rank Adaptation}
\newacronym{mbse}{MBSE}{Model-Based Systems Engineering}
\newacronym{mcs}{MCS}{Maximum Common Subgraph}
\newacronym{mlp}{MLP}{Multilayer Perceptron}
\newacronym{ml}{ML}{Machine Learning}
\newacronym{nlp}{NLP}{Natural Language Processing}
\newacronym{nlu}{NLU}{Natural Language Understanding}
\newacronym{nmt}{NMT}{Neural Machine Translation}
\newacronym{nnqt}{NNQT}{Normalized Natural Question Template}
\newacronym{ocl}{OCL}{Object Constraint Language}
\newacronym{oml}{OML}{Ontological Modeling Language}
\newacronym{oov}{OOV}{Out-of-Vocabulary}
\newacronym{orkg}{ORKG}{Open Research Knowledge Graph}
\newacronym{owl2}{OWL2}{Web Ontology Language 2}
\newacronym{owl}{OWL}{Web Ontology Language}
\newacronym{pcst}{PCST}{Prize-Collecting Steiner Tree}
\newacronym{peft}{PEFT}{Parameter-Efficient Fine-Tuning}
\newacronym{pgp}{PGP}{Placeholder Graph Pattern}
\newacronym{pllm}{PLLM}{Pre-trained Large Language Model}
\newacronym{plm}{PLM}{Pre-trained Language Model}
\newacronym{qa}{QA}{Question Answering}
\newacronym{qlora}{QLoRA}{Quantized Low Rank Adaption}
\newacronym{qu}{QU}{Question Understanding}
\newacronym{rag}{RAG}{Retrieval-Augmented Generation}
\newacronym{rdf}{RDF}{Resource Description Framework}
\newacronym{s2s}{S2S}{SQUALL-to-SPARQL}
\newacronym{se}{SE}{Systems Engineering}
\newacronym{sparql}{SPARQL}{SPARQL Protocol and RDF Query Language}
\newacronym{sp}{SP}{Semantic Parsing}
\newacronym{sql}{SQL}{Structured Query Language}
\newacronym{squall}{SQUALL}{Semantic Query and Update High-Level Language}
\newacronym{swrl}{SWRL}{Semantic Web Rule Language}
\newacronym{sysml}{SysML}{Systems Modeling Language}
\newacronym{tbox}{TBox}{Terminological Box}
\newacronym{uml}{UML}{Unified Modeling Language}
\newacronym{uri}{URI}{Uniform Resource Identifier}
\newacronym{vtl}{VTL}{Velocity Template Language}
\newacronym{zsl}{ZSL}{Zero-Shot Learning}


%
%  END OF UGHENT HEADER
%

%\usepackage[acronym,nomain]{glossaries}
%\usepackage[square,numbers]{natbib}
\usepackage[table]{xcolor}
%\usepackage{amsmath}
\usepackage{amssymb}
\usepackage{amsthm}
%\usepackage{amsfonts}
\usepackage{arydshln}
\usepackage{braket}
\usepackage{csvsimple}
\usepackage{datetime}
\usepackage{geometry}
%\usepackage{listings}
\usepackage{longtable}
\usepackage{mathtools}
\usepackage{multirow}
\usepackage{pdfcomment}
\usepackage{pdflscape}
\usepackage{pgfplotstable} % For reading CSV files
\usepackage{tabularx}
\usepackage{tikz}
\usepackage{todonotes}
\usepackage{unicode-math}
\usepackage{varwidth}
\usepackage{svg}

\usepackage{cleveref}
\usepackage{mathrsfs}
\usepackage{tikzpeople}
\usepackage{tikz-layers}

% Define a custom format for subsection references using the section symbol (§)
%\crefformat{subsection}{\S#2#1#3}
% Note: Referencing does not feel as consistent anymore, e.g., see table 4 VS see §4

\newcommand{\TODO}[1]{\todo[color=yellow,inline]{TODO: #1}}
\newcommand{\DONE}[1]{\todo[color=green,inline]{DONE: #1}}

\newcommand{\mydateformat}[3]{#1.#2.#3}

% Listings settings.
%\lstset{
  	%frame=single,                      % Frame around the listing
  	%numbers=left,                      % Line numbers
  	%numbersep=5pt,                     % Distance between line numbers and code
  	%numberstyle=\tiny\color{gray},     % Line number style
  	basicstyle=\ttfamily\scriptsize,     % Font setting
  	breaklines=true,                    % Line breaking to fit within column
  	captionpos=b,                       % Caption position
  	columns=fullflexible,               % Improved spacing
  	commentstyle=\color{green},         % Comment color
  	keepspaces=true,                    % Keeps spaces in text
  	keywordstyle=\bfseries\color{blue}, % Bold and colored keywords
  	showspaces=false,                   % Show spaces
  	showstringspaces=false,             % Show spaces in strings
  	stepnumber=1,                       % Line number step
  	stringstyle=\color{red},            % String color
  	tabsize=2,                          % Tab size
  	upquote=true,                       % Fix quotes.
	float,                              % Treat listings as floats
	floatplacement=htbp,                % Placement preference for floats
}

%%% SPARQL
\definecolor{darkgreen}{rgb}{0.0, 0.5, 0.0}

\lstdefinestyle{SPARQL_APPENDIX}{
    basicstyle=\ttfamily\scriptsize,  % Use small text and typewriter font
    breakindent=7pt,                  % Indentation for wrapped lines
    breaklines=true,                  % Enable line breaks
    commentstyle=\color{red},         % Red comments (single-line)
    keywordstyle=\ttfamily\scriptsize\color{black}, % Keyword style
    morecomment=[s][\color{darkgreen}]{/*}{*/}, % Green comments (block)
    morekeywords={},                  % Add specific keywords if needed
    showstringspaces=false,           % Do not show spaces in strings
    stringstyle=\ttfamily\scriptsize\color{black}, % String style
    tabsize=2,                        % Set tab size
    upquote=true,                     % Fix quotes
    xleftmargin=10pt,                 % Add margin for readability
    xrightmargin=10pt,                % Add margin for readability
}

%%% SQUALL
\lstdefinestyle{SQUALL}{
  %backgroundcolor=\color{white},     % Background color
  basicstyle=\ttfamily\scriptsize,         % Use small text and typewriter font
  breakindent=0pt,
  breaklines=true,                    % Enable line breaks
  commentstyle=\color{green},         % Comment color
  keywordstyle=\bfseries\color{blue}, % Keyword color
  showstringspaces=false,             % Do not show spaces in strings
  stringstyle=\color{red},            % String color
  tabsize=2,                          % Set tab size
  xleftmargin=0pt,                    % No left margin
  xrightmargin=0pt,                   % No right margin
}

%%% JSON
\colorlet{punct}{red!60!black}
\definecolor{background}{HTML}{EEEEEE}
\definecolor{delim}{RGB}{20,105,176}
\colorlet{numb}{magenta!60!black}
\lstdefinelanguage{json}{
    basicstyle=\ttfamily\scriptsize,
    numbersep=8pt,
    showstringspaces=false,
    breaklines=true,
    literate=
     *{0}{{{\color{numb}0}}}{1}
      {1}{{{\color{numb}1}}}{1}
      {2}{{{\color{numb}2}}}{1}
      {3}{{{\color{numb}3}}}{1}
      {4}{{{\color{numb}4}}}{1}
      {5}{{{\color{numb}5}}}{1}
      {6}{{{\color{numb}6}}}{1}
      {7}{{{\color{numb}7}}}{1}
      {8}{{{\color{numb}8}}}{1}
      {9}{{{\color{numb}9}}}{1}
      {:}{{{\color{punct}{:}}}}{1}
      {,}{{{\color{punct}{,}}}}{1}
      {\{}{{{\color{delim}{\{}}}}{1}
      {\}}{{{\color{delim}{\}}}}}{1}
      {[}{{{\color{delim}{[}}}}{1}
      {]}{{{\color{delim}{]}}}}{1},
}



% TikZ settings and styles 
\usetikzlibrary{
	angles,
	arrows,
	arrows.meta,
	automata,
	backgrounds,
	calc,
	decorations.pathreplacing,
	fit,
	positioning,
	quotes,
	shapes,
	shapes.multipart,
	matrix,
	graphs,
}

\tikzset{
	arrowtext/.style={
		font=\scriptsize,
	},
	% Vertical and horizontal distance.
	%node distance = 15pt and 60pt,
	block/.style={
		draw,
		rectangle,
		minimum width=2.4cm,
		minimum height=1cm,
		text width=2.4cm,
		text centered,
		fill=white,
		%font=\scriptsize,
	},
	triple_block/.style={
		draw,
		rectangle,
		minimum width=6cm,
		minimum height=1cm,
		text width=6cm,
		text centered,
		fill=white,
	},
	input_output/.style={
		draw,
		ellipse,
		minimum width=1.5cm,
		minimum height=1cm,
		text width=1.5cm,
		text centered,
		fill=white,
	},
	descr/.style={
		midway,
		%font=\scriptsize,
		above,
	},
	connector/.style={
		->,
		%font=\scriptsize
	},
	rectangle connector/.style={
		connector,
		to path={(\tikztostart) -- ++(#1,0pt) \tikztonodes |- (\tikztotarget) },
		pos=0.5
	},
	rectangle connector horizontal/.style={
		connector,
		to path={(\tikztostart) -- ++(0pt,#1) \tikztonodes -| (\tikztotarget) },
		pos=0.5
	},
	rectangle connector/.default=-2cm,
	straight connector/.style={
		connector,
		to path=--(\tikztotarget) \tikztonodes
	},
}

%%% OLD
\tikzstyle{invisible_block_placeholder} = [rectangle, text width=2cm, minimum height=0.5cm]
\tikzstyle{basic_block} = [rectangle, draw, text centered, rounded corners, minimum height=0.75cm]
\tikzstyle{arrow} = [draw, -latex']
\tikzstyle{circular} = [draw, circle, radius=1.0cm,]
\tikzstyle{floatingtext} = [rectangle, text width=7em, text centered, rounded corners, minimum height=2em]
\tikzstyle{line} = [draw]



\theoremstyle{definition}
\newtheorem{definition}{Definition}
\newenvironment{defitemize}
  {\begin{itemize}[left=0pt, labelsep=5pt, itemsep=0pt, topsep=4pt, label=--]}
  {\end{itemize}}

% Adjust this value as needed
\renewcommand{\arraystretch}{1.1}

\setcounter{secnumdepth}{3}


%
%  END OF HEADER
%  The actual latex document content starts here.
%

\begin{document}
\frontmatter
\pagestyle{empty}

% Download the cover sheet from Plato
\includepdf{cover-sheet.pdf}

\chapter*{Acknowledgment}

I would like to express my deepest gratitude to Professor Dr. Mohammad Hamdaqa from Polytechnique Montréal for
his guidance and support throughout this research journey.
I am immensely thankful for the opportunity he provided, the intellectually stimulating conversations we shared,
and the academic freedom he granted me to explore and develop my research ideas.
His encouragement to engage with other experts significantly enriched this work.

I am profoundly grateful to Dr. Maged Elaasar from NASA Jet Propulsion Laboratory (JPL)
and Professor Dr. Bentley James Oakes from Polytechnique Montréal
for their invaluable insights and constructive feedback.
Their expertise and generosity with their time allowed me to refine my ideas and enhance the overall quality of this
thesis.

My sincere thanks go to Professor Dr. ir. Chris Develder at Ghent University for his administrative support,
detailed feedback, and thorough review of the manuscript.
His commitment to excellence and attention to detail were crucial in ensuring this thesis achieved the desired quality.

I also wish to thank François Goybet for granting me the opportunity to utilize his bachelor project,
which proved invaluable in testing my proof of concept.
I had the pleasure of meeting him through Prof. Dr. Hamdaqa during my exchange in Montréal,
where he was also participating in an exchange program at the time.
His collaboration and generosity in sharing his work were sincerely appreciated.

I am deeply grateful to the Faculty of Engineering and Architecture (FEA) and Ghent University for enabling me to
undertake this research in collaboration with Professor Hamdaqa.
This experience has been a highlight of my academic journey.
I extend my thanks to the International Relations Office of FEA for facilitating my study exchange and ensuring a
rewarding experience, as well as to Polytechnique Montréal for hosting me as an exchange student.

Finally, I want to acknowledge the broader society for the opportunities that allowed me to pursue higher education and
embark on this academic journey.
The privilege to engage in research and study abroad is something I deeply value.

Above all, my heartfelt gratitude goes to my mother.
Her unwavering support, encouragement, and belief in me have been the cornerstone of my achievements.
Her strength continues to inspire me every day.


\include{chapters/2-disclaimer-oral-exam.tex}
\chapter*{Abstract}
\chaptermark{Abstract}
\addcontentsline{toc}{chapter}{Abstract}  

%This chapter should contain three things.

%\begin{itemize}
%    \item A copy of all the information on the title page of your master's thesis. This includes things like the name of your master's thesis and your advisors.
%    \item A one-paragraph description of your master's thesis. This should be 15 to 20 lines long. This should include the context of your master's thesis, the problem statement of your master's thesis. The results of your master's thesis, and the evaluation of the work.
%    \item Five keywords that describe the subject best.
%\end{itemize}
%
%The chapter should be one page at most.

\textbf{Title:} Mo-Lab: Interactive model reporting and analysis in openCAESAR through natural language processing and SPARQL integration

\textbf{Author:} Thomas Decloedt
\textbf{Student Number:} 01808629
\textbf{Supervisors:} Prof. Dr. Mohammad Hamdaqa, Prof. Dr. ir. Chris Develder 
\textbf{Degree:} Master of Science in Computer Science Engineering
\textbf{Academic Year:} 2024-2025

% Context
The task of generating reports in \glsentrylong{mbse} often demands significant effort and technical expertise,
creating barriers resulting in inflexibility.
Recent advancements in \glsentryshort{ai}, particularly in \glsentrylong{nlp},
offer promising opportunities to tackle these challenges by facilitating more interactive and less technical approaches.

% Problem statement
A significant challenge in applying neural approaches to \gls{mbse} lies in the scarcity of datasets available
in these specialized domains.
This thesis addresses the issue of data scarcity primarily through domain adaptation,
which leverages data from other domains to benefit specialized \gls{mbse} contexts.
Additional techniques employed to mitigate data limitations include the use of \glsentrylong{cnl},
which reduces the training data requirements,
and \glsentrylong{rag}, which integrates unseen domain information to enhance performance and adaptability.

% Results
The experimental results demonstrate that data scarcity can be effectively addressed through \glsentrylong{da} and
\glsentrylong{cnl}.
Using high-quality synthetic data resulted in excellent performance, even on a challenging benchmark.
However, the effectiveness of \glsentrylong{rag} remains inconclusive,
largely due to limitations inherent to the benchmark used for evaluation.
Additionally, as with related approaches, challenges persist regarding the generalizability of the proposed method.
Nonetheless, the results reveal promising potential, warranting further exploration.

% Evaluation
This thesis contributes to the field of \gls{mbse} by introducing an interactive model reporting paradigm,
demonstrated through the proof-of-concept implementation, Mo-Lab.
The approach addresses the challenges of data scarcity often encountered in specialized domains by designing an
architecture tailored to these constraints.
Its evaluation on a representative domain underscores its feasibility and effectiveness.
Finally, this work demonstrates promising practical applications,
paving the way for more accessible and flexible model reporting in \gls{mbse}.

\textbf{Keywords:} Model-Based Systems Engineering, openCAESAR, text-to-SPARQL, domain adaptation, data scarcity


\IfFileExists{extended-abstract.pdf}{\includepdf[pages={1-5}]{extended-abstract.pdf}}{}
\tableofcontents\newpage
\listoffigures\newpage
\listoftables\newpage
\include{chapters/4-list-of-acronyms.tex}
\listoflistings\newpage

%
% Include the main chapters of the thesis below
% Note: it's best to avoid spaces in filenames as Latex might complain about them.
%

\mainmatter
\pagestyle{fancy} % Use header
\chapter{Introduction}
\label{s:introduction}

%% Story, i.e., motivation

% Why is this interesting?
%\TODO{Downsides and dangers of not properly documenting these complex systems similar to ML models?}
The need to clearly document complex systems is evident across domains,
one example hereof is the relatively recent innovation of model cards for \gls{ml} models
\cite{mitchellModelCardsModel2019}.
The idea behind them is to document various aspects of the \gls{ml} model, 
e.g., how it works or for what it is intended,
such that the various stakeholders can easily access relevant information.
They are meant to be complementary to datasheets for datasets \cite{gebruDatasheetsDatasets2021}
which address the concerns of two stakeholder groups:
dataset creators and consumers.
For the latter, these datasheets provide the essential information needed to make informed decisions about how to use the data effectively.
This need for information communication is not limited to the \gls{ai} community.

% Context
\gls{se} is one such domain that deals with complex systems
and could reap the benefits of improved information communication.
The aerospace sector is a pertinent example where \gls{se} is crucial,
but to deal with the increasing complexity of systems,
a particular type of \gls{se} is gaining traction: \gls{mbse}.
It deviates from the document-centric approach of traditional \gls{se},
opting instead for domain models that serve as single source of truth.

% What is reporting & why do we do it?
In \gls{mbse}, a system can be documented and presented in a way that is targeted to specific stakeholder groups,
--- also called reporting
--- adapting to their expertise and interests \cite{elaasarOpenCAESARBalancingAgility2023}.
Reports are a type of document-based presentation of the system providing specific information.
One example are information reports \cite{wagnerCAESARModelBasedApproach2020},
which are quite similar to the previously mentioned model cards and datasheets,
from an information communication point of view.
They give insight, provides analyses, and assist in decision-making throughout a system's life cycle.

% Current approach
The current approach to the reporting of models in \gls{mbse} is based around pre-defined templates called viewpoints.
It involves generating a document-based view, or representation, of the model from the perspective of a certain viewpoint.
These views have been recognized for their contribution in enhancing 
stakeholder engagement \cite{hendersonValueBenefitsModelbased2021}.
This general approach is adapted in various formal languages or tools, such as
the \gls{sysml} and the Cameo Systems Modeler tool,
the Arcadia Language and the Capella tool
or the \gls{oml} and the openCAESAR platform.
To help with report creation, tools can have report generators and include pre-defined viewpoints
which can be quite numerous in quantity \cite{wagnerCAESARModelBasedApproach2020}.
% View & viewpoint
% Viewpoint: https://docs.nomagic.com/display/SYSMLP190/Viewpoint
% View: https://docs.nomagic.com/display/SYSMLP190/View
% In other words, a view is a representation of the system from the perspective of a certain viewpoint.

% Gap in current approach
Although the current industry-standard approach is quite agile
--- reports can be tailored to specific stakeholder groups
--- it is not flexible.
Creating a new type of report or updating an existing viewpoint to address a stakeholder's interest in a previously
unaddressed aspect of the model requires a significant investment of both time and resources.
The reason for this is two-fold; in order to create or update a viewpoint both knowledge of the domain
and of the specific technology used for engineering the viewpoints is needed.
For example, the technologies used might include a formal query language such as \gls{sparql}
\cite{hofgenEnhancingModelBasedSystem2024, elaasarOpenCAESARBalancingAgility2023, wagnerCAESARModelBasedApproach2020}
and a templating language such as \gls{vtl}
\cite{lutfiIntegrationSysMLVirtual2023, hofgenEnhancingModelBasedSystem2024, traseModeldrivenVisualizationTool2014}.
Any change thus requires two highly specialized profiles: a domain expert and a technologist,
the former having the necessary knowledge of the domain in question
--- while typically not having a formal education in software engineering \cite{bialySoftwareEngineeringModelBased2017}
--- and the latter having the appropriate skills to define the queries, templates, etc.

% Our paradigm: model reporting
To address this flexibility gap, model reporting is proposed:
a no-code interactive reporting paradigm for \gls{mbse}.
It is a type of information reporting restricted to and aimed at documenting the model of the system under consideration.
Essential information about the system is conveyed to stakeholders through text, diagrams, tables, etc.
Different from the current approach to reporting, a technologist is not required.
Instead, its role is to be supplanted by an envisioned agent who takes care of the technicalities,
performing the tasks given to it by a user through interaction.

% Proof-of-concept
A proof-of-concept is put forward based around the notebook environment
from within which a user can interact with an agent
--- through natural language interaction 
--- who answers question about the model with explanations, tabular results and visualizations.
Reports can be created in this environment in an easy and quick way without help from a technologist.
For example, the need to know a formal query language is obviated and instead delegated to an agent.

% Approach
This agent takes the role of the technologist
and is responsible for generating appropriate responses to user inquiries.
In the current approach the technologist usually writes the appropriate formal language queries, for example, in \gls{sparql}.
The well-established domain of \gls{qa} might be used to replace the technologist,
since it traditionally focuses on generating answers to questions, both in natural language.
However, model reporting extends beyond text-based answers:
visualizations and access to tabular data are essential components too.
Therefore, the agent must not only generate natural language responses
but also explicitly produce queries that facilitate these additional tasks.
To achieve this, the agent leverages advancements in \gls{nlp},
particularly by harnessing the capabilities of state-of-the-art \glspl{llm},
which are increasingly being used to address complex, multi-faceted challenges in data-driven tasks.

% Problem
The key problem with explicit query construction for model reporting is the lack of data.
The domains of \gls{mbse} are often private and highly specialized.
Lack of data is of course an extremely common issue, but the above two realities make it especially troublesome.
No high-quality, sizeable and domain relevant labeled datasets can be assumed to exist.
This challenge is referred to as the ``data scarcity scenario'' throughout this work.
The primary focus of the remainder of this study is to address and overcome this issue of data scarcity,
in order to realize model reporting.

% Illustrative example
To illustrate the issue, consider the following example involving a domain called \gls{orkg},
which includes papers, their contributors, topics, and more.
Suppose a non-technical person would like a question answered which has not yet been addressed in any of the previous 
viewpoints.
Given the time-sensitivity of their inquiry, they give it a shot by directly querying the model.
They have no extensive knowledge of the query language, and thus give ChatGPT a try,
for instance, they might ask:
\mintinline[breaklines]{text}{Provide a list of papers that have utilized the Depth DDPPO model and include the links to their code?}
\cite{auerSciQAScientificQuestion2023}.
Despite providing ChatGPT with the relevant information necessary to answer the question
--- for example, \gls{rdf} triples
--- they would likely find that ChatGPT is unable to generate a sensible query.
This issue arises even though ChatGPT is knowledgeable about \gls{sparql} and has access to domain-specific information.
The challenge lies in the relative obscurity of \gls{orkg} compared to more widely-known domains
and the high complexity of the query itself.
See \Cref{listing:keyProblemExamples} for the expected answer and an example of a generated query.
Evidently, this is a rather naive approach and much more intricate solutions can be built using \glspl{pllm}.
Nevertheless, it does get to the heart of the problem:
for certain domains, generating complex queries is not trivial.
%\TODO{I forget why a paper \cite{dialloComprehensiveEvaluationNeural2024} was relevant here.}
A recent experimental study \cite{lehmannLargeLanguageModels2024} supports these intuitions by analyzing the performance
of \glspl{llm} in generating challenging queries for a specialized domain.
The findings clearly demonstrate that \gls{zsl} (i.e., prompts without examples) is entirely impractical for complex
queries.
While some generated queries may resemble the expected results,
they are never fully correct, with no exact matches observed \cite{lehmannLanguageModelsControlled2023}.
Comparable results to fine-tuning can be achieved by selecting an appropriate model and employing extensive prompt
engineering, which involves incorporating relevant examples into the prompt.

\begin{listing}[!ht]
	\inputminted{sparql}{src/listings/key-problem-example-ground-truth.sparql}
	\caption{
		Ground truth SciQA (above) and ChatGTP-generated (below) query for the same question about \glsentryshort{orkg}
		\cite{auerSciQAScientificQuestion2023}.
	}
	\label{listing:keyProblemExamples}
\end{listing}

% How is the problem addressed
Data scarcity is addressed through a three-pronged approach.
First, the generation target language is shifted from a formal query language (i.e., \gls{sparql}) to
a \gls{cnl} (i.e., \gls{squall} \cite{ferreSQUALLExpressivenessSPARQL2014}),
see \Cref{lst:squall_example} for an example.
Using \glspl{cnl} has been shown to reduce data requirements for \glspl{llm} because they are closer to natural language
\cite{lehmannLanguageModelsControlled2023}.
Second, transfer learning is leveraged through a \gls{rag} paradigm, involving:
\begin{itemize}

	\item \textbf{Initial Fine-Tuning:}
		Fine-tuning of an \gls{llm} in a data-rich environment, followed by freezing it,
		as the first phase of a sequential training strategy.

	\item \textbf{Integration in \gls{rag} Model:}
		Using the fine-tuned \gls{llm} as a generator within a \gls{rag} framework.

	\item \textbf{Retrieval and Augmentation:}
		Using a static retriever and post-retrieval processing for non-parametric information, and integrating it through
		a trainable augmentation method incorporated in the intermediate layers of the frozen \gls{llm} (soft prompting).
	
	\item \textbf{Domain Adaptation:}
		Conduct a second phase of training on the target domain, leveraging the frozen \gls{llm} for its retained
		capabilities while adapting through augmentation.

\end{itemize}
The resulting model is dubbed the ``domain expert''.
Finally, to simulate a realistic scenario with some available data,
synthetic data is generated from the limited dataset to test the effectiveness of the proposed approach.
In conclusion, data scarcity is addressed through a combination of strategies,
including \glsentryfull{da},
the use of a \glsentryfull{cnl},
the application of the \gls{rag} paradigm,
and the generation of synthetic data.

\begin{listing}[!ht]

	\mint{text}{SQUALL:}
	\mint{text}{What is the name of the author-s of PaperX?}
	\mint{text}{}

	\mint{text}{SPARQL:}
	\inputminted{sparql}{src/listings/author-of-paper.sparql}

	\caption{\glsentrylong{s2s} example.}
	\label{lst:squall_example}
\end{listing}

% Research questions
From the core problem, the following research questions are formulated:
\textbf{RQ1:}
How does the performance of the proposed domain expert compare to other approaches
across various data availability scenarios?
\textbf{RQ2:}
Does the domain expert effectively utilize the retrieved information with which it is soft-prompted,
and what factors influence the effectiveness of these prompts?
\textbf{RQ3:}
What is the impact of using synthetic data, as opposed to ground truth data,
on domain adaptation performance during the second learning phase?

The contributions of this thesis include the introduction of
a novel interactive model reporting paradigm for the \gls{mbse} community
with an accompanying proof-of-concept implementation, Mo-Lab,
that allows a previously unattainable level of flexibility.
Furthermore, an approach to the text-to-\gls{sparql} task is presented that is specifically designed for realistic data
scarce scenarios using a pipeline that incorporates the proposed domain expert.
Finally, an evaluation and comparison of the presented approach with similar methods on a representative domain 
is included.

% Recap
This work advances the state of the art in reporting for \gls{mbse} by its proposed paradigm shift,
enabled through an approach that overcomes data scarcity through domain adaptation.


\chapter{Preliminaries}

% General
A significant body of research has been dedicated to streamlining and simplifying tasks
such as writing code, documentation, reports, etc.
characterized by a repetitive, tedious or otherwise prohibitively technical nature.
In the same vein current \gls{mbse} techniques exist for generating documents such as reports from model viewpoints
\cite{delpModelBasedDocument2013}.
Viewpoints are used in \glsentrylong{se}, but in other domains as well, e.g., software engineering.
A concrete example is the physical viewpoint in the context of space systems.
This viewpoint focuses on aspects such as the physics of motion, system components, their interconnections,
and external forces.\footnote{
	Example from \url{https://web.archive.org/web/20100527225452/http://trs-new.jpl.nasa.gov/dspace/bitstream/2014/39798/1/06-1543.pdf}.
}
The output resulting from a specific viewpoint is referred to as a view;
\gls{cad} models representing physical systems are one example,
the reports used in \gls{mbse} another.
While the techniques used in \gls{mbse} to generate reports are useful,
they can be supplemented with new \gls{ai}-based approaches.

% Why this research?
Some fundamental issues arise in the proposed approaches, often at least one of two common presuppositions ought to be 
met for the proposition to be viable.
There is either an abundance of high-quality information related to the domain,
typically in the form of an extensive dataset comprising domain-specific questions and their corresponding queries.
Or, the domain is implicitly assumed to be sufficiently well-known,
for example, when prompting a \gls{pllm} has sufficiently seen information about the domain during its pre-training
such that it can be effective given a proper prompting strategy.
A contrived example can highlight the issue, take for example a domain that is private
for which no data engineering has of yet been done, hence, high-quality datasets are missing.
Clearly, when domain-specific information is not publicly available, \glspl{pllm} become ineffective.
Furthermore, the investment required to create the necessary datasets can be prohibitive,
especially if manual annotation is chosen,
as it demands a high level of domain-specific expertise from data annotators \cite{liSemanticParsingLimited2023}.
When neither of these conditions is met, this situation
--- hereafter referred to as data scarcity
--- poses a significant barrier to progress.
Addressing and overcoming this challenge is the central motivation for the research presented here,
as it has been identified as the primary obstacle in advancing model reporting.

% Rest of this section
If a method existed that could flawlessly answer questions, generate queries, and adapt effectively to any domain
--- no matter how obscure
--- then implementing model reporting would be primarily a matter of engineering rather than a conceptual challenge.
However, the reality is more complex.
The subsequent sections will explore existing methods,
highlighting their relevance while demonstrating that they are either incompatible with data-scarce scenarios
or neither straightforward to implement nor trivial to adapt.
This analysis underscores the necessity of the research presented in this work.
First, the foundational background of the core technologies underpinning this work is introduced.
Next, a cursory survey of relevant literature and industry practices for document generation in other domains is presented.
Finally, the limitations of previous approached are discussed.

\section{Relevant NLP Technologies}

% What?

This section explores key \gls{nlp} technologies that form the foundation of the research presented in this work.
The primary focus is on the following core techniques:  

\begin{itemize}

	\item \textbf{\glsentrylong{qa}}:
		A discipline focused on creating systems capable of automatically answering natural language questions posed by users.  

	\item \textbf{\glsentrylong{sp}}:
		A task that involves mapping natural language to machine-understandable representations.  

	\item \textbf{\glsentrylong{da}}:
		A field concerned with applying an \gls{ml} domain, trained on a source domain,
		to a related but different target domain.

	\item \textbf{\glsentrylong{cnl}}:
		A subset of natural languages that have restricted grammar and vocabulary, and are thus less complex.
		
	\item \textbf{\glsentrylong{rag}}:
		A technique that integrates \glsentrylong{ir} capabilities with generative \gls{ai}
		to enhance the performance of models.

\end{itemize}

% Why relevant?

These technologies are critical in addressing the research challenges related to data scarcity
and enabling model reporting in \gls{mbse} domains.
Recent \gls{qa} systems informed the general architecture for the proof-of-concept implementation of the agent.
Meanwhile, \gls{sp} played a more specialized role, informing this work on tackling text-to-\gls{sparql} task.
It proved essential for realizing the domain expert used for generating the queries necessary to create
relevant artifacts like visualizations and explanations required for model reporting.
The final three technologies all played a key role in overcoming data scarcity:
\gls{da} enabled the transfer of knowledge from well-resourced domains to the data-scarce ones of \gls{mbse},
the use of a \gls{cnl} contributed by lowering training data requirements,
and \gls{rag} aimed to enhance the performance of the domain expert by incorporating information unseen during training.

%%% Subsections

% Semantic Parsing, Question Answering and Domain Adaptation : subsection
% Controlled Natural Languages : subsection
% Retrieval Augmented Generation : subsection

% For each subsection:
% What?
% Why relevant?
% Why not straightforwardly implementable?

\subsection{Question Answering, Semantic Parsing and Domain Adaptation}
\label{s:QASPDA}

% Formal meaning representation
% Representation of meaning
% Logical form

% Intro
This section introduces three foundational technologies: \gls{qa}, \gls{sp}, and \gls{da}.
These are prioritized as they are integral to the approach developed in this work,
and their roles are deeply interconnected within the context of model reporting.
Following an overview of each technology, their relevance to the research is analyzed,
emphasizing the challenges that make their implementation in this domain non-trivial.

% What

\paragraph{QA}

Firstly, \gls{qa} is a discipline at the intersection of \gls{nlp} and \gls{ir} that studies systems concerned with
answering questions posed by humans.
Typically, the \gls{qa} approach involves three key steps:
understanding the question (i.e., \glsentrylong{nlu}),
retrieving relevant information (i.e., \glsentrylong{ir}),
and generating the answer.
The information source a \gls{qa} system can retrieve from varies widely depending on the application.
One examples would be technical support where the source of information could be a system's documentation.
Interest in \gls{qa} has surged with the advancements in deep learning technologies 
\cite{abdel-nabiDeepLearningbasedQuestion2023}.

\paragraph{SP}

Semantic parsing is a task in \gls{nlp} that involves translating natural language statements into corresponding logical
forms.
If, for example, the following question is posed: 
\mintinline[breaklines]{text}{Tell me the name of the author of PaperX?},
it could be parsed into a dependency graph, see \Cref{fig:dependencyGraph}.

\begin{figure}[h]
	\centering
	\includesvg[width=0.95\textwidth]{images/dependency-graph}
	\caption{The dependency graph of an example question.}
	\label{fig:dependencyGraph}
\end{figure}

The task is alternatively defined as converting natural language to formal representations of meaning
or meaning representations.
In any case, the resulting output of \gls{sp} is diverse and includes
constituency graphs,
dependency graphs,
database queries,
query graphs \cite{yihSemanticParsingStaged2015}, 
semantic graphs \cite{reddyLargescaleSemanticParsing2014},
lambda-calculus \cite{wongLearningSynchronousGrammars2007},
etc.
Furthermore, various natural languages are possible, and the logical forms can be task-specific,
i.e., dependent on the area of application.
Examples of application areas are machine translation and, more pertinently, \gls{qa}.
\gls{sp} has benefited from deep learning's progress \cite{shaRetrievalAugmentedKnowledgeGraph2023,
chenHiQAHierarchicalContextual2024}.

\paragraph{DA}

Domain adaptation is categorized as transductive transfer learning in the \gls{nlp} domain.
It aims to leverage data and knowledge from a source domain to the benefit of a target domain under the assumption
that source and target domain are distinct,
that the task is identical
and that some (unlabeled) target domain data is available at training time
\cite{panSurveyTransferLearning2010}.
As a concrete example, a neural dependency parser could be trained to parse English sentences
and be adapted to a less common language through \glsentrylong{da}.

%\textit{Transductive transfer learning} has been characterized \cite{panSurveyTransferLearning2010}
%as aiming to improve the learning of a target predictive function $f_T$,
%given a source $\mathcal{D}_S$ and target target domain $\mathcal{D}_T$
%with corresponding learning tasks $\mathcal{T}_S$ and $\mathcal{T}_T$, respectively.,
%where $\mathcal{D}_S \neq \mathcal{D}_T$ and $\mathcal{T}_S = \mathcal{T}_T$,
%and some unlabeled target-domain data must be available at training time.

\paragraph{Relatedness of QA and SP: Domains and Knowledge Bases}

The domains of \gls{qa} and \gls{sp} naturally converge when investigated in the context of \glsentryfullpl{kb}.
In general, a \gls{kb} is a structured repository designed to store, manage, and retrieve information efficiently.
They are widely used to represent domains and can, for example, be utilized to represent systems in \gls{mbse}.
Typically, \glspl{kb} consist of sentences stated in a particular formal language,
and information can be retrieved from the repository using another formal language called the query language.
When the data in a \gls{kb} is organized in a graph structure, it is referred to as a \gls{kg}.
A notable example is Wikidata, a \gls{kg} that serves as a central knowledge repository,
supplying structured data to Wikipedia and other Wikimedia projects.
The convergence of \gls{qa} and \gls{sp} can be illustrated using the previous question
if it is stated in the context of, for example, a particular domain consisting of scientific papers and their authors,
implemented as a \gls{kg}.
Instead of parsing the question into a constituency graph, it could then be translated into a \gls{sparql} query:
\mintinline[breaklines]{sparql}{SELECT DISTINCT ?x1 WHERE { :PaperX :author ?x2 . ?x2 :name ?x1 . }}.
Execution of the query would retrieve the relevant information needed by a \gls{qa} system to answer the question.
In this fashion a \gls{qa} system could leverage \gls{sp} techniques in order to answer a user's question about
a particular domain.
In \gls{qa}, when the answers are derived from a \gls{kb} (or a \gls{kg}),
the task is referred to as \gls{kbqa} (or \gls{kgqa}).
However, \gls{sp} is not a necessary component of a \gls{kbqa} system.

\paragraph{Relevance of DA}

Meanwhile, many \gls{sp} scenarios are domain-specific \cite{liSemanticParsingLimited2023},
meaning that for the same task variations of a semantic parser are needed if the target domains differ.
The previous question could be parsed into a different query if the domain did not consist of scientific papers,
but instead was concerned with newspapers, yet intuitively there are similarities.
Hence, adapting a \gls{sp} model from a source domain to a target domain has previously been the topic of research 
\cite{liSemanticParsingLimited2023} using transfer learning
with some work aimed focusing on using domain adaptation \cite{liDomainAdaptationSemantic2020}.
Interest in domain adaptation is driven by a desire to overcome data scarcity in domains of interest 
since contemporary neural semantic parsing techniques usually require significant annotated datasets
\cite{jiangSurveySemanticParsing2024, liSemanticParsingLimited2023, liDomainAdaptationSemantic2020}.
However, transferring a trained neural model to a new domain is inherently challenging,
primarily due to differences in meaning representations across domains \cite{liDomainAdaptationSemantic2020}.

% Why relevant?

These three technologies are central to model reporting paradigm.
%% QA
The proof-of-concept implementation of the proposed paradigm centers on openCAESAR,
a representative \gls{mbse} framework detailed in \Cref{c:proposedSolution}.
This framework employs a \gls{kg} to model systems.
As indicated earlier, in this context answering stakeholder requests entails
understanding the question,
retrieving information from the \gls{kg} through querying,
and generating the answer in the expected.
Hence, from the \gls{qa} point of view,
the proposed implementation of model reporting can be seen as a straightforward application.
As will be discussed later on, the proof-of-concept follows a \gls{kgqa} pipeline combined with further downstream
tasks to achieve the expected results such as visualizations and natural language answers.
%% SP
Now it is clear how \gls{sp} is vital: it enables those later tasks crucial to the paradigm.
Namely, \gls{sp} provides the logical forms which are needed to query the system.
Note that although \gls{qa} cane be done without \gls{sp}, as mentioned previously,
this is not the case for model reporting where an explicit logical form is required.
\Cref{s:requirements} elaborates further upon this requirement.
%% DA
Finally, overcoming data scarcity, which is ubiquitous in \gls{mbse},
will be achieved using effective \gls{da} strategies applied to the model used for \gls{sp}
(i.e., the domain expert).

% Why not straightforwardly implementable?

Although these three technologies are applicable to model reporting,
it remains a significant challenge to apply previous work because of data scarcity.
While some \gls{qa} and \gls{sp} approaches do target data-scarce conditions,
their applicability is constrained particularly when addressing larger domains,
making it hard to apply to \gls{mbse}.
This section concludes with some examples from the literature.
%% QA
A relevant \gls{qa} system is \glsentryfull{kgqan} \cite{omarUniversalQuestionAnsweringPlatform2023},
which aims to be universal and capable of answering user questions for any arbitrary \gls{kg}.
It employs a two-stage process:
a coarse stage for understanding the question
and a fine stage resulting in the final logical forms (e.g., \gls{sparql} queries).
The question understanding step uses a domain-independent \gls{lm} to generate sketches of the logical forms
(i.e. lists of triples),
while the second step finalizes them without relying on a trained model.
The platform's approach diverges from traditional \gls{sp} in an attempt to be applicable to more than one domain
without domain specific training data.
The approach results suffers fundamental limitations though as detailed in \Cref{s:kgqan}.
Many natural language utterances cannot be correctly converted into equivalent logical forms due to the
limitations of the question understanding \gls{llm} during the coarse stage,
and the reliance on heuristic methods for the fine stage.
These constraints highlight the need for alternative strategies to address the demands of model reporting.
%% SP
ProtoParser \cite{liSemanticParsingLimited2023} exemplifies a neural semantic parser designed for data-scarce conditions.
It operates in two stages: template generation and slot filling,
leveraging synthetic data generation and transfer learning to manage data scarcity.
ProtoParser’s evaluation spans three domains with predicate counts of 15, 24, and 88, respectively,
where the number of unique predicates is a measure of the size of a domain represented by a \gls{kg}.
Although ProtoParser is evaluated for new, i.e., unseen predicates,
it remains unclear how its performance scales to more complex, encompassing domains.
For instance, the domain evaluated later in this thesis involves nearly \num{7000} unique predicates
(see \Cref{s:knowledge_graph_and_benchmark}).
This raises doubts about the feasibility of applying ProtoParser to model reporting tasks,
where the diversity and scale far exceed those in ProtoParser’s tested scenarios.
ProtoParser assumes that domain knowledge
--- acquired during training on structured data, such as database schematics 
--- can generalize to unseen predicates within the same domain.
%% DA
While ProtoParser does not employ domain adaptation, other approaches do,
\cite{liDomainAdaptationSemantic2020, suCrossdomainSemanticParsing2017, rayFastDomainAdaptation2019}.
However, these methods share similar limitations:
they focus on the same domains with relatively few predicates,
between 15 and 45 predicates \cite{wangBuildingSemanticParser2015}, 
which is not representative of the scale of \glspl{kg} used in \gls{mbse}.
%Futhermore, \cite{rayFastDomainAdaptation2019} considers an alternative type of \gls{da}
%focusing on personalization rather than general scalability.
%The proposed approach adapts to user vocabulary and paraphrase variations,
%focusing on the natural language component of semantic parsing rather than the logical forms.
%This flavor of \gls{da} assumes a source domain and a slightly deviating target domain,
%making it unsuitable for handling entirely new and highly-specialized domains.

\subsection{Controlled Natural Languages}
\label{s:cnl}
% CNLs are also logical forms, i.e., database queries

% What are CNLs?  
A \glsentryfull{cnl} is a language characterized by a restricted grammar and vocabulary,
making it less complex than natural languages.
These \glspl{cnl} are designed for various purposes, such as enhancing readability for non-native speakers.  
In the context of this thesis, their primary significance lies in their applicability to \gls{nlp}.
Specifically, they are employed for their capacity to intuitively and naturally represent formal notations,
such as \gls{sparql} queries \cite{kuhnSurveyClassificationControlled2014}.  
The semantic parser introduced in this work is tailored to address the challenge of data scarcity,
and \glspl{cnl} play a supporting role in this effort.
The following sections will elucidate how leveraging a \gls{cnl} contributes to mitigating this central obstacle to
model reporting.
Details of the semantic parser can be found in \Cref{c:methodology}.

% Why relevant?
Neural semantic parsers have garnered significant attention,
particularly with advancements in \glspl{llm} and \glspl{pllm}.
These parsers typically generate logical forms such as \gls{sql} or \gls{sparql} queries.
However, alternative output formats have also been investigated.
A notable hypothesis proposed by \cite{lehmannLanguageModelsControlled2023} suggests that generating outputs in
\gls{cnl} instead of traditional logical forms can enhance performance while reducing the data requirements for the
underlying \glspl{llm}.
This highlights the relevance of \glspl{cnl} to this work:
their use has the potential to decrease the data demands of the semantic parser,
a critical factor in addressing the challenges of data scarcity.

% What is SQUALL?
A whole host of \glspl{cnl} exists, but this thesis only focuses on one: \gls{squall}.
It is a \gls{cnl} and a formal language that can be unambiguously translated or mapped into \gls{sparql}.
\gls{squall} covers all \gls{sparql} constructs,
including many of \gls{sparql} 1.1 \cite{ferreSQUALLExpressivenessSPARQL2014},
e.g., it can be used to query and update \gls{rdf} graphs.
Returning to the earlier example question:
\mintinline[breaklines]{text}{Tell me the name of the author of PaperX?},
which was translated to a \gls{sparql} query:
\mintinline[breaklines]{sparql}{SELECT DISTINCT ?x1 WHERE { :PaperX :author ?x2 . ?x2 :name ?x1 . }}.
This query can be mapped to the following equivalent \gls{squall} expression:
\mintinline[breaklines]{text}{What is the name of the author-s of PaperX?},
which is not significantly different from the original natural language question upon first sight.
However, a very particular sentence structure is used, and the vocabulary is indeed restricted.

% Why SQUALL relevant?
As introduced in \Cref{s:QASPDA}, the proof-of-concept implementation targets a domain represented by a \gls{kg},
more specifically, an \gls{rdf} graph.
\gls{rdf} graphs are queried in \gls{sparql}, hence a \gls{cnl} mappable to \gls{sparql} was needed.
It was shown \cite{lehmannLanguageModelsControlled2023} that \gls{squall} is a good choice
likely due to its remarkable similarity to natural language,
leading to relatively good performance.

% Why not straightforwardly implementable?
Although the use of \glspl{cnl} has been identified as a promising new avenue
\cite{dialloComprehensiveEvaluationNeural2024} for \gls{kgqa} and \gls{sp},
\gls{qa} implementations applicable to model reporting were not found.
Furthermore, a difficulty related to neural \gls{sp} is that \glspl{llm} generally lack extensive knowledge of the
target \glspl{cnl} due to the infrequent occurrence of \glspl{cnl} in the training data of \glspl{llm}.
Prompting these models to generate \gls{cnl} expressions seems highly non-trivial.  
Indeed, the work \cite{lehmannLanguageModelsControlled2023} that proposed this hypothesis relied on fine-tuning
\glspl{llm} using custom datasets consisting of natural language utterance/\gls{cnl}-expression pairs.  
Thus, data scarcity once again presents a significant and unresolved hurdle.

\subsection{Retrieval Augmented Generation}
\label{s:rag}

% What?
\gls{rag} models have been suggested \cite{lewisRetrievalAugmentedGenerationKnowledgeIntensive2020}
as a way to tackle knowledge-intensive \gls{nlp} tasks by incorporating non-parametric external knowledge in \glspl{lm}.
Since then numerous novel variations have proliferated, occasionally dubbed an architecture, a framework or a paradigm.
Irrespective of the particular \gls{rag} description utilized three steps and corresponding components are essential:
retrieval, augmentation and generation \cite{gaoRetrievalAugmentedGenerationLarge2024}.
The first component is responsible for retrieving knowledge relevant to the task (e.g., \gls{kgqa}).
Knowledge can be contained in many formats, including textual documents or graphs.
Augmentation refers to the process of enhancing retrieval to optimize \gls{llm} generation.
This process can be employed at different stages, including \gls{llm} pre-training, fine-tuning, or inference.  
A notable example of augmentation is iterative retrieval,
which involves progressively refining search queries based on feedback from previous results.
This approach aims to improve the search process by iteratively narrowing the focus to the most relevant information
through a dynamic feedback loop \cite{gaoRetrievalAugmentedGenerationLarge2024}.
The final component is the generator, typically involving only a generation step by the \gls{llm}.
However, post-processing the generated output can also be included to enhance results.

% Why relevant?
The reason for introducing \gls{rag} is to solve the problems associated with the lack of an
\gls{llm}'s knowledge.
One of these issues are the well-researched hallucinations, shown to have many causes,
including the inevitably limited or outdated knowledge contained in the \gls{llm}'s parameters
\cite{huangSurveyHallucinationLarge2024, gaoRetrievalAugmentedGenerationLarge2024}.
\gls{qa} is a prominent example of a task where \gls{rag} is beneficial,
but it is applied to a range of generative tasks, 
for example, dialogue response generation and machine translation \cite{liSurveyRetrievalAugmentedText2022}.
Its usefulness to \gls{qa} stems from the inclusion of external knowledge not contained in the \gls{lm}'s weights 
which can increase accuracy and credibility \cite{gaoRetrievalAugmentedGenerationLarge2024}.

% Why not straightforwardly implementable?
As before, two complications present themselves: data scarcity and domain size.  
Both factors increase the complexity of information retrieval.  
The large domain size, in particular, makes the prompting of \glspl{pllm} non-trivial.  
Thus, identifying the relevant facts from the domain and incorporating them as manageable input to a model is challenging.

\subsection{Synthesis of Relevant Technologies}

To enable effective model reporting, a \gls{qa} system framework is adopted,
with the semantic parser serving as its central component.
This parser is designed to generate queries that address questions posed by stakeholders.
Leveraging domain adaptation, the model utilizes data from data-rich domains to mitigate the challenge of data scarcity. 

Additionally, the semantic parser outputs a \gls{cnl} known as \gls{squall},
which has been shown to reduce data requirements, thereby further addressing the issue of data scarcity.
This specialized parser is referred to as the \gls{squall} expert.

Finally, the integration of the \gls{squall} expert into a \gls{rag} architecture incorporates non-parametric domain
knowledge.
This approach aims to address the inherent limitations of \glspl{llm} in handling knowledge-intensive tasks such as
\gls{qa} and \gls{sp}.
The resulting \gls{rag} model, previously referred to as the domain expert,
represents the culmination of this synthesis of technologies.

\section{Related Frameworks for Automating Reporting}

% What?
% Why relevant?
% Why not straightforwardly implementable?

Numerous paradigms and frameworks have attempted to leverage \gls{ai} to assist, automate, accelerate or simplify
the often tedious and time-consuming tasks of creating documentation, presentations, reports, etc.
These approaches span a spectrum of automation, from entirely manual to fully automated,  
with assisted generation occupying an intermediate position.  
Interest in this area has grown significantly, and with recent advancements in deep learning,
practical implementations have begun to proliferate.  
Examples include an interactive agent for generating financial reports, such as slide decks,
from natural language \cite{raviDocuBotGeneratingFinancial2021},  
report generation from medical images \cite{biswalCLARAClinicalReport2020, messinaSurveyDeepLearning2022},  
bash command generation \cite{agarwalNeurIPS2020NLC2CMD2021},  
and agent-assisted presentation generation from data science notebooks \cite{wangSlide4NCreatingPresentation2023}.  
To provide context for this work and clarify the rationale behind its design choices,  
some examples that have tackled automating reporting are discussed below.  

% Remainder
Concretely, this section explores the following:

\begin{itemize}

	\item \textbf{Reporting Systems for \gls{sql} Databases}:
		Systems that simplify querying of \gls{sql}-based databases using natural language,
		facilitating enterprise reporting and analytics, often incorporating \glspl{rag} and \gls{llm}.

	\item \textbf{\gls{ai} Agents}:
		\gls{ai} assistants that help automate or augment tasks like creating reports or presentations from computational
		notebooks, emphasizing human-in-the-loop approaches for maintaining coherence and flexibility.

	\item \textbf{\gls{kbqa} and \gls{kgqa} Systems}:
		Frameworks focusing on answering user questions through querying \glspl{kg} using \glspl{llm} and 
		retrieval techniques,
		aiming to enhance knowledge-intensive tasks with \gls{rag} paradigms and non-parametric memory.

\end{itemize}

\subsection{Reporting Systems for SQL Databases}

% What?
Systems simplifying reporting by enabling natural language querying of \gls{sql}-based databases have been previously suggested.
An end-to-end Query Enterprise Data (QED) system was proposed \cite{joshiNaturalLanguageInteractive2020} to
facilitate natural language and conversational search on large databases,
it is aimed at enterprise reporting and analytics and uses graphical, tabular, and textual reporting interfaces.
As possible future research, extending the approach to knowledge graphs was also mentioned.
Vanna is another \gls{kbqa} system whose approach also hinges on the formation of formal language
(i.e., \gls{sql}) queries explicitly allowing post-processing such as tables and charts.\footnote{
	Vanna available at \url{https://vanna.ai}.
}
It is a newer framework making use of the \gls{rag} paradigm.
Finally, FinSQL \cite{zhangFinSQLModelAgnosticLLMsbased2024} is a financial analysis framework designed for financial
professionals also based on text-to-\gls{sql}.
It addresses the technical hurdles faced by stakeholders lacking programming skills to interact with the \glspl{kb}
used in the finance sector. 
To overcome these barriers, the authors present a text-to-\gls{sql} framework for financial analysis
with interactive features
--- although these are only mentioned in passing 
such as context retention, natural language summaries, and various forms of visualization.

% Why relevant?
These reporting systems have quite similar aims as model reporting.
Key observations include:

\begin{itemize}

	\item The need for interactivity and trust in a reporting system, 
		and the use of synthetic data to combat data scarcity \cite{joshiNaturalLanguageInteractive2020}.
		To combat data scarcity in industrial settings, synthetic data is generated to train the models.

	\item The integration of \glspl{llm} and \gls{rag} forms the core foundation of Vanna's functionality.

	\item Achieving data efficient domain transfer
		(i.e., reusing base model for different databases)
		through \gls{lora} modules \cite{zhangFinSQLModelAgnosticLLMsbased2024}.

\end{itemize}

% Why not straightforwardly implementable?
However, it is important to note that FinSQL, QED and Vanna
are primarily designed for relational databases and the text-to-\gls{sql} task.
In contrast, the models used in \gls{mbse} are defined and queried differently.
For instance, \gls{sysml}/\gls{uml} models are queried using \gls{ocl}.

Similarly, openCAESAR and its associated models cannot be queried using these techniques.
For further details on querying in the context of the proof-of-concept implementation,
see \Cref{s:particularsProofOfConcept}.

\subsection{AI Agents}
% Notebooks to slides

% What?
Slide4N \cite{wangSlide4NCreatingPresentation2023} is an \gls{ai} assistant that helps data scientists to create slides
from computational notebooks.
A human-in-the-loop solution was chosen, resulting in an interactive framework, instead of going for full automation.
Downsides to full automation that where mentioned include limited choice in generation, and decreased efficiency if
generated content is not satisfactory.
The architecture consists of a front-end and a \gls{nlp} powered back-end.

% Why relevant?
Slide4N's relevancy to model reporting is clear through analogy: 
ideally an \gls{ai} assistant would help \gls{mbse} stakeholders to create reports from models.
Like Slide4N the proposed proof-of-concept also introduces a human-in-the-loop approach
instead of going for complete automation.
That way some creative control over the final model report is kept.

\subsection{Knowledge Base/Graph Question Answering}
%\TODO{SPARQL-QA-v2 system for Knowledge Base Question Answering}

% What?
Various frameworks have been suggested in the literature focused on chatting with \glspl{kg}.
Common techniques that stand out are the use of \glspl{llm} to power conversations
and nearest neighbor-based embedding retrieval techniques.
Utilizing \glspl{llm} in graph analysis has been explored \cite{pengChatGraphChatYour2024},
they submit a framework that allows chatting with for example chemical molecules or social networks.
A user's questions are answered through a dialog, and their analysis performed, including graph comparison and cleaning.
The graph operations needed to do this are implemented through consecutive \gls{api} calls, dubbed chains.
Similar to document embeddings, the calls are embedded in a space.
The user prompt is made up of an inputted text and an uploaded graph.
The first is embedded into the \gls{api} space after which the most similar calls are retrieved using an approximate
nearest neighbor search.
While the latter is first sequentialized, then using the retrieved \gls{api} calls and sequentialized graph, 
an \gls{llm} constructs a chain of calls to operate on the graph in order to achieve the user's question.
The G-Retriever chat graphing framework \cite{heGRetrieverRetrievalAugmentedGeneration2024} puts emphasis on \gls{kgqa} for
real-world graphs.
It tackles the problem using an \gls{llm} and the \gls{rag} paradigm.
Initially the vertices and edges of the target graph are embedded  using a \gls{plm} 
and stored in a nearest neighbor data structure.
When the user poses a question it is embedded by the same \gls{plm} after which the $k$ most similar 
--- based on a cosine similarity function
--- vertices and edges are retrieved and from which a connected subgraph is constructed using a
modified Prize-Collecting Steiner Tree algorithm.
The prize, i.e., value of both vertices and graphs, is determined by the cosine similarity.
The algorithm finds the connected subgraph with maximal total prize subtracted by a cost function,
dependent on the size of the subgraph.
%\TODO{Check paper, not described as reg. param.}
The cost functions act as a regularization parameter,
without it the biggest connected subgraph would always be chosen.
\glspl{llm} have limited context windows, hence this regularization helps when scaling to bigger graphs.
A final example is RAG-end2end \cite{siriwardhanaImprovingDomainAdaptation2023},
a \gls{rag} paradigm proposed to specifically handle \gls{qa} on \glspl{kb} of specialized domains.

These first two approaches are relevant to the proof-of-concept implementation as they specifically address inquiring into
\glspl{kg} using natural language while G-Retriever and RAG-end2end exploit the \gls{rag} paradigm incorporating a \gls{kg} and
\gls{kb} as non-parametric memory, respectively.
However, these approaches do not perform explicit query construction and the latter two have significant data requirements.

\subsection{Recap of Related Frameworks}

In this section, multiple frameworks were explored that were aimed at automating reporting using \gls{ai}.

Systems like Vanna and FinSQL focus on natural language
querying of \glspl{sql} databases, emphasizing \gls{llm} and \gls{rag} paradigms,
though they target relational databases, not \gls{mbse} models.
   
An assistant like Slide4N assists in generating reports from computational
notebooks using a human-in-the-loop approach, importantly maintaining some level of creative control.
   
Frameworks like G-Retriever and RAG-end2end utilize \glspl{llm} and \gls{rag} paradigms for querying \glspl{kb},
focusing on incorporating non-parametric memory.

These frameworks highlight valuable techniques, though their focus on relational databases and heavy data dependencies
pose difficulties for \gls{mbse} and its data-scarce domains.

\section{Open Challenges}
\label{s:limitations}

%\TODO{Verify above claim
%	\cite{reydAssessingGeneralizationCapabilities2023}
%	\cite{dialloComprehensiveEvaluationNeural2024}
%	Can't adapt to new templates hence neither new domains?
%}

The current state of research presents several limitations to model reporting for \gls{mbse}.

Existing approaches in \gls{qa} and \gls{sp} (see \Cref{s:QASPDA})
do not fully address the unique demands of model reporting.
The scale and nature of \gls{mbse} domains complicate the application of these solutions.

% QA, SP & DA
Many \gls{qa} and \gls{sp} approaches depend on labeled training data for the specific domains, 
and generally certain public datasets are used to evaluate them, e.g., LC-QuAD 2.0 and QALD-9-PLUS.
These systems are inapplicable when no data is available,
although \gls{da} and synthetic data generation approaches show some limited success,
this is mainly for much smaller domains than encountered in practice.

% CNL
Recent work has indicated the promising potential of \glspl{cnl} to further aid in this regard by reducing 
training data requirements of the neural models used for the \gls{sp} task.
However, current \glspl{llm} have limited or no knowledge of \glspl{cnl},
because of the scarcity of such data in their training corpora.
Fine-tuning \glspl{llm} for \gls{cnl} generation tasks still requires substantial datasets,
which are typically unavailable in specialized domains.

% RAG
\gls{rag}-based architectures show promise for \gls{qa} in specialized domains,
but in practice \gls{sp}, i.e., explicit query construction is often eschewed for answers
which can complicate or make impossible downstream post-processing tasks that are
essential to model reporting.

% Frameworks
After exploring these relevant \gls{nlp} techniques, a closer look was taken into particular frameworks
with similar goals as model reporting, including some \gls{qa} systems and semantic parsers.
Although these frameworks make use of the aforementioned techniques
--- for example, using \glspl{cnl} to lower data requirements and \gls{rag} to incorporate non-parametric memory
--- it is unclear how they can be adapted or emulated for \gls{mbse}.

% Final
The fundamental challenge for model reporting lies in the scarcity of domain-specific data,
which is inherent to \gls{mbse}.
Despite the exploration of various frameworks, none of the examined ones have successfully addressed this critical issue.
Using the presented \gls{nlp} techniques this thesis solves the data scarcity issue in \gls{mbse},
paving the path for model reporting.


\chapter{Proposed Solution}
\label{c:proposedSolution}

% Section intro
This chapter presents the proposed paradigm shift along with its proof-of-concept implementation, Mo-Lab.
The requirements for the implementation are subsequently defined.
Finally, the chapter demonstrates how the proof-of-concept satisfies these requirements through validation.
To provide a clearer context, the discussion first begins with a detailed introduction of \gls{mbse}.

\section{Introduction}

% MBSE tools
As previously stated, \gls{mbse} is a relatively recent development in \gls{se} turning away from the document-centric earlier
approaches towards a formalized model-centric one where a model serves as single source of truth instead of a variety of
sources such as text files and schematics.
Multiple tools exist that cater to this methodology, for example,
Cameo Systems Modeler, Innoslate and openCAESAR.
Hereafter, the latter will be used to showcase some general capabilities of an \gls{mbse} framework
--- given that the proof-of-concept implementation, presented later on, is targeted at the openCAESAR platform
--- however, the proposed paradigm of model reporting is meant for all of \gls{mbse}.

% openCAESAR as an example tool
At the heart of the openCAESAR platform is the \gls{oml} language used to author models.
Model authoring is supported by openCAESAR through several \gls{oml} workbenches,
with the Rosetta workbench being the most prominent
\cite{elaasarOpenCAESARBalancingAgility2023}.\footnote{
	Rosetta is an Eclipse RCP, 
	making it comparable in use to the \glsentryshortpl{ide} commonly used in software engineering.
}\footnote{
	Rosetta is available at \url{https://github.com/opencaesar/oml-rosetta}.
}

% Representativeness
A survey of 148 ontology engineering projects from academia and industry provides valuable insights into the preferred
languages for ontology development.
Notably, \gls{owl} emerged as the most commonly used language,
accounting for 30\% of the projects \cite{simperlAchievingMaturityState2009}.
Given that \gls{oml} is mapped to \gls{owl} (for details, see \Cref{s:particularsProofOfConcept}),
openCAESAR can be considered representative.
Consequently, the proof-of-concept is applicable to a significant portion of \gls{mbse},
further reinforcing its relevance.

\paragraph{Kepler16b Running Example}

The Kepler16b demo project is used as a running example throughout this chapter with examples taken from the \gls{oml}
tutorials and from the work presenting the openCAESAR platform \cite{elaasarOpenCAESARBalancingAgility2023}
as well.\footnote{
	Kepler16b demo project available at \url{https://github.com/opencaesar/kepler16b-example}.
}\footnote{
	Kepler16b demo documentation used for \Cref{fig:keplerDoc} 
	is available at \url{http://www.opencaesar.io/kepler16b-example/doc}.
}\footnote{
	\gls{oml} tutorials used for \Cref{lst:kepler16bDescriptionModelRequirement}
	is available at \url{https://www.opencaesar.io/oml-tutorials}.
}
%
Kepler16b is an exoplanet orbiting the binary star system Kepler16.
It serves as an illustrative mission and a case study for the openCAESAR platform \cite{elaasarOpenCAESARBalancingAgility2023}.
The mission comprises two spacecraft: an orbiter and a lander.
Each spacecraft is tasked with scientific missions,
such as characterizing the planet's atmosphere, environment, and gravitational field.
To achieve these objectives, the spacecraft are composed of multiple subsystems,
including electrical, thermal, telecom, mechanical, and propulsion, along with their respective components.
This example thus represents a complex system, necessitating the use of \gls{mbse} methodologies and tool support.

Five key aspects of \gls{mbse} are discussed:
system design, requirements, analysis, verification and validation.\footnote{
	Key aspects identified by The International Council on Systems Engineering (INCOSE) as reported at
	\url{https://sdincose.org/wp-content/uploads/2011/12/SEVision2020_20071003_v2_03.pdf}.
}

% System design
\paragraph{System Design}
\label{s:systemDesign}

First, a model representing the system is designed.
In openCAESAR's case, the platform enables authors to define their system of interest using a language called \gls{oml}
as the central formalism for representing domain knowledge in models
\cite{elaasarOpenCAESARBalancingAgility2023}.
\gls{oml} is elaborated upon in \Cref{s:particularsProofOfConcept}, here an example is more illustrative.
See \Cref{lst:kepler16bDescriptionModel}, it shows the Lander Mission along with its
objectives and its components.
%
System design is complex, and to aid in the process \gls{mbse} toolchains typically provide ways to visualize the
models.
An example of this is given by \Cref{fig:keplerSystemDesignSchematic},
which shows the objectives pursued by the familiar Lander Mission and a new Orbiter Mission.

\begin{listing}[!ht]

	\inputminted{text}{src/listings/kepler16b-description-model.oml}

	\caption{
		Kepler16b description model excerpt showcasing the Lander Mission,
		its objectives and its components \cite{elaasarOpenCAESARBalancingAgility2023}.
	}
	\label{lst:kepler16bDescriptionModel}
\end{listing}

\begin{figure}[h]
	\centering
	\includegraphics[width=0.95\textwidth]{images/kepler16b-lander-mission-p2-missions-description.png}
	\caption{Schematic of the objectives pursued by the Lander and Orbiter Mission.}
	\label{fig:keplerSystemDesignSchematic}
\end{figure}

% Requirements
\paragraph{System Requirements}

Defining a system's requirements is essential to \gls{se}.
%
For example, the Ground Datasystem component
--- deployed by the Lander Mission, see \Cref{lst:kepler16bDescriptionModel}
--- could be required to present a particular interface.
See \Cref{lst:kepler16bDescriptionModelRequirement} for how it can be specified in \gls{oml}.

\begin{listing}[!ht]

	\inputminted{text}{src/listings/kepler16b-description-model-requirement.oml}
	\mint{text}{}

	\begin{minted}{text}
// Inclusion of the requirement in the requirement documentation. 
> Requirement 'R.04' specifies that component 'Lander Ground Datasystem' shall present interface 'Command Out'.
	\end{minted}

	\caption{
		Kepler16b example requirement and documentation.
	}
	\label{lst:kepler16bDescriptionModelRequirement}
\end{listing}

\paragraph{System Analysis}

Broadly, \gls{mbse} tools allow for analysis of the defined system.
What the analysis entails depends on the domain of the system, the tools used, etc.
%
For example, if the defined system is a digital twin, i.e., virtual representation of a physical object
then an \gls{mbse} tool could be used to run simulations with the intent of analyzing how the system might behave
under certain conditions.
%
Specifically for openCAESAR,
analyses can be run to check for any logical inconsistencies in the model \cite{wagnerCAESARModelBasedApproach2020}.
If, for example, missions were defined to pursue objectives,
an error would be indicated upon specifying that the Lander Mission pursues the Lander Ground Datasystem component,
given that the latter is not an objective.

% Documentation and report generation
\paragraph{System Verification and Validation}

System validation by stakeholders can be effectively conducted using documentation and reports.
%
Tools can provide automatic documentation of the system as is the case in openCAESAR.
Furthermore, \gls{mbse} tools often provide graphical wizards that enable users to create reporting templates.
These templates are then utilized by the tool's report generator to produce up-to-date reports directly from the model.
Unlike traditional document-based approaches to \gls{se},
this method ensures that reports remain consistent with the model.
%
\Cref{fig:keplerDoc} presents an example of a generated system documentation in openCAESAR,
while \Cref{fig:keplerReport1}--\Cref{fig:keplerReport3} shows a report.
%
Another example is the generation of requirement documents from the model using a requirement document generator.
For the Lander Mission and its Ground Datasystem component, refer to \Cref{lst:kepler16bDescriptionModelRequirement},
which illustrates how the requirement is integrated into the requirement documentation.

\begin{figure}[h]
	\centering
	\includegraphics[width=0.75\textwidth]{images/kepler-documentation.png}
	\caption{Kepler16b documentation excerpt.}
	\label{fig:keplerDoc}
\end{figure}

% Highest level of abstraction
\section{System Description}
\label{s:systemDescription}

Now that a general impression of \gls{mbse} practices is painted,
an example follows of a current reporting approach,
then the model reporting paradigm is introduced,
followed by the proof-of-concept implementation.
Attention is paid as to how the implementation practically differs from current \gls{mbse} approaches.

\subsection{The openCAESAR Analysis Pipeline}

The openCAESAR framework supports the creation of viewpoints from \gls{oml} repositories
\cite{elaasarOpenCAESARBalancingAgility2023}.
These viewpoints are templates serving as tailored perspectives that consolidate and interpret data from the source models,
converting raw information into insights.

The openCAESAR analysis pipeline, utilized to produce these reports, consists of three key steps:
querying, reduction, and rendering.
This process is exemplified through the Kepler16b running example.

Querying, or executing a query, entails retrieving data from the model in a structured and systematic manner using a
formal language
--- specifically, \gls{sparql} in this context.
\Cref{s:particularsProofOfConcept} provides an in-depth explanation of how an \gls{oml} source model is queried.
For the purpose of this section, however, \gls{sparql} queries can be viewed as a systematic, programmatic approach to
answering questions about the system, as opposed to a manual process.

Executing a viewpoint involves three key stages:
retrieving the most up-to-date information from the source model (querying),
transforming the retrieved data (reduction),
and presenting the results as a view or report (rendering).

\subsubsection{Querying}

The starting point of the views and reports are \gls{sparql} queries.
Essentially, they serve to retrieve information from the \gls{oml} models,
as elaborated in detail in \Cref{s:particularsProofOfConcept}.
\Cref{lst:kepler_query} shows a query that aims to retrieve components and their mass,
its results are shown in \Cref{lst:kepler_results}.
Notably, the query demonstrates a high level of complexity, incorporating two \mintinline{sparql}{OPTIONAL} clauses.
These relational operations enable the query to return data from the model where applicable,
accommodating scenarios where certain information may not always be available.

\begin{listing}[!ht]
	\inputminted{sparql}{src/listings/components-query.sparql}
	\caption{Kepler16b components query \cite{elaasarOpenCAESARBalancingAgility2023}.}
	\label{lst:kepler_query}
\end{listing}

\begin{listing}[!ht]
	\inputminted{json}{src/listings/components-results.json}
	\caption{Kepler16b components query results \cite{elaasarOpenCAESARBalancingAgility2023}.}
	\label{lst:kepler_results}
\end{listing}

\subsubsection{Reduction}

Reductions transform the data into a format closer to the view.
Much can be done through querying using available transformations, for example,
relational operations such as \mintinline{sparql}{FILTER} 
or aggregation and solution modifiers such as \mintinline{sparql}{GROUP BY} and \mintinline{sparql}{COUNT}
\cite{anglesFoundationsModernQuery2017}.
Still, various reasons make the reduction step important including limitations of the query language,
and difficulty in expressing the desired manipulations relative to a more high-level programming language.

\subsubsection{Rendering}

Finally, the data can be presented to stakeholders through static documents or interactive viewers,
often published to a repository for easy access.

For example, \Cref{fig:kepler} illustrates a mass roll-up of the Orbiter Mission,
detailing the mass of all the spacecraft's components and their subcomponents.
%
A more comprehensive demonstration is provided in \Cref{fig:keplerReport1}--\Cref{fig:keplerReport3},
which showcase a complete report.

\begin{figure}[H]
	\includegraphics[width=0.95\textwidth]{images/an-interactive-mass-roll-up-visualization-in-an-opencaesar-report.png}
	\caption{Interactive mass roll-up visualization of the Orbiter Mission \cite{elaasarOpenCAESARBalancingAgility2023}.}
	\label{fig:kepler}
\end{figure}

\subsubsection{Gap}

Some key inflexible aspects of the openCAESAR analysis pipeline have now become clear:

\begin{enumerate}
	\item \textbf{Querying:}
		Writing non-trivial \gls{sparql} queries requires technical expertise.
	\item \textbf{Reduction:}
		To handle the query results, data engineering skills are necessary.
	\item \textbf{Rendering:}
		Appropriate visualization is possible through data visualization skills.
\end{enumerate}

\subsection{Paradigm}

In contrast to current practices,
model reporting is here envisioned as an \gls{ai}-assisted process centered around a notebook environment.
This approach empowers stakeholders to create reports through natural language interactions with an intelligent agent.
The agent acts as a mediator, bridging the gap between the user and the system by handling technical complexities.

This paradigm aims to enhance flexibility in reporting by freeing stakeholders from rigid, pre-defined templates.
The agent responds to stakeholder queries with a variety of outputs,
including textual explanations, tables, schematics, figures, and more.
Users can reorganize and adapt these outputs as needed, leveraging the notebook environment's text-processing features,
such as markup, to customize and structure their reports.

Rather than requiring stakeholders to construct formal language queries
or possess specialized knowledge about report generation in conventional tools and platforms,
this framework shifts the focus to a conversational process.
Through natural language prompts, the notebook environment facilitates seamless interaction with the \gls{ai}-agent.

The ideal implementation enables the agent to engage in dynamic conversations,
accommodate specific requirements for outputs, and shield stakeholders from technical intricacies,
fostering an intuitive and efficient reporting experience.

\subsection{Implementation}
\label{s:implementation}

This section gives a summary introduction on the implementation further elaborated upon in \Cref{c:methodology}.

\begin{figure}[ht]

	\centering

	\begin{tikzpicture}[scale=0.65, transform shape]
		\tikzset{node distance = 30pt and 50pt}
		
		\node[businessman,minimum size=1.5cm] (S) at (0,0) {Stakeholder};
		\node[right=of S] (N) {\includesvg[width=1.5cm]{jupyter.svg}};
		\draw[->] ([yshift=+2mm]S.east) -- ([yshift=+2mm]N.west) node[descr, midway, above] {Question};
		\draw[<-] ([yshift=-2mm]S.east) -- ([yshift=-2mm]N.west) node[descr, midway, below] {Response};

		\node[block, right=of N] (NLP) {Text-to-\glsentryshort{sparql}};
		\draw[->] ([yshift=+6pt]N.east) -- ([yshift=+6pt]NLP.west) node[descr, midway, above] {Question};

		\node[block, right=of NLP] (E) {Execution};
		\draw[->] ([yshift=+6pt]NLP.east) -- ([yshift=+6pt]E.west) node[descr, midway, above] {\glsentryshort{sparql}};

		% Downstream components

		\node[block, above right=of E] (Exp) {Explanation};
		\node[block, right=of E] (Tab) {Tabulation};
		\node[block, below right=of E] (Viz) {Visualization};
		\draw[->] ([yshift=+6pt]E.east) -- ++ (1.1,0) |-                                         ([yshift=+6pt]Exp.west);
		\draw[->] ([yshift=+6pt]E.east) --               node[descr, above, fill=white]{Results} ([yshift=+6pt]Tab.west);
		\draw[->] ([yshift=+6pt]E.east) -- ++ (1.1,0) |-                                         ([yshift=+6pt]Viz.west);

		%\node[right=of Exp, label={below:ChatGPT}] (gpt) {\includesvg[width=1.25cm]{chatgpt.svg}};
		%\draw[->] ([yshift=+2mm]Exp.east) -- ([yshift=+2mm]gpt.west) node[descr, midway, above] {prompt};
		%\draw[<-] ([yshift=-2mm]Exp.east) -- ([yshift=-2mm]gpt.west) node[descr, midway, below] {\glsentryshort{nla}};

		\node[input_output, right=of Tab] (R) {Response};
		\draw[-] (Exp.east) -- ++ (0.8,0) |- ([yshift=+6pt]R.west);
		\draw[->] ([yshift=+6pt]Tab.east) -- ([yshift=+6pt]R.west);
		\draw[-] (Viz.east) -- ++ (0.8,0) |- ([yshift=+6pt]R.west);

		\node[draw, dashed, fit=(NLP) (E) (Exp) (Tab) (Viz) (R), label={\gls{kgqa} Pipeline}, inner ysep=5pt] (P) {};

		% New node in the upper left corner of P
		\node[maninblack, minimum size=1.5cm, anchor=south east] at ([shift={(2cm, -0.5cm)}]P.north west) (A) {\glsentryshort{ai} Agent};

		\node[below=of P, label={below:KG}] (KG) {\includesvg[width=1.5cm]{kg.svg}};
		\draw[<->] (P.south) -- (KG.north);

		%\path[->] (R.south) edge[bend left=50] (N.south);

		\begin{scope}[on background layer]
			\draw[->, densely dotted, thin]
				([yshift=-6pt]R.west) -- ([yshift=-6pt]N.east)
				node[descr, below] {};
		\end{scope}

	\end{tikzpicture}
	\caption{Proof-of-concept implementation of the model reporting paradigm.}
	\label{fig:pocImplementation}
\end{figure}


Due to the inherent complexity of realizing the proposed paradigm,
the initial version simplifies the agent's functionality.
This proof-of-concept implements the agent as a straightforward and representative \gls{kgqa} pipeline,
as depicted in \Cref{fig:pocImplementation}.
The process follows these sequential steps:  

\begin{enumerate}
    \item A stakeholder submits a natural language inquiry.
	 \item The inquiry is translated into \gls{sparql} and executed against the \gls{kg}.
    \item Post-processing is applied, limited to one of the following actions:
    \begin{itemize}
        \item Execute the \gls{sparql} query and return tabulated results.
        \item Ask a \gls{pllm}, such as ChatGPT, for an explanation based on the question and query execution results.
        \item Pass the query to an available visualization engine from the literature
        	\cite{raissyaVizKGFrameworkVisualizing2021}, which can infer and produce the most suitable type of figure.
    \end{itemize}
    \item The final results are returned to the stakeholder's working environment, such as their notebook.
\end{enumerate}

\Cref{fig:implementationExample} shows an excerpt of a model report with an example of each post-processing option.

\begin{figure}[H]

	\begin{subfigure}[t]{\textwidth}
		\begin{minted}{text}
Question:
What evaluation metrics are commonly used when benchmarking models on the FSNS - Test dataset?

Natural Language Explanation:
> The evaluation metric commonly used for benchmarking models on the FSNS - Test dataset is sequence error, which measures the fraction of predictions where the entire predicted sequence does not exactly match the ground truth. It focuses on the accuracy of the full output sequence, making it particularly suited for tasks like text recognition, where exact matches are crucial.
		\end{minted}
	\end{subfigure}

	\vspace{0.5cm}

	\begin{subfigure}[t]{\textwidth}

		\begin{minted}{text}
Question:
Provide a list of research paper titles and IDs that have benchmarked models on the Penn Treebank dataset?

Tabulation:
		\end{minted}

		\begin{tabular}{p{6cm}|p{8cm}}
		\textbf{Paper} & \textbf{Paper Label} \\
		\hline
		\href{http://orkg.org/orkg/resource/R129331}{http://orkg.org/orkg/resource/R129331} & Generalizing Natural Language Analysis through Span-relation Representations \\
		\href{http://orkg.org/orkg/resource/R130817}{http://orkg.org/orkg/resource/R130817} & Direct Output Connection for a High-Rank Language Model \\
		\end{tabular}
	\end{subfigure}

	\vspace{0.5cm}

	\begin{subfigure}[t]{\textwidth}

		\begin{minted}{text}
Question:
Provide a histogram showing the mean installed capacity grouped by energy source, with data aggregated into 5-year intervals?

Visualization:
		\end{minted}
		\centering
		\includegraphics[width=0.7\textwidth]{images/reporting-example-visualization.png}
	\end{subfigure}

	\caption{An example of a model report including three questions about a scientific domain.}
	\label{fig:implementationExample}
\end{figure}

The most significant challenge in the proof-of-concept implementation was the text-to-\gls{sparql} component
(see \Cref{fig:pocImplementation}).
As a result, this component serves as the primary focus of this work.
Secondary considerations include components post-processing the results of executed \gls{sparql} queries,
for which some initial solutions, such as visualization, are proposed.
%
The goal is to assess the feasibility of the proposed paradigm
and its effectiveness in reducing technical barriers for stakeholders
(assuming a data-scarce scenario such that the assessment is as representative as possible).
%
The following chapter elaborates upon the implementation, going into depth on the notebook environment
and the \gls{kgqa} pipeline.

% Practically
\subsubsection{Shift}

The proof-of-concept illustrates an alternative analysis workflow,
addressing the inflexibility of the openCAESAR pipeline:

\begin{enumerate}

	\item \textbf{Querying:}

		The introduction of a text-to-\gls{sparql} approach eliminates the need for technical expertise.
		The \gls{ai}-agent processes natural language prompts from users,
		seamlessly translating them into queries.

	\item \textbf{Reduction:}

		Data manipulation is fully automated, relieving users of the burden of manually processing data.

	\item \textbf{Rendering:}

		Visualization is handled automatically by an external library that intuitively determines the most appropriate format
		(e.g., pie charts, bar graphs).
		Users retain control over report formatting,
		ensuring the presentation aligns with their specific goals.

\end{enumerate}

\section{Requirements}
\label{s:requirements}

This section discusses the goals that have guided the architectural and developmental choices for the proof-of-concept
implementation of the proposed paradigm presented previously (\Cref{s:systemDescription}).

\subsection{Minimize Data Requirements}

\gls{mbse} is often concerned with specialized domains and the models in question might be private.
Thus, as indicated in \Cref{s:limitations}, access to for example question-query pairs is not guaranteed.
In the absence of domain specific labeled data, many approaches proposed in the literature are not applicable.
Although a state-of-the-art approach might be performant on the domain it was trained for,
generalization to other domains is generally weak, as assessed by
\cite{reydAssessingGeneralizationCapabilities2023},
because of unseen question-query templates, unseen \glspl{uri}, etc.
Hence, the system must be designed to work for domains for which there is potentially limited or no data available.
This scenario is denoted as data scarcity or limited resource conditions.

\subsection{Maintain Simplicity}

As touched upon in the introduction, the target users are non-technical stakeholders with no specifically assumed level
of technical expertise.
Hence, the general paradigm should be easily usable
and the approach should require not require user intervention of a highly technical level,
e.g., direct editing of \gls{sparql} queries.

\subsection{Facilitate Comprehensive Report Creation}

To support the creation of comprehensive reports, a wide range of downstream processing capabilities should be enabled,
including the generation of explanations, tables, figures, and more.
This goal is best achieved through a \gls{qa} system incorporating a \glsentrylong{sp} step,
where queries are explicitly generated.
% What?
Contrast this with a naive \gls{qa} approach where, given a user question,
\glsentryfull{ir} might be employed to retrieve relevant information,
which is then appended to a prompt alongside the original question and passed to a \glsentryfull{pllm},
such as ChatGPT, for an answer.
This approach raises the critical question of how essential elements of model reporting
--- such as visualizations, e.g., the mass roll-up of the Orbiter Mission (see \Cref{fig:kepler})
--- can be achieved.
Achieving this would necessitate the \gls{pllm} to generate results in a consistent, predictable format to enable
subsequent reduction and rendering.
% Why?
However, the generative nature of \glspl{llm} introduces inherent unpredictability,
which can pose significant challenges.
Additionally, the output token limits of \glspl{llm} may act as a restricting factor.
For example, the components query for the Orbiter Mission (see \Cref{lst:kepler_query}) retrieves fewer than twenty
components, as reflected in the mass roll-up visualization.
However, if the mission had significantly more components, relying on a \gls{llm} to generate all the results would be
inefficient in terms of time, cost, and computational resources
--- particularly when a straightforward query could achieve the same outcome. 
%
In conclusion, while explicit query construction might be unnecessary when a \gls{qa} system is designed to provide
natural language responses alone, it becomes indispensable for effective model reporting.

\subsection{Enable Human Oversight}

An \gls{ai}-assisted approach is chosen over a fully automated solution for generating reports.
%
First, the output of generative models, such as \glspl{lm} used for text-to-\gls{sparql} generation,
cannot be guaranteed to adhere to certain requirements \cite{wangSlide4NCreatingPresentation2023}.
%
Second, full automation often comes at the expense of user control.
In the context of reporting, this could result in the inability to customize the structure of the generated report.
For instance, a stakeholder might prefer a strict format in which all tabular results are preceded by a concise natural
language explanation, as shown in \Cref{fig:implementationExample}.
When such preferences are not accommodated, the stakeholder must extensively revise the report's structure,
leading to inefficiencies.
Fully automated systems typically introduce these types of challenges by prioritizing automation over user control
\cite{zhengTellingStoriesComputational2022}.
%
Given these downsides, a human-in-the-loop approach that balances automation and human intervention is more suited
\cite{heerAgencyAutomationDesigning2019}.

\section{Validation}
\label{s:validation}

This chapter concludes by demonstrating how the proof-of-concept implementation satisfies the stated requirements,
setting the stage for the detailed methodology outlined in the next chapter.

First, the requirement for minimal data needs is addressed through the use of domain adaptation
and synthetic data generation techniques, ensuring the system operates effectively with limited resources.
Second, the implementation maintains accessibility by leveraging the familiar Jupyter Notebook environment
requiring no specialized skills beyond familiarity with its interface.
Third, the implementation incorporates three key post-processing features:
natural language explanation, tabulation, and visualization.
These are facilitated by the pipeline's explicit query construction.
Finally, stakeholders are empowered to structure and format reports freely using the built-in features of
Jupyter Notebook, enhancing customization.

\chapter{Methodology}
\label{c:methodology}

This chapter details the methodology adopted to develop and validate the proof-of-concept implementation for the
proposed model reporting paradigm.
It begins with an in-depth technical discussion of the approach taken,
followed by detailing how it was implemented,
and concludes with an explanation of the evaluation strategy used to assess the approach.

% Specifics to openCAESAR
Although the model reporting paradigm is proposed for all of \gls{mbse},
the implementation is scoped to the openCAESAR framework.
Hence, certain specifics of the approach are tailored to its nature,
see \Cref{s:particularsProofOfConcept}.

\section{Approach}

The following section will first detail the architecture,
afterwards the choices that determined its design are justified.
Lastly, the contributions of the approach are put forward.

\subsection{System Architecture}
\label{s:sysArch}

Before delving into the specifics of the architecture,
a high-level overview of the pipeline and its constituents is given.
Each component is then explored in detail,
with a running example introduced at the start of each section when applicable,
to illustrate its functionality in context.
The running example is rooted in a specific domain, namely \glsentryfull{orkg}, 
that treats scientific papers and authors.
Further details about \gls{orkg} are provided in \Cref{s:knowledge_graph_and_benchmark},
since it will also be used for the experimental evaluation, see \Cref{s:experiments}.
The question starting the running example is given by \Cref{lst:runningExampleQuestion}.

\begin{listing}[!ht]

	\mint{text}{Question:}
	\mint{text}{Provide a list of papers that have utilized the Depth DDPPO model and include the links to their code?}

	\caption{Running example: question.}
	\label{lst:runningExampleQuestion}
\end{listing}

\subsubsection{High Level Overview}

Circling back to the model reporting paradigm (\Cref{fig:implementationExample}),
three aspects are essential:
the notebook environment, the agent that interacts with the system,
and finally the bridge between the two.
This section details how all three were implemented in practice as
a Jupyter Notebook, a \gls{kgqa} pipeline and a Python package.

\paragraph{Notebook Environment}

Jupyter was selected as the notebook environment, given its ubiquity,
and the existence of similar \gls{ai} extensions to the platform.\footnote{
	Jupyter AI is an example of such an \gls{ai} extension, available at \url{https://github.com/jupyterlab/jupyter-ai}.
}
Besides serving as the editing environment, it also renders the returned results from the pipeline and is the source
of the final report.

% Mid level of abstraction
\paragraph{Pipeline}

The \gls{kgqa} pipeline was introduced in \Cref{s:implementation} as part of the proof-of-concept implementation,
see \Cref{fig:pocImplementation}.
A more detailed view of the text-to-\gls{sparql} component of the pipeline is illustrated in \Cref{fig:textToSparql},
which includes its subcomponents: a domain expert, a translator, and a linker.

\input{src/schematics/tikzstyles}

\begin{tikzpicture}[auto, scale=1.0]

\node (nlq) {\small Question};

\node[block, below=0.5cm of nlq] (squall) {\small Text-to-prelinked-SQUALL};
\node[block, below=0.5cm of squall] (sparql) {\small SQUALL-to-SPARQL};
\node[block, below=0.5cm of sparql] (link) {\small Linking};

\node[below=0.5cm of link] (link_out) {\small SPARQL};

\path[arrow] (nlq.south) -- (squall.north);
\path[arrow] (squall.south) -- node[midway, right] {\small 1} (sparql.north);
\path[arrow] (sparql.south) -- node[midway, right] {\small 2} (link.north);
\path[arrow] (link.south) -- node[midway, right] {\small 3} (link_out.north);

\end{tikzpicture}


Common approaches for the text-to-\gls{sparql} task frequently train an \gls{llm} on a sizeable high-quality dataset
consisting of question-\gls{sparql} pairs for the target domain.
Due to the specialized domains in \gls{mbse} and associated data scarcity, this is not realistically feasible.
Hence, the text-to-\gls{sparql} task is deconstructed in three of the five stated components: the domain expert, translator and linker components;
combined they are responsible for the task.
The domain expert is tailored to a specialized domain and can generate a query for an input question.
Instead of \gls{sparql}, the target language is \gls{squall}, which was introduced in \Cref{s:cnl}.  
For further justifications behind choosing this \gls{cnl}, refer to \Cref{s:SQUALLOverSPARQL}.
However, translation from \gls{squall} to \gls{sparql} is needed in order to query the system.
Furthermore, the domain expert's output does not contain explicit identifiers from the target domain,
relying on textual placeholders instead.
For example, where a normal query targeted at Wikidata might contain an identifier such as \mintinline{sparql}{wd:Q30642},
output of the domain expert would contain the following placeholder: \mintinline{text}{natural language processing}.
The linker component stands in for associating the placeholders to identifiers.
To recap, for any given question the domain expert generates equivalent \gls{squall} containing placeholders,
which is mapped to \gls{sparql} by the translator and finally associated to the target \gls{kg} by the linker,
see \Cref{fig:textToSparql}.
Hereafter, assuming semantic correctness, a query can be executed, and its results handled by the ultimate component.
Practically, in case a user chose for visualization the execution of the query is also handled by the visualization engine.
For tabular results only execution is necessary.
A natural language answer can be generated by prompting a \gls{pllm} to answer the input question given the query results.
To that end, the prompt is constructed from the input question, task description and limited query results based
on context window size.

The order of the translator and linker as well as the choice of using placeholders instead of identifiers
are explained in \Cref{s:order_translator_and_linker} and \Cref{s:why_placeholders}, respectively.

% Low level of abstraction
\paragraph{Domain Expert}

The domain expert’s role involves generating \gls{squall} queries that are specifically ``tailored'' to the target domain.
While many syntactically correct queries can be equivalent in some sense for a given query language and input question,
only those appropriately tailored to the target domain would be relevant, matching the structure of the target \gls{kg},
i.e., those queries that are semantically correct.
This is essential in order to have executable queries.
The core issue of the data-scarce scenario then becomes clear:
it is not possible to reuse a model-centric and data-driven approach tailored to an original domain
for a distinct new one without any alterations, generated queries would not be semantically correct.
The idea presented here to overcome this problem is a type of transfer learning where knowledge gained in the 
original domain is transferred to the target domain.
Achieving this tailoring to the target domain and overcoming the data-scarce scenario.
An \gls{llm} is trained in the original domain on text-to-\gls{squall} task,
from this point onward it is referred to as the ``\gls{squall} expert''.
The knowledge it gained is leveraged by incorporating it in a \gls{rag} model as the generator component.
This final model is named the domain expert, since it has access to non-parametric domain information.
The training regime can be described as sequential training \cite{fanSurveyRAGMeeting2024},
where the generator is first fine-tuned on a specific task,
then frozen for further training of the \gls{rag} model.
Freezing the \gls{squall} expert means preventing catastrophic forgetting of its knowledge.
The retriever approach is straightforward,
and post-retrieval processing \cite{gaoRetrievalAugmentedGenerationLarge2024} is static,
neither require training.
The augmentation process does retrieval only once, supplying structured data to the generator component
(\gls{squall} expert) \cite{gaoRetrievalAugmentedGenerationLarge2024}.
Currently, only the augmentation process stands in for adaption to the target domain during the second phase of 
sequential training.
In conclusion, the domain expert is a \gls{rag} architecture consisting of
a generator separately fine-tuned during an initial training phase
--- dubbed the ``\gls{squall} expert''
--- a static retriever with post-retrieval processing,
and an augmentation method optimized during the second round of sequential training,
see \Cref{fig:domain_expert}.

\begin{figure}[ht!]
	\centering
	\begin{tikzpicture}[scale=1.0, transform shape]
		\tikzset{node distance = 30pt and 30pt}

		\node[draw, ellipse, minimum height=30pt, minimum width=90pt] (Q) {Question};
		\node[right=of Q, minimum height=30pt, minimum width=90pt, label={above:KG}] (KG) {};
		\node (kg) at (KG.center) {\includesvg[height=30pt]{kg.svg}};
		\node[draw, below=50pt of KG, minimum height=30pt, minimum width=90pt] (I) {Indexing};

		\node[below=50pt of Q, minimum height=30pt, minimum width=90pt] (n1) {};
		\node[below=of n1, minimum height=30pt, minimum width=90pt] (n2) {};
		%\node[below=of I, minimum height=30pt, minimum width=90pt] (n3) {};
		%\node[draw,fill=white, fit=(n2) (n3), inner sep=0pt] (R) {};
		%\node[fill=white] (n5) at (R.center) {Retrieval};

		\node[draw, text centered, below=of I, minimum height=30pt, minimum width=90pt, text width=80pt] (R) {Retrieval};

		\node[below=of n2, minimum height=30pt, minimum width=90pt] (n6) {};
		\node[below=of n6, minimum height=30pt, minimum width=90pt] (n10) {};
		\node[below=of n10, minimum height=30pt, minimum width=90pt] (n11) {};
		\node[draw, below=of n11, minimum height=30pt, minimum width=90pt, fill=blue!20] (E) {Text Embedder};

		\node[draw, text centered, below=of R, minimum height=30pt, minimum width=90pt, text width=80pt] (PR) {Post-Retrieval};
		\node[draw, below=of PR, minimum height=30pt, minimum width=90pt, fill=red!20] (GE) {Graph Embedder};
		\node[draw, below=of GE, minimum height=30pt, minimum width=90pt, fill=red!20] (P) {Projector};

		\node[below=of E, minimum height=30pt, minimum width=90pt] (n7) {};
		\node[below=of P, minimum height=30pt, minimum width=90pt] (n8) {};
		\node[draw, below=of n8, minimum height=30pt, minimum width=90pt] (n12) {};
		\node[draw, fit=(n7) (n12), inner sep=0pt, fill=blue!20] (SAL) {};
		\node (sal) at (SAL.center) {Self-Attention Layers};

		\node[draw, ellipse, below=50pt of SAL, minimum height=30pt, minimum width=90pt] (O) {\glsentryshort{squall}};

		\draw[->] (Q) -- (E) {};
		\draw[->] (Q) |- (R) node[midway, fill=white] {1};
		%\draw[->] (n2) -- (E) node[midway, fill=white] {1};
		\draw[->] (E) -- (n7) node[midway, fill=white] {2};
		\draw[->] (KG) -- (I) {};
		\draw[->] (I) -- (R) node[midway, fill=white] {3};
		\draw[->] (R) -- (PR) node[midway, fill=white] {4};
		\draw[->] (PR) -- (GE) node[midway, fill=white] {5};
		\draw[->] (GE) -- (P) node[midway, fill=white] {6};
		\draw[->] (P) -- (n12) node[midway, fill=white] {7};
		\draw[->] (SAL) -- (O) {};

		\node[draw, fit=(E) (SAL), densely dotted, very thin, inner sep=5pt,
		label={[xshift=20pt, yshift=0pt]below left:\scriptsize \glsentryshort{squall} Expert (Generator)}] (SE) {};

		\node[draw, fit=(GE) (P), densely dotted, very thin, inner sep=5pt,
		label={[xshift=45pt, yshift=0pt]above left:\scriptsize Augmentation}] (AP) {};

		\begin{scope}[on background layer]
			\node[draw, fit=(I) (R) (PR) (AP), dash dot dot, very thin, inner sep=5pt,
			label={[yshift=-120pt, rotate=90]above left:\scriptsize Learned Soft Graph Prompt}] (LSGP) {};
		\end{scope}

		\node[draw, fit=(LSGP) (R) (SE), dashed, very thin, inner sep=15pt,
		label={[xshift=-100pt, yshift=0pt]above:\footnotesize Domain Expert}] (DE) {};

		% Legend using a matrix
		\matrix [fill=white, matrix of nodes, nodes={anchor=west}, column sep=0.5cm, row sep=0.2cm, right=of DE, font=\small] {
			1 & Question \\
			2 & Text tokens \\
			3 & Vertex and edge embeddings \\
			4 & Top-$k$ vertices and edges \\
			5 & Minimum connected subgraph \\
			6 & Graph embedding \\
			7 & Aligned graph token \\
			\node[align=left, circle, minimum height=5pt, minimum width=5pt, fill=blue!20] (First) {};
				& \node[align=left, text width=145pt] {First phase of sequential training:\\
					{\footnotesize The \glsentryshort{squall} expert is independently fine-tuned on the source domain.}
				}; \\
			\node[align=left, circle, minimum height=5pt, minimum width=5pt, fill=red!20] (Second) {};
				& \node[align=left, text width=145pt] {Second phase of sequential training:\\
					{\footnotesize The domain expert is trained on the target domain, only the augmentation process is tuned,
					the \glsentryshort{squall} expert remains frozen, and other components are static.}
				}; \\
		};


	\end{tikzpicture}
	\caption{Domain Expert.}
	\label{fig:domain_expert}
\end{figure}



\paragraph{Retrieval}

Static retrieval consist of finding the most appropriate vertices and edges of the \gls{kg} for a given question.
This is achieved through textual similarity between the question and \mintinline{sparql}{rdfs:label} of all vertices and edges.

\paragraph{Post-Retrieval Processing}

The retrieved information is used to find a question-relevant subgraph of the \gls{kg} (structured data).
First, the sets of appropriate vertices and edges are ranked based on the similarity score.
Then they are assigned a certain value, called prize, based on their rank.
Using the \gls{pcst} algorithm,
a connected subgraph of the \gls{kg} can be found that both maximizes the total prize,
while minimizing the total cost, which is related to its size.

\paragraph{Augmentation}

The augmentation process first embeds the graph resulting from post-retrieval processing into an embedding
followed by ``aligning'' it with the \gls{squall} expert, i.e., with the \gls{llm}'s text embedding space.

\paragraph{SQUALL Expert}
\label{s:squallExpert}

The augmentation process employed to support the \gls{squall} expert is atypical in its approach,
as it integrates with the intermediate layers of the model \cite{fanSurveyRAGMeeting2024}.
Specifically, retrieved data is utilized at the self-attention layers of the \gls{llm},
rather than at the input or output layers
(i.e., input-layer or output-layer integration, respectively) \cite{gaoRetrievalAugmentedGenerationLarge2024}.
% Example
As a concrete example, a graph resulting from post-retrieval processing can contain over \num{300000} tokens.
%
Furthermore, since embeddings instead of text are used, other techniques can be leveraged.
If input-layer integration was used, the question-relevant graph would need to be linearized,
e.g., using a tabular overview of its vertices and edges.
The idea to use intermediate-layer integration for graph learning tasks was recently put forward \cite{liuCanWeSoft2024}.
This leads to an approach dubbed soft graph prompting,
and has the advantage of having the ability to pass the retrieved graph
through a graph embedder and directly using that advantageous representation instead of a linearized representation,
which results in loss of the graph's topological information.
The hypothesis that soft graph prompting can aid \gls{llm} generation for graph learning tasks was further investigated
by \cite{heGRetrieverRetrievalAugmentedGeneration2024} on a \gls{qa} task.
%
A notable distinction between the prior works \cite{liuCanWeSoft2024, heGRetrieverRetrievalAugmentedGeneration2024}
and the approach presented here lies in the following:
the \gls{squall} expert undergoing soft graph prompting has already been fine-tuned for a specific task.
%
It is important to note that the level of access required for implementing soft graph prompting is unavailable for most
\glspl{llm} accessed via inference \glspl{api} \cite{fanSurveyRAGMeeting2024}.
These models operate largely as black boxes:
users provide input and receive output, but the internal processes remain inaccessible.
Prominent examples, such as ChatGPT, Gemini, and Claude, do not support this type of prompting.

% Recap: Proof-of-Concept --> OML ontolgies --> KG --> RDF graph
\subsubsection{Particulars of the Proof-of-Concept}
\label{s:particularsProofOfConcept}

First, a short note on the particulars of the proof-of-concept.
As briefly touched upon in \Cref{s:systemDesign}, the proof-of-concept implementation of the model reporting paradigm
is tailored to the openCAESAR framework, a representative \gls{mbse} approach.
Like other \gls{mbse} approaches, the openCAESAR framework uses models to represent systems.
Distinct to openCAESAR, these models are defined using the \glsentrylong{oml},
a formal language inspired by \gls{owl2} and \gls{swrl}.

Looking again at Kepler16b,
an example of how a relation between its missions and their objectives could be modeled in \gls{oml}
is given by \Cref{lst:kepler16bVocabularyModel}.

\begin{listing}[!ht]
	\inputminted{text}{src/listings/kepler16b-vocabulary-model.oml}
	\caption{Kepler16b vocabulary model excerpt \cite{elaasarOpenCAESARBalancingAgility2023}.}
	\label{lst:kepler16bVocabularyModel}
\end{listing}

Like \gls{owl2}, \gls{oml} is used for authoring ontologies, but it is specifically tailored for \gls{se}.
% Ontology
An ontology is essentially a formal representation of knowledge about some domain, in this context a system.
% Implications ontology use
It is through \gls{oml} that openCAESAR introduces the ontological approach to \gls{mbse}.
This has many benefits for model authoring, e.g., enabling the use of a logical reasoner
that can check whether there are any logical contradictions present in the \gls{oml} model
\cite{elaasarOpenCAESARBalancingAgility2023}.

% Model authoring not the same as information retrieval
Answering questions or retrieving information about the model necessitates a different method: querying.
\gls{oml} constructs (such as \mintinline{text}{relation}) can be translated into patterns represented within subsets of
\gls{owl2} and \gls{swrl}.
Consequently, \gls{oml} ontologies can effectively be converted into \gls{owl} ontologies.

% Endpoints
Various tools can be used to query \gls{owl} ontologies in \gls{sparql} such as the \gls{owl} ontology editor
Protégé.
Alternatively, a so-called \gls{sparql} endpoint can be set up using platforms such as Apache Jena,
which only take the ontology and then allow users to query it using \gls{sparql}.

% RDF
Regardless of the approach chosen,
the ontology is typically stored in the \gls{rdf} format (discussed further below).
Internally, what is queried against is not a set of textual documents wherein the ontology is defined,
but rather a graph constructed from the \gls{rdf} format, i.e., the \gls{rdf} graph.

% OML --> OWL --> RDF
Thus, after model authoring the openCAESAR framework makes the model queryable
by mapping it to an \gls{owl} ontology and subsequently setting up a \gls{sparql} endpoint.
Note that it is necessary to repeat the process if the model is changed in order for the endpoint to be up-to-date.

% KG
Although in this proof-of-concept a model is ultimately made queryable through an \gls{rdf} graph,
this is just one type of graph data format and alternatives exist.
Neo4j, for example, uses property graphs instead of \gls{rdf} graphs.
Generally, this concept of using a graph of data to represent knowledge is called a \glsentryfull{kg}
\cite{hoganKnowledgeGraphs2022}.

\paragraph{Ontologies and Statements in Knowledge Bases}

The statements in any \gls{kb}/\gls{kg} can be categorized as either belonging to its \gls{tbox} or \gls{abox}.
The \gls{tbox} contains statements describing the domain of interest,
while the \gls{abox} contains ground statements, i.e., facts associated with, and compliant to, the \gls{tbox}.
For example, in the Kepler16b context, a \gls{tbox} statement might be, 
\mintinline[breaklines]{text}{"Every objective is pursued by a mission,"}
while a corresponding \gls{abox} statement could be,
\mintinline[breaklines]{text}{"The Lander Mission is a mission."}
%
If, for example, the \gls{tbox} statements are related with object-oriented classes,
the \gls{abox} statements would be instances of those classes.
%
Given the above highlighted relation between an ontology and a \gls{kb},
this distinction can also be made in the ontology,
both \gls{owl} and \gls{oml} ontologies do this.
However, the distinction is not set in stone, different ontologies categorize differently.
%
In openCAESAR,
the \gls{tbox} and \gls{abox} are referred to as the vocabulary and description models,
and they are similar to \gls{owl}'s \gls{tbox} and \gls{abox},
respectively \cite{elaasarOpenCAESARBalancingAgility2023}.
%
The Kepler16b system design example of \Cref{lst:kepler16bDescriptionModel} showed a part of the description model,
see \Cref{lst:kepler16bVocabularyModel} for its related vocabulary model.

\subsubsection{Formalization}

% Intro
The formalization presented in this section establishes the foundation for the subsequent discussions.
As outlined in the introduction,
the proposed approach is specifically tailored to models that can be represented as \glspl{kg},
implemented as \gls{rdf} graphs.
The graph formalisms introduced here are particularly relevant to \Cref{s:postRetrieval},
which describes an algorithm designed to identify subgraphs relevant to specific questions.
The aim is to use standardized terminology, fostering consistency and clarity across the methodology.
Established concepts from the literature are leveraged to ensure alignment with existing conventions 
and to improve comprehensibility.
In particular, the definitions provided by \cite{zouGraphBasedRDFData2017}
and the notations introduced by \cite{thakkarIntegratedGraphAlgebra2019} are adopted.

% Next
First, the \gls{rdf} format is first introduced in order to gain insight on how a model in openCAESAR is actually
represented when it is queried, which is crucial for retrieval and post-retrieval,
see \Cref{s:retrieval} and \Cref{s:postRetrieval}, respectively.
Basic concepts from the \gls{rdf} framework are introduced, for example,
\glspl{uri}, \glspl{bnode} and literals, which are important for linking (\Cref{s:linking}).

% Introduce RDF framework
Beginning then with the \gls{rdf} framework which represents data as semantic triples,
such as \emph{"model reporting is flexible."} 
Each triple adheres to the structure \emph{"subject predicate object"}, where:  
\begin{itemize}
	\item \textbf{Subject} identifies the resource being described (\emph{model reporting}).
    \item \textbf{Predicate} specifies the property or relationship (\emph{is}).
    \item \textbf{Object} provides the value or linked resource (\emph{flexible}).
\end{itemize}

These triples collectively form a graph, as illustrated in \Cref{fig:rdfGraph}.\footnote{
	Example from a W3C working draft about RDF available at
	\url{https://www.w3.org/2001/sw/RDFCore/TR/WD-rdf-concepts-20030117}.
}

\begin{figure}[h!]
	\centering
	\begin{tikzpicture}[
    node/.style={draw, rounded corners, align=center, minimum width=2.5cm, minimum height=1cm},
    edge/.style={->, thick},
    scale=0.6, transform shape
]

% Nodes
\node[node, label=above:{$v_1$}] (v1) {\small \texttt{http://www.example.org/staffid/85740}};
\node[node, below=2cm of v1, label=below:{$v_2$}] (v2) {};

\node[node,  below left=2cm and 3cm of v2, label=below:{$v_3$}] (v3) {'Bedford'};
\node[node,  below left=5cm and 0cm of v2, label=below:{$v_4$}] (v4) {'1501 Grant Avenue'};
\node[node,  below right=5cm and 0cm of v2, label=below:{$v_5$}] (v5) {'01730'};
\node[node,  below right=2cm and 3cm of v2, label=below:{$v_6$}] (v6) {'Massachusetts'};

% Edges
\draw[edge] (v1) -- node[midway, left] {\small \texttt{http://www.example.org/city}} (v2);
\draw[edge] (v2) -- node[pos=0.85, left] {\small \texttt{http://www.example.org/address}} (v4);
\draw[edge] (v2) -- node[midway, left=0.2cm] {\small \texttt{http://www.example.org/street}} (v3);
\draw[edge] (v2) -- node[pos=0.85, right] {\small \texttt{http://www.example.org/postalCode}} (v5);
\draw[edge] (v2) -- node[midway, right=0.2cm] {\small \texttt{http://www.example.org/state}} (v6);

\end{tikzpicture}

	\caption{
		Example \glsentryshort{rdf} graph, representing a staff member and relationships to its associated properties
		such as city, address, street, postal code, and state.
	}
	\label{fig:rdfGraph}
\end{figure}

The figure highlights three fundamental \gls{rdf} terms:  
\begin{itemize}
    \item \textbf{\gls{uri}}: A globally unique identifier representing a resource.
	 \item \textbf{\gls{bnode}}: An unnamed resource or existential variable.
    \item \textbf{Literal}: A value such as a string, number, or date.
\end{itemize}

In \gls{rdf}, resources can act as classes, which define categories or types;
entities, which are specific instances of those categories;
or properties, which describe relationships or attributes connecting entities or values.

Formally these notions can be defined as follows.

\begin{definition}
	\label{def:rdfDataset}
	An \textit{\gls{rdf} dataset } is a set $\mathscr{D}$ of triples
	$t = (\text{subject}, \text{property}, \text{object}) \in (I \cup B) \times I \times (I \cup B \cup L)$.\footnote{
		Triples are sequences of three elements.
	}
	The pairwise disjoint infinite sets $I$, $B$ and $L$ indicate \glspl{uri},
	\glsentryfullpl{bnode} (or vertices) and literals.

\end{definition}

Before querying an \gls{oml} ontology is mapped to an \gls{owl} ontology and stored in an \gls{rdf} dataset.

\begin{definition}
	An \textit{\gls{rdf} graph} is a graph $G = (V, L_V, E, L_E)$
	where 

	\begin{defitemize}

		\item 
			$V = V_c \cup V_e \cup V_b \cup V_l $ is a set of vertices consisting of all subjects and objects in the
			\gls{rdf} data, with $V_c$, $V_e$, $V_b$ and $V_l$ being the set of class, entity, blank and literal vertices,
			respectively.

		\item $L_V$ is a set of vertex labels, for a vertex $v \in V_c \cup V_e$, $v \in V_b$ and $v \in V_l$ 
			their labels correspond to \glspl{uri}, \mintinline{text}{NULL} values and literals, respectively.

		\item $E \subseteq V \times V $ is a set of directed edges, and

		\item $L_E$ is a set of edge labels, for an edge $e \in E$, its label is its property.

	\end{defitemize}

\end{definition}

Applying to the example graph (\Cref{fig:rdfGraph}) with vertices $V = V_c \cup V_e \cup V_b \cup V_l $: \\
$V_c = \emptyset, \quad V_e = \Set{ v_1 }, \quad V_b = \Set{ v_2 }, \text{and} \quad V_l = \Set{ v_3, v_4, v_5, v_6 }$.
	
Querying is done against the \gls{rdf} graph representation of an ontology,
though any user is naturally shielded from any graph-related concepts and technical details,
as these are abstracted away (see \Cref{s:particularsProofOfConcept}).

However, a foundational understanding of elementary graph theory is essential to clearly explain the proposed approach
in the sections that follow.
Therefore, these concepts are introduced upfront.
While the following formalizations might often be avoided,
they are considered to serve an important role in aligning with a theoretical framework that promotes academic rigor and,
crucially, ensures consistency. 
For instance, the domain expert could be introduced solely in terms of generated queries,
but this approach would not seamlessly align with the other components.
See, for example, the discussion of the post-retrieval component in \Cref{s:postRetrieval},
it outputs graphs.
However, since queries are inherently related to graphs, as will be elaborated upon,
structuring the domain expert's discussion around graphs ensures a cohesive and unified narrative.

\begin{definition}
	A \textit{subgraph} \( H \) of a graph \( G \) is defined as a graph such that:
	\[
		H = (V_H, E_H), \quad \text{ where } V_H \subseteq V \text{ and } E_H \subseteq E.
	\]
	In this notation, \( V_H \) and \( E_H \) represent the sets of vertices and edges of the subgraph \( H \), respectively.
\end{definition}

\begin{definition}
	Two vertices \( v \) and \( w \) in a graph \( G \) are said to be \textit{adjacent}
	if there is an edge connecting them:
	\[
		v \text{ and } w \text{ are adjacent} \iff (v, w) \in E \text{ or } (w, v) \in E.
	\]
\end{definition}

\begin{definition}
	A \textit{path} $p$ in a graph $G$ is a finite sequence of vertices:
	\[
		\begin{aligned}
									& p = \Set{ v_i }_{i=0}^n \in E^*, \quad n < \infty \\
			\text{s.t.} \quad & \forall i, j \in \Set{ 0, 1, \ldots, n }: i \neq j \land v_i \neq v_j 
						\implies v_i \text{ and } v_j \text{ are adjacent},
		\end{aligned}
	\]
	where $^*$ is the Kleene star operation.\footnote{
		The Kleene star operation $^*$ constructs the free monoid $E^* = \bigcup_{n=0}^{\infty} E^n$,
		where $E^0 = \Set{ \varepsilon }$ and $\varepsilon$ is the identity (or empty) element
		\cite{thakkarIntegratedGraphAlgebra2019}.
	}
\end{definition}

\begin{definition}
	A graph $G$ is said to be \textit{connected} when there exists a path in between every pair of its vertices joining 
	the pair:
	\[
		G \text{ is connected } \iff \forall v, w \in V, \; v \neq w, \; \exists \; p \in E^*: v, w \in p
	\]
\end{definition}

\paragraph{Connected Subgraphs}
Without delving further into graph theory, a shorthand notation can be introduced to represent all possible connected
subgraphs of a graph $G$ using the concept of the hyperspace graph of connected subgraphs $ \mathscr{C}(G) $,
which is defined for any connected graph \cite{likincsimonromeroUniquenessHyperspaceGraph2007, likincsimonromeroHyperspaceGraphConnected2005}.
Relevant here is that any connected subgraph $H$ of $G$ is an element of the vertex set of $\mathscr{C}(G)$
by definition:
\[
	V_{\mathscr{C}(G)} = \{ H; H \in \mathscr{C} \land H \subseteq G \},
\]
where $\mathscr{C}$ is the set of all connected graphs.

\paragraph{Vertex and Edge Similarity}

The similarity between a string and a vertex or edge of a graph $G$ is defined using an arbitrary encoder \gls{lm} $\lambda$:
\[
	\lambda: \Sigma^* \rightarrow X
\]
where $\Sigma^*$ is the set of all strings over some alphabet $\Sigma$,
and $X$ is an arbitrary set, e.g., $\mathbb{R}^n$.
By a slight abuse of notation, vertices and edges can also be used as input to $\lambda$,
with the implicit assumption that with each vertex or edge a string can be associated.
In the case that $G$ is a \gls{kg}, that string is a vertex's or edge's metadata property \mintinline{sparql}{rdfs:label},
this attribute being the human-readable name of the resource.
A set of graph embeddings is referred to as an index $ \mathscr{I} $.
If all embeddings result from one graph $ G $, e.g., a \gls{kg},
the graph's index is defined as follows:
\begin{align}
	\mathscr{I}_G &= \Set{ \lambda(x) \mid x \in V \cup E }.
	\label{eq:index}
\end{align}
A function $\sigma$ is used to assess the similarity between embeddings, denoted by $\sigma(x, y)$.
For convenience, this notation is also extended to allow strings, vertices, and edges as inputs,
with the understanding that the similarity function implicitly operates on their embeddings.
Specifically, define:
\begin{align}
	\sigma(s, x) \coloneqq \sigma(\lambda(s), \lambda(x)), 
	\quad s \in \Sigma^*, \; x \in V \cup E,
	\label{eq:similarityFunction}
\end{align}
The specific similarity function $\sigma$ used in the experiments is detailed in \Cref{s:settings_retrieval}.

\begin{definition}
	A \textit{\glsentryfull{bgp}} is defined as graph $Q = (V_Q, E_Q) \in \mathscr{C}$ s.t.:
	\begin{defitemize}

		\item 
			$V_Q \subseteq I \cup L \cup V_{\text{var}}$ is a set of vertices,
			with $I$, $L$ and $V_{\text{var}}$ being the of \glspl{uri}, literals and variables, respectively.

		\item 
			$E_Q \subseteq V_Q \times V_Q $ is a set of edges

		\item An edge in $E_Q$ either has an edge label in $I$, i.e., property, or is a variable.
		
	\end{defitemize}
\end{definition}

An example of a \gls{bgp} is illustrated by \Cref{fig:sparqlBGP}.

\begin{figure}[!ht]
	\centering

	\begin{subfigure}[t]{0.45\textwidth}
		\centering
		\vspace{0pt}
		\inputminted{sparql}{src/listings/bgp-pattern.sparql}
	\end{subfigure}
	\hfill
	\begin{subfigure}[t]{0.45\textwidth}
		\centering
		\vspace{0pt}
		\centering
		\begin{tikzpicture}[
    node distance=2cm and 3cm,
    node/.style={draw, circle, align=center},
    edge/.style={->, thick},
    slopedlabel/.style={midway, sloped, above, text=black},
]

% Nodes
\node[node] (x) at (0, 0) {?x};
\node[node] (y1) at (-3, -3) {?y1};
\node[node] (y2) at (0, -3) {?y2};
\node[node] (y3) at (3, -3) {?y3};

% Edges
\draw[edge] (x) -- node[slopedlabel] {likes} (y1);
\draw[edge] (x) -- node[slopedlabel] {likes} (y2);
\draw[edge] (x) -- node[slopedlabel] {follows} (y3);

\end{tikzpicture}

	\end{subfigure}

	\caption{Example of \glsentrylong{bgp} with a star shape \cite{schatzleS2RDFRDFQuerying2016}.}
	\label{fig:sparqlBGP}
\end{figure}

\glspl{bgp} are the starting point of \gls{sparql} queries which are used to inquire into \gls{rdf} graphs.
To formalize the entirety of the \gls{sparql} language \glspl{cgp} are needed, 
which are in essence \glspl{bgp} with additional operations including projections and unions,
e.g., the \mintinline{sparql}{FILTER} \cite{thakkarIntegratedGraphAlgebra2019}.
Hereafter, they are both mentioned as \glspl{gp}.

Presented here is the \glsentryfull{pgp}, a graph similar to the \glsentryfull{gp}.
However, it uses string literals, referred to as placeholders, instead of \glspl{uri}.  

The \gls{bgp} example shown in \Cref{fig:sparqlBGP} is in fact a \gls{pgp}, given that instead of \glspl{uri},
strings are used for the edges.
In contrast to the previous example of \Cref{fig:rdfGraph}.

\begin{definition}
	A \textit{\glsentryfull{pgp}} is defined as a graph $\hat{Q} = (V_{\hat{Q}}, E_{\hat{Q}}) \in \mathscr{C}$ s.t.:
	\begin{defitemize}
		\item 
			$V_{\hat{Q}} \subseteq P \cup L \cup V_{var}$ is a set of vertices,
			with $P$, $L$ and $V_{var}$ being the set of placeholders, literals and variables, respectively.,
		\item $E_{\hat{Q}} \subseteq V_{\hat{Q}} \times V_{\hat{Q}} $ is a set of edges,
		\item An edge in $E_{\hat{Q}}$ either has an edge label in $P$, i.e., property, or is a variable,
	\end{defitemize}
	where $P \subseteq \Sigma^*$.
\end{definition}

Reviewing the pipeline, a \gls{pgp} essentially serves as a precursor to a \gls{gp}:

\begin{enumerate}
	\item The domain expert generates a \gls{squall} expression containing placeholders,
		see \Cref{s:retrieval}--\Cref{s:generation}.
	\item Translation maps this expression to a \gls{pgp},
		i.e., a \gls{sparql} query where the placeholders have not yet been filled in with actual \glspl{uri},
		see \Cref{s:squallToSparql}.
	\item Linking maps the \gls{pgp} to a \gls{gp} in \gls{sparql},
		i.e., an executable query,
		see \Cref{s:linking}.
\end{enumerate}

See \Cref{fig:textToSparql} for the pipeline.\footnote{
	For clarity the schematics state \gls{sparql} instead of \gls{pgp}.
}

Finally, given a set of graphs $\mathscr{D}$,
the average number of vertices and edges is explicitly defined for clarity as follows:

\begin{align}
	\wideoverbar{V}_{\mathscr{D}} &= \frac{1}{|\mathscr{D}|}\sum_{G \in \mathscr{D}}|V_G| \label{eq:avgVertices} \\
	\wideoverbar{E}_{\mathscr{D}} &= \frac{1}{|\mathscr{D}|}\sum_{G \in \mathscr{D}}|E_G| \label{eq:avgEdges}.
\end{align}

\Cref{s:index} uses \Cref{eq:avgVertices} and \Cref{eq:avgEdges} to compare datasets containing graphs.

\subsubsection{Retrieval}
\label{s:retrieval}

Given a question targeted at a \gls{kg} = $(V, E)$,
retrieval consists of finding relevant information from its index $\mathscr{I}_{\text{\gls{kg}}}$,
see \Cref{fig:runningExampleRetrieval}.

\begin{figure}[!ht]
	\centering

	\mint{text}{Question:}
	\inputminted{text}{src/listings/running-example/question.txt}
	\mint{text}{}

	% Subfigure for the listing
	\begin{subfigure}[t]{0.26\textwidth}
		\vspace{0pt} % Ensures alignment at the top
		\inputminted{text}{src/listings/running-example/retrieval.txt}
		\caption{Retrieved vertices and edges.}
		\label{lst:runningExampleRetrievalLinearized}
	\end{subfigure}
	\hfill
	% Subfigure for the TikZ schematic
	\begin{subfigure}[t]{0.73\textwidth}
		\vspace{0pt} % Ensures alignment at the top
		\centering
		\begin{tikzpicture}[
	node/.style={draw, rounded corners, align=center, minimum width=2cm, minimum height=1cm},
	edge/.style={->, thick},
	scale=1, transform shape
]

% Left side nodes
\node[node] (modelA) at (-4.5, 0) {Model A};
\node[node] (model) at (-4.5, 2) {Model};
\node[node] (label) at (-4.5, -2) {Model A's Label};

% Right side nodes
\node[node] (benchmark) at (4.5, 0) {Benchmark};
\node[node] (dataset) at (4.5, 2) {Dataset};
\node[node] (code) at (4.5, -2) {Code};

% Edges
\draw[edge] (benchmark) -- node[midway, left] {has dataset} (dataset);
\draw[edge] (modelA) -- node[midway, right] {type} (model);
\draw[edge] (modelA) -- node[midway, right] {label} (label);

\end{tikzpicture}

		\caption{Graph representation.}
		\label{fig:runningExampleRetrievalGraph}
	\end{subfigure}

	\caption{Running example: retrieval.}
	\label{fig:runningExampleRetrieval}
\end{figure}

As stated, $\mathscr{I}_{\text{\gls{kg}}}$ is created by encoding all its vertices and edges using an \gls{lm}.
The subset of $k$ most similar vertices and edges for a given string $s$ is defined as:
\begin{align}
	&V_{s, k} = \arg \text{top}\,k _ { \, v \in V } \, \sigma(s, v), \label{eq:topkVertices} \\
	&E_{s, k} = \arg \text{top}\,k _ { \, e \in E } \, \sigma(s, e), \label{eq:topkEdges}
\end{align}
respectively, where $\arg \text{top}\,k _ { \, x \in X } \, f(x)$
essentially returns the $k$ elements from the set $ X $ that maximize the function $ f $ stated as its argument.
The similarity function $ \sigma $ was defined in \Cref{eq:similarityFunction}.
For an input strings $s \in \Sigma^*$,
the relevant information is defined as the subsets \( V_{s, k} \subseteq V \) and \( E_{s, k} \subseteq E \).

\paragraph{Nature of the Index}
\label{s:index}

The process of indexing
--- i.e., creating a set of graph embeddings $ \mathscr{I} $ (see \Cref{eq:index})
--- and retrieval differs significantly from the implementation in \cite{heGRetrieverRetrievalAugmentedGeneration2024}.
In their approach, a specific graph $H$ is associated with each question,
resulting in multiple graph indices $ \mathscr{I}_{H} $ rather than a unified index.
This design allows for retrieval within smaller, question-specific indices,
simplifying the search process compared to a comprehensive combined index.
However, the use of question-specific graphs assumes preprocessing steps if the starting point is a \gls{kg}.
%
In contrast, this thesis employs a single \gls{kg} as the foundation for all questions,
requiring retrieval from one unified, large-scale index $ \mathscr{I}_{\text{\gls{kg}}} $.
This approach leads to a significant disparity in graph sizes used as input for the \gls{pcst} algorithm.
Moreover, the \glspl{kg} used in \gls{mbse} are generally much larger than the graphs analyzed in
\cite{heGRetrieverRetrievalAugmentedGeneration2024} (see \Cref{s:size_kgb}).
Of the three datasets utilized in the original work, only WebQSP includes an underlying \gls{kg}.

The datasets used in \cite{heGRetrieverRetrievalAugmentedGeneration2024} are summarized as follows:

\begin{itemize}
	\item \textbf{ExplaGraphs} \cite{sahaExplaGraphsExplanationGraph2021}:
		Explanatory graphs, supporting and countering arguments are generated given a belief and an argument.
	\item \textbf{SceneGraphs} \cite{hudsonGQANewDataset2019}:
		Questions are generated from a scene graph structure.
	\item \textbf{WebQSP} \cite{yihValueSemanticParse2016}:
		Questions with subgraphs extracted from Freebase
		by extracting all triples that include entities related to the question within the maximum reasoning hops specified in WebQSP
		\cite{luoReasoningGraphsFaithful2024}.
\end{itemize}

For a detailed comparison of the datasets and their associated \glspl{kg},
refer to \Cref{table:comparisonDatasetsSamples} and \Cref{table:comparisonDatasetsKnowledgeGraphs}, respectively.

\begin{table*}[h]
	\centering

	\begin{tabular}{l|SSSS|r}

		\textbf{Dataset}
		& \textbf{\# Questions} & \textbf{\# Graphs} & \textbf{Average \# Vertices} & \textbf{Average \# Edges} & \textbf{Source} \\

		\hline
		
		\textbf{ExplaGraphs} & 2766  & 2766  & 5       & 4    & \cite{heGRetrieverRetrievalAugmentedGeneration2024} \\
		\textbf{SceneGraphs} & 10000 & 10000 & 19      & 68   & \cite{heGRetrieverRetrievalAugmentedGeneration2024} \\
		\textbf{WebQSP}      & 4737  & 4737  & 1371    & 4252 & \cite{heGRetrieverRetrievalAugmentedGeneration2024, luoReasoningGraphsFaithful2024} \\
		\textbf{SciQA}       & 2565  & 1     & 180184  & 6888 & \cite{auerSciQAScientificQuestion2023, auerSciQABenchmarkDataset2023} \\

	\end{tabular}

	\caption{
		Summary of datasets with key statistics including the number of questions, graphs, vertices, edges, and their sources.
	}

	\label{table:comparisonDatasetsSamples}
\end{table*}

\begin{table*}[h]
	\centering

	\begin{tabular}{l|rrrr|r}

		\textbf{Dataset}
		& \textbf{\glsentrylong{kg}} & \textbf{\# Triples} & \textbf{\# Vertices} & \textbf{\# Edges} & \textbf{Source} \\

		\hline
		
		\textbf{ExplaGraphs} & N/A         & N/A           & N/A           & N/A        & \cite{heGRetrieverRetrievalAugmentedGeneration2024} \\
		\textbf{SceneGraphs} & N/A         & N/A           & N/A           & N/A        & \cite{heGRetrieverRetrievalAugmentedGeneration2024} \\
		\textbf{WebQSP}      & Freebase    & 126M          & 88M           & 20K        & \cite{heGRetrieverRetrievalAugmentedGeneration2024, luoReasoningGraphsFaithful2024} \\
		\textbf{SciQA}       & \gls{orkg}  & \num{1133217} & \num{180184}  & \num{6888} & \cite{auerSciQAScientificQuestion2023, auerSciQABenchmarkDataset2023} \\

	\end{tabular}

	\caption{
		Overview of datasets and their associated (\glsentrylongpl{kg}),
		including triples, vertices, edges, and source references.
		\glsentryshort{orkg} version: February 14, 2023.
	}

	\label{table:comparisonDatasetsKnowledgeGraphs}
\end{table*}


\subsubsection{Post-Retrieval}
\label{s:postRetrieval}
%\TODO{Review ranking functions}

After retrieval, one set of vertices and one set of edges are obtained.
Together they do not necessarily form a connected subgraph.  
Crucial vertices and edges required to construct a semantically correct query,  
including triples that may not be easily inferable from the question, could also be absent.  
These challenges are addressed through post-retrieval processing,  
which produces a connected subgraph ideally encompassing all necessary triples.  
Refer to \Cref{fig:runningExamplePostRetrieval},  
which represents the post-processed version of \Cref{fig:runningExampleRetrieval}.  
It includes those additional vertices and edges that may not be as straightforward to infer as others, 
yet are essential for accurately parsing the question, e.g., \mintinline{text}{content}.

\begin{figure}[!ht]
	\centering

	\mint{text}{Question:}
	\inputminted{text}{src/listings/running-example/question.txt}
	\mint{text}{}

	% Subfigure for the listing
	\begin{subfigure}[t]{0.26\textwidth}
		\vspace{0pt} % Ensures alignment at the top
		\inputminted{text}{src/listings/running-example/post-retrieval.txt}
		\caption{Linearized connected graph.}
		\label{lst:runningExamplePostRetrievalLinearized}
	\end{subfigure}
	\hfill
	% Subfigure for the TikZ schematic
	\begin{subfigure}[t]{0.73\textwidth}
		\vspace{0pt} % Ensures alignment at the top
		\centering
		\input{src/schematics/running-example-post-retrieval-graph}
		\caption{Connected graph.}
		\label{fig:runningExamplePostRetrievalGraph}
	\end{subfigure}

	\caption{Running example: post-retrieval.}
	\label{fig:runningExamplePostRetrieval}
\end{figure}

Post-retrieval starts from the subsets of information
and constructs a graph $ H \in V_{\mathscr{C}(G)} $
that is relevant to the input string and constricted in size.
This is done by posing the task as a \glsentryfull{pcst} optimization problem
\cite{heGRetrieverRetrievalAugmentedGeneration2024}.
For a graph with vertices and edges assigned prizes and costs
--- essentially rewards and penalties
--- the \gls{pcst} algorithm aims to identify a connected subgraph that maximizes the total prize while minimizing the
total cost.

This approach aligns with the intuition that a question-relevant graph should include as many relevant vertices and
edges as possible (maximizing prizes),
but only up to the point where additional information becomes less relevant or detrimental
(minimizing cost by limiting subgraph size).
The algorithm thus seeks to strike a balance between poor-retrieval and over-retrieval,
optimizing the relevance of the subgraph.

The prizes are defined as follows:
\begin{align}
	&p_{s, k}(v) = \mathbb{1}_{v \in V_{s, k}}(k - r(v)), \quad &v \in V, \label{eq:vertexPrize} \\
	&p_{s, k}(e) = \mathbb{1}_{e \in E_{s, k}}(k - r(e)), \quad &e \in E, \label{eq:edgePrize}
\end{align}
where \( \mathbb{1}_{P(x)}(f(x)) \) is the indicator function for some arbitrary predicate $P$ and function $f$ 
\[
    \mathbb{1}_{P(x)}(f(x)) = 
    \begin{cases}
        f(x) & \quad P(x), \\
        0    & \quad \neg P(x).
    \end{cases}
\]
and
$r_{\sigma, s}(v)$ and $r_{\sigma, s}(e)$ are the rank of a vertex and edge, respectively,
according to the similarity function $\sigma$ compared to a string $s$.
The more similar a vertex or edge is to the string, the lower its rank and thus the higher its prize.
The cost $c$ of a graph $H = (V_H, E_H)$ is defined as follows
\begin{align}
	c_C(H) = |V_H| \cdot C
	\label{eq:graphCost}
\end{align}
where $C$ indicates the cost per edge, meant to give control over the subgraph size.
The function of $c_C$ is to restrict the size of the subgraph such that the algorithm scales better
w.r.t. the size of the graph in question.
Finally, the objective is to find $ H = (V_H, E_H) \in V_{\mathscr{C}(G)} $,
that maximizes:
\begin{align}
	\sum _ { v \in V_H } p_{s, k}(v) + \sum _ { e \in E_H } p_{s, k}(e) - c_C(H)
\end{align}
The practical computation of this objective is detailed in previous literature
\cite{heGRetrieverRetrievalAugmentedGeneration2024}.
Note that neither retrieval nor post-retrieval processing require any training on labeled \gls{kg}-specific data.

For an example of (the linearized version of) a constructed subgraph, see \Cref{lst:runningExampleRetrievalLinearized}.
The \gls{pcst} algorithm has three parameters,
\Cref{s:experiments} goes into depth on initializing the retrieval component along with the performance impact.

\subsubsection{Augmentation}
\label{s:augmentation}

Augmentation, initially discussed in \Cref{s:rag}, aims to optimize generation by enhancing retrieval.
While not always explicitly considered a distinct component,
the augmentation process outlined here is notably more intricate.
This contrasts sharply with the Naive \gls{rag} paradigm,
which gained prominence with the widespread adoption of ChatGPT \cite{gaoRetrievalAugmentedGenerationLarge2024}.
In that paradigm, the primary step involves synthesizing the retrieved context with the posed query into a single prompt,
which is then provided as input to the \gls{llm} \cite{gaoRetrievalAugmentedGenerationLarge2024}.
The complexity of the current approach, however, necessitates a more thorough explanation.

First, the augmentation process is achieved through soft graph prompting \cite{liuCanWeSoft2024},
which adapts previous work on soft prompt to graph learning tasks. 
Specifically, it involves integrating the augmentation process at an intermediate layer of a \gls{rag}'s generator.
This is essential, as otherwise the graph would need to be linearized,
resulting in the loss of crucial structural information \cite{heGRetrieverRetrievalAugmentedGeneration2024},
as previously discussed (see \Cref{s:squallExpert}).

The subgraph obtained of post-retrieval processing is embedded using a graph embedder
which in turn is projected to the latent space of the generator's text embedder using an \gls{mlp},
resulting in an ``aligned'' graph token.
Afterwards, concatenated graph token and text tokens are inputted to the generator's self-attention layers
and pass through the remainder of the \gls{llm}, i.e., \gls{squall} expert as normal.
Both the graph embedder and projector are jointly trained during the second phase of sequential training
in order to be effective at representing and aligning the graph, respectively.
%\TODO{Extra ref. for the above? Maybe \cite{GuidingFrozenLanguage}}
The construction of the dataset and training method are detailed in \Cref{s:dataset} and \Cref{s:training_finetuning},
respectively.

\paragraph{Graph Embedding}

Graph embedding maps a graph $H$ to embeddings using a model $\gamma$,  such as a \gls{gnn} or a \gls{gat},
followed by a mean pooling operation $\rho$:
\[
	h_{\rho \circ \gamma} = \rho(\gamma(H)) \in \mathbb{R}^{d_g}, \quad H \in V_{\mathscr{C}(G)},
\]
where $\mathbb{R}^{d_g}$ is the hidden dimension of the graph embedder.

Essentially, \( \gamma \) generates vector representations,
which are then aggregated by \( \rho \) into a single embedding of size \( \mathbb{R}^{d_g} \),
effectively summarizing the graph's structural and feature information.

\paragraph{Projection}

Due to the fact that the text and graph embedding spaces are different,
the introduction of another \gls{ann} that functions as a learned mapping from graph to text embedding space
improves performance, as empirically shown by \cite{heGRetrieverRetrievalAugmentedGeneration2024}.
The \gls{ann} responsible for this function is conceptualized as a projector $\pi$
mapping input graph embeddings to the text embedding space,
or in other words it aligns graph tokens with the \gls{squall} expert:
\[
	h_{\pi} = \pi (h) \in \mathbb{R}^{d_t},
	\quad h \in \mathbb{R}^{d_g},
\]
where $\mathbb{R}^{d_t}$ is the hidden dimension of the text embedder.

\subsubsection{Generation}
\label{s:generation}

Given a question and its associated soft graph prompt, a \gls{squall} expression is generated,
see \Cref{fig:runningExampleGeneration}.

\begin{figure}[!ht]
	\centering

	\mint{text}{Question:}
	\inputminted{text}{src/listings/running-example/question.txt}
	\mint{text}{}

	\begin{subfigure}[t]{\textwidth}
		\vspace{0pt} % Ensures alignment at the top
		\centering
		\input{src/schematics/running-example-post-retrieval-graph}
		\caption{Connected graph from post-retrieval used to create the soft graph prompt.}
	\end{subfigure}
	\mint{text}{}

	\begin{subfigure}[t]{\textwidth}
		\vspace{0pt} % Ensures alignment at the top
		\mint{text}{SQUALL:}
		\inputminted{text}{src/listings/running-example/squall.txt}
		\caption{Generated SQUALL expression.}
		%\label{lst:runningExamplePostRetrievalLinearized}
	\end{subfigure}
	\mint{text}{}

	\caption{Running example: generation.}
	\label{fig:runningExampleGeneration}
\end{figure}

The text embedder $\lambda$ of the \gls{squall} expert is applied to an input string:
\[
	h_{\lambda} = \lambda(s) \in \mathbb{R}^{N \times d_t}, \quad s \in \Sigma^*,
\]
where $N$ is the amount of tokens.
Afterwards,
the concatenation of the string and graph embedding are inputted to the self-attention layers of the \gls{squall}
expert and pass through the model as normal:
\[
	\hat{s} = \delta(h_{\pi}h_{\lambda}) \in \Sigma^*,
\]
where $h_{\pi}$ is the soft graph prompt.

\subsubsection{SQUALL-to-SPARQL}
\label{s:squallToSparql}

The \gls{squall} language \cite{ferreSQUALLExpressivenessSPARQL2014} has an accompanying tool,
called the squall2sparql translator \cite{ferreSquall2sparqlTranslatorControlled2013},
which can map any syntactically correct \gls{squall} to \gls{sparql} (\glsentryshort{s2s}),
and in a more limited capacity, \gls{sparql} to \gls{squall}.
Thus, assuming proper generation, a string $S$ in \gls{squall} can be extracted from the model's output $\hat{s}$
and mapped to \gls{sparql}:
\[
	f_{\glsentryshort{s2s}}: \quad S \mapsto \hat{Q},
\]
Note that $S$ and $\hat{Q}$ contain placeholders and not \glspl{uri},
the latter is thus a \gls{pgp}, see \Cref{lst:runningExampleSquallToSparql} for the example.

\begin{figure}[!ht]
	\centering

	\mint{text}{Question:}
	\inputminted{text}{src/listings/running-example/question.txt}
	\mint{text}{}

	\begin{subfigure}[t]{\textwidth}
		\vspace{0pt} % Ensures alignment at the top
		\centering
		\input{src/schematics/running-example-post-retrieval-graph}
		\caption{Connected graph from post-retrieval used to create the soft graph prompt.}
	\end{subfigure}
	\mint{text}{}

	\begin{subfigure}[t]{\textwidth}
		\vspace{0pt} % Ensures alignment at the top
		\mint{text}{SQUALL:}
		\inputminted{text}{src/listings/running-example/squall.txt}
		\caption{Generated SQUALL expression.}
	\end{subfigure}
	\mint{text}{}

	\begin{subfigure}[t]{\textwidth}
		\vspace{0pt} % Ensures alignment at the top
		\centering
		\mint{text}{PGP:}
		\inputminted{sparql}{src/listings/running-example/pgp.sparql}
		\caption{\gls{pgp}.}
	\end{subfigure}

	\caption{Running example: mapping a \glsentryshort{squall} expression to a \glsentrylong{pgp}.}
	\label{lst:runningExampleSquallToSparql}
\end{figure}

\subsubsection{Linking}
\label{s:linking}

%\TODO{diestelGraphTheory2017: \gls{pgp} to \gls{gp} is a homomorphism, number of vertices is order, number of edges is size.}
%\TODO{
%	Semantic graph matching (subgraph isomorphism) of the pgp/query graph with the constructed minimal subgraph?
%	\cite{maOntologybasedEntityMatching2019, zhengSemanticSPARQLSimilarity2016}.
%}

% General
Given an \gls{rdf} graph $G = (V, E)$, here, linking is conceptualized as the task of mapping a \gls{pgp} to a \gls{gp}
(in \gls{sparql}), i.e., associating vertices and edges to placeholders, see \Cref{lst:runningExampleLinking}.

\begin{listing}[!ht]
	
	\mint{text}{Question:}
	\inputminted{text}{src/listings/running-example/question.txt}
	\mint{text}{}

	\mint{text}{PGP:}
	\inputminted{sparql}{src/listings/running-example/pgp.sparql}
	\mint{text}{}

	\mint{text}{Placeholders}
	\mint{text}{Model, rdfs:label, has dataset, has benchmark, has model, has source code}
	\mint{text}{}

	\mint{text}{Potential URIs}
	\mint{text}{orkgc:Model, rdfs:label, orkgp:HAS_DATASET, orkgp:HAS_BENCHMARK, orkgp:HAS_MODEL, orkgp:HAS_SOURCE_CODE}
	\mint{text}{}

	\mint{text}{GP:}
	\inputminted{sparql}{src/listings/running-example/gp.sparql}
	\mint{text}{}

	\caption{Running example: linking a \glsentrylong{pgp} to a \glsentrylong{gp}.}
	\label{lst:runningExampleLinking}
\end{listing}

Many possible approaches have been proposed that focus specifically on the linking task.
The focus of this work being on different parts of the text-to-\gls{sparql} puzzle,
a linker from the literature was adopted that agrees with the data scarcity initially supposed.
It is a dynamic approach part of a proposed platform \cite{omarUniversalQuestionAnsweringPlatform2023},
that requires no training nor data, instead relying on \gls{jit}-linking.
Although significant changes were made to the framework itself, it proved to be a valuable starting point.
Parts of the codebase dealing with server and \gls{kg} interactions,
including requests and querying, have for example been retained.
Hereafter, a summary of the original \gls{jit}-linking procedure is given,
while the remainder of this section deals with the formalization of the linking approach
as it was ultimately implemented.

Assuming a list of triples, where each triple consists of the placeholders $(s, p, o)$,
linking \cite{omarUniversalQuestionAnsweringPlatform2023} involves the following steps.
First, exploratory queries
are utilized to retrieve potential vertices and edges for each placeholder
--- hereafter, they are referred to as probing queries.
They query for vertices or edges whose \mintinline{sparql}{rdfs:label}
contains the individual words of the generated placeholder in question.
Then, the results are scored and ranked using a function that evaluates the semantic similarity
between the \mintinline{sparql}{rdfs:label}s of the retrieved vertices/edges and their corresponding placeholders.

Owing to how previous components were implemented, here, the linker is the final step before result handling.
In contrast, the original approach \cite{omarUniversalQuestionAnsweringPlatform2023} requires some further 
steps.
Adaptations to their linking approach are explained and justified in \Cref{s:adaptation_of_the_linker}.

% Formalisation
The implemented linking procedure can be formally stated as follows.
Given a \gls{pgp} ($\hat{Q}$), the goal is to map it to a set of possible \glspl{gp},
achieved by interacting with the \gls{rdf} graph $G = (V, E)$ through probing queries
\[
	\nu: \Sigma^* \rightarrow (V \cup E) \times \Sigma^*.
\]
For any arbitrary placeholder, a probing query retrieves all potential vertices/edges
with their respective \mintinline{sparql}{rdfs:label}:
\[
	I_p = \nu(p), \quad p \in P
\]
The correspondence of \gls{uri} and vertex or edge is implicitly assumed here.
Doing this for all placeholders and scoring them:
\[
	\begin{aligned}
		X_p 			&= \Set{ (x, y)\ | \begin{array}{l}
			(x, s) \in I_p \\
			\land \, y = \sigma(p, s),
	\end{array}} \quad p \in P \\
		\mathscr{X} &= \Set{ X_p } _ {p \in P}
	\end{aligned}
\]
where $\sigma$ is a similarity function whose implementation is detailed in \Cref{s:settings_linking}.

Resulting in a set of scored vertex/edge-\mintinline{sparql}{rdfs:label} pairs for each placeholder.\footnote{
	The sets of tuples are preordered: $(x_1, y_1) \lesssim (x_2, y_2) \iff y_1 \geq y_2$.
}
Through $\mathscr{X}$ a set of mappings $\mathscr{F}$ is defined:\footnote{
	The code implementation of $\mathscr{F}$ is a generator over the Cartesian product
	$\bigtimes _ {X \in \mathscr{X}} X$.
}
\[
	\mathscr{F} = \Set{ f\ | \begin{array}{l}
			f: P \rightarrow (V \cup E) \times [0, 1] \\
			\land \, f(p) = X_p^{(k_p)} \\
			\land \, k_p \in \{1, \ldots, |X_p|\} \\
			\land \, p \in P
	\end{array}}
\]
The mappings are ranked using a scoring functional $\phi$ defined as follows:\footnote{
	A preorder $ \lesssim $ can be defined on $ \mathscr{F} $ using $\phi$,
	leading to the preordered set $ (\mathscr{F}, \lesssim) $.
	Specifically, the preorder $ \lesssim $ is defined by:
	$ (f_1, f_2) \in \; \lesssim \iff \phi(f_1) \geq \phi(f_2) $
	where $ \lesssim \; \subseteq (\mathscr{F} \times \mathscr{F}) $.
}
\[
	\phi: \quad \mathscr{F} \rightarrow \mathbb{R}^+, \quad f \mapsto \sum_{p \in P} f(p)_2,
\]

%\TODO{Find cleaner notation.}
Finally, the set of possible \glspl{gp} is built:
\[
	\mathscr{Q} = \Set{ Q\ | \begin{array}{l}
			f \in \mathscr{F} \land Q = (V_{Q, f}, E_{Q, f})
	\end{array}}
\]
where
\[
	\begin{aligned}
		V_{Q, f} = &\Set{ v\ | p \in P \land v = f(p)_1 \land v \in V}
		      \cup \Set{ v\ | v \in V_{\hat{Q}} \land v \notin P } \\
		E_{Q, f} = &\Set{ e\ | p \in P \land e = f(p)_1 \land e \in E}
		      \cup \Set{ e\ | e \in E_{\hat{Q}} \land e \notin P }
	\end{aligned}
\]
%\TODO{Compare \cite{maOntologybasedEntityMatching2019}:
%	A match of P in G induced by f, denoted as P (G, f ),
%	is the induced subgraph of G with nodes and edges from matching f.
%}

The implementation of $\nu$ is slightly different depending on the \gls{rdf} engine used in practice,
and whether the input is a vertex or edge.
Furthermore, placeholder strings are split up into individual words: $s = \Set{ w_1w_2\ldots }$.
An example of a vertex and edge probing query is illustrated in \Cref{lst:probingQuery}.
For practical reasons, the amount of results is capped.

\begin{listing}[!ht]

	\mint{text}{QUERY: potential_vertices(n_max_vertices, w_1, w_2, ...)}
	\inputminted{sparql}{src/listings/probing-vertex.sparql}
	\mint{text}{}

	\mint{text}{QUERY: potential_edges(n_max_edges, w_1, w_2, ...)}
	\inputminted{sparql}{src/listings/probing-edge.sparql}
	\mint{text}{}

	\caption{Vertex (above) and edge (below) probing query.}
	\label{lst:probingQuery}
\end{listing}

\subsubsection{Result Handling}

Three straightforward post-processing steps are implemented for this proof-of-concept:
natural language explanation, tabulation and visualization.

\paragraph{Natural Language Explanation}

A natural language explanation is straightforwardly achieved through prompting a \gls{pllm} with the original stakeholder 
question and the results of the executed query along with the task to answer and explain if possible.

\paragraph{Tabulation}

Tabulation consists of a simple formatting function that takes as input the resulting data from an executed query
and returns a table to be displayed in the notebook.

\paragraph{Visualization}  

Due to scope limitations, an exemplary implementation from the literature \cite{raissyaVizKGFrameworkVisualizing2021}  
was integrated into the approach.  
The VizKG framework provides multiple visualization options for query results over \glspl{kg},  
along with a visualization recommendation system.  
This system was used to automatically generate appropriate visualizations without requiring user input.  

For an example see \Cref{lst:runningExampleResultHandling}.

\begin{listing}[!ht]
    \centering

    \mint{text}{Question:}
    \inputminted{text}{src/listings/running-example/question.txt}
    \mint{text}{}

    \begin{minted}{markdown}
Natural Language Explanation:
The papers that utilized the Depth DDPPO model and include links to their code are as follows:  
1. [Efficient Exploration with Depth DDPPO](https://github.com/repo1)  
2. [Optimizing Depth Models](https://github.com/repo2)
    \end{minted}

    % Caption for the figure
    \caption{Running example: result handling.}
    \label{lst:runningExampleResultHandling}
\end{listing}

\subsection{Architectural Choices}

Having provided a detailed explanation of the architecture, this section focuses on the key design decisions made.
It elaborates on the rationale behind each choice, highlighting how these decisions contribute to the effectiveness
of the architecture in enabling model reporting for \gls{mbse}.

\subsubsection{Fine-Tuning over Prompting}

Two primary categories of models emerge when considering a neural semantic parser:
fine-tuned \glspl{llm} and \glspl{pllm} with a suitable prompting strategy.
The choice between the two is mainly informed by syntactic and semantic correctness.

Syntactic correctness pertains to the formal structure of the output logical forms,
while semantic correctness is concerned with the domain-specific understanding for which the parser is trained.
Intuitively, a \gls{pllm} like ChatGPT might excel in syntactic correctness,
assuming it has encountered the logical forms during its extensive pre-training.
On the other hand, a smaller, fine-tuned \gls{lm} could perform better in terms of semantic correctness,
given its tailored adaptation to the domain.

\paragraph{Syntactic Correctness}

The original paper \cite{lehmannLanguageModelsControlled2023},
which proposed using \glspl{lm} as \glsentryfull{cnl} semantic parsers, used fine-tuning.
As previously discussed, \gls{cnl} data is typically rare in the pre-training datasets of \glspl{lm},
suggesting that fine-tuning was likely necessary instead of relying solely on prompting a \gls{pllm}.
Although this rationale is not explicitly stated, it is considered reasonable and is thus adopted.

\paragraph{Semantic Correctness}

Furthermore, more recent work \cite{lehmannLargeLanguageModels2024},
which established baselines for the benchmark used to evaluate this approach (see \Cref{s:benchmark}),
demonstrated superior performance a fine-tuned \gls{llm} compared to prompted \glspl{pllm}.
While the focus was on generating \gls{sparql} rather than \gls{squall}
--- two formal languages with distinctly different syntaxes
--- the choice between fine-tuning and prompting remains relevant to semantic correctness.
These baselines indicate that fine-tuning likely enhances semantic correctness compared to prompting.

\paragraph{Feasibility of RAG}

Additionally, the \gls{rag} paradigm outlined in \Cref{s:sysArch} is generally infeasible with \glspl{pllm}
(e.g., ChatGPT) because such models do not provide access to their intermediate layers,
a requirement for implementing soft graph prompting.
Consequently, fine-tuning serves a dual purpose:
it teaches the \gls{llm} to generate \gls{squall} expressions effectively
and incorporates non-parametric memory through graph-based representations.

\subsubsection{Placeholders}
\label{s:why_placeholders}

The proposal to employ placeholders instead of identifiers in the generator output is based on three key thoughts:
identifier hallucination, placeholder hallucination and domain adaptation.
Each is explained followed by a demonstrative example.

\paragraph{Identifier Hallucinations}

By relying on placeholders, identifier hallucination can be avoided.
%
If the \gls{squall} expert were fine-tuned to generate identifiers,
it is reasonable to assume that its integration into the domain expert would often result in erroneous identifiers.
The generation of faulty identifiers would likely be influenced by those encountered during the fine-tuning of the
\gls{squall} expert.
Clearly, this outcome is highly undesirable.
%
This intuition is corroborated by recent results pertaining to a similar situation.
A pre-trained \gls{llm}, that was fine-tuned to generate queries for a specialized domain, frequently hallucinated
predicates akin to those found in popular \glspl{kg} such as Wikidata,
likely due to the presence of those domains in its pre-training data \cite{lehmannLargeLanguageModels2024}.
An illustrative example is given by \Cref{lst:identifierHallucination},
it shows a predicate that is nonexistent in the fine-tuning data
(SciQA dataset for \gls{orkg} domain, see \Cref{s:knowledge_graph_and_benchmark}),
but does exist in the pre-training data.
%
\begin{listing}[!ht]
	\inputminted[escapeinside=||]{sparql}{src/listings/identifier-hallucination.sparql}
	\caption{
		Representative example of identifier hallucination by GPT-3.5 \cite{lehmannLargeLanguageModels2024}.
		P2067 is an existing Wikidata property.
	}
	\label{lst:identifierHallucination}
\end{listing}
%
Although unfreezing the generator component during the second round of training likely reduce this problem,
it would most definitely increase the data requirements, due to the higher number of parameters,
which is challenging given the practical data limitations in \gls{mbse}.
While other studies address this issue by augmenting datasets to achieve better \gls{kb} coverage
\cite{rangelSPARQLGenerationAnalysis2024}, such extensive data engineering was deemed out of scope for this work.
Finally, reasoning from the similarity of the sequential training strategy with continual learning,
the fear of catastrophic forgetting is introduced \cite{liSemanticParsingLimited2023}:
the general ``knowledge`` the \gls{squall} expert learned during the initial training phase would deteriorate.
%
Assuming the generator component is frozen, harmonizing the \gls{squall} expert with the new domain is likely
to remain problematic in terms of identifier hallucination, as a substantial portion of identifiers will continue
to be \glsentryfull{oov}, i.e., unseen during the second round of training.
This is the case even in high-quality datasets, for example, LC-QuAD 2.0 \cite{dialloComprehensiveEvaluationNeural2024}.
%
The incorporation of non-parametric memory via \gls{rag} is unlikely to fully resolve this issue.
Two key challenges have been identified that diminish the quality of retrieval,
thereby increasing the risk of hallucination \cite{zhaoRetrievalAugmentedGenerationAIGenerated2024}.
The subgraph produced by the retrieval (\Cref{s:retrieval}) and post-retrieval (\Cref{s:postRetrieval}) components may:
%
\begin{itemize}
	\item Fail to include all necessary vertices and edges.
	\item Contain multiple distinct vertices or edges with identical \mintinline{sparql}{rdfs:label},
		leading to the potential generation of incorrect \glspl{uri} for vertices or edges.
\end{itemize}
%
These challenges are common in \gls{ir} and are often attributable to noise in retrieval results.
The former represents an issue of poor retrieval (low recall), while the latter, referred to as over-retrieval,
can confuse the model \cite{zhaoRetrievalAugmentedGenerationAIGenerated2024}.
Both issues are inherent limitations of the \gls{rag} paradigm.
%
Moreover, it is unclear how these methods would perform with a heterogeneous dataset,
composed of multiple question-answering datasets targeting different \glspl{kg}.
At face value, using a heterogeneous dataset to train the proposed \gls{squall} expert does not pose issues,
given the use of placeholders.

\paragraph{Placeholder Hallucination}

Previous work shows how \gls{llm} hallucination of plausible relation placeholders can be leveraged if 
appropriate post-hoc adjustment is used, in the context of \gls{squall} generation
\cite{lehmannLanguageModelsControlled2023}.
In this work, the placeholder hallucination is extended to entities as well,
and post-processing is performed by the linker. 
It is assumed that more plausible hallucinations
--- i.e., placeholders that are more similar to the expected labels
--- would improve linking too, since placeholders are more easily associated.

\paragraph{Domain Adaptation}

A key novel insight of this work is that the use of placeholders can effectively facilitate domain adaptation,
further elaborated upon in \Cref{s:contributions}.
The \gls{squall} expert's task is to generate queries with likely placeholders,
a role that remains consistent across different domains, with only the domain itself changing when used as generator
in the domain expert.
Through retrieval augmentation, these placeholders are made more plausible within the context of the target domain.
By ensuring that the \gls{squall} expert has extensive knowledge of \gls{squall},
it can be reused in another domain, transferring its capabilities effectively.
This approach allows the \gls{squall} expert to be applied to any target domain,
ensuring that the significant efforts invested in data collection and training
to enhance its \gls{squall} and \gls{sparql} coverage are not wasted.

\paragraph{Example}

The previous points are illustrated by revisiting an example.
The \gls{squall} expert trained on Wikidata learns that for a question relating to the \gls{nlp} domain,
the \mintinline{text}{natural language processing} placeholder is expected, which corresponds to the \mintinline{sparql}{wd:Q30642} \gls{uri}
in the Wikidata \gls{kg}.
Now the jump must be made to another domain, for example \gls{orkg}, a \gls{kg} including scientific papers.
A stakeholder might be interested in finding papers with the \mintinline{text}{natural language processing} keyword,
this corresponds to the \mintinline{sparql}{orkgr:R51020} \gls{uri}.
Thankfully, the \gls{squall} expert has no knowledge of identifiers, instead it generates a likely placeholder
--- ideally \mintinline{text}{natural language processing}
--- where its generation is augmented by extra context from the retriever.
To illustrate the transfer learning capabilities of the proposed approach,
consider the queries presented in \Cref{lst:domainAdaptation}.

\begin{listing}[!ht]

	\inputminted{sparql}{src/listings/domain-adaptation/source.sparql}

	\mint{text}{---}

	\inputminted{sparql}{src/listings/domain-adaptation/target.sparql}

	\caption{
		Example query from the source domain (above) and the target domain (below).
	}
	\label{lst:domainAdaptation}
\end{listing}

\subsubsection{SQUALL over SPARQL}  
\label{s:SQUALLOverSPARQL}  

The decision to use \gls{squall} rather than \gls{sparql} for query generation in this work is
rooted in one critical advantage: \gls{squall} is a \gls{cnl}.
Its similarity to natural language reduces the complexity of generating logical forms
lowering training data requirements \cite{lehmannLanguageModelsControlled2023}, as previously mentioned (\Cref{s:cnl}).
Moreover, \gls{squall} supports all \gls{sparql} constructs, including many from \gls{sparql} 1.1,
and allows for unambiguous mapping to \gls{sparql}.
These properties make \gls{squall} a practical choice for tasks such as query generation and \gls{kgqa},
particularly in the context of limited data availability.

\subsubsection{Order Translator and Linker}
\label{s:order_translator_and_linker}

Although the order of translator and linker could be switched,
it is more practical to convert to \gls{sparql} first, 
because existing parsing libraries for the language prevent further unnecessarily complicating linking.
For example, \gls{sparql} parsing allows the disambiguation of subjects and objects on the one hand from predicates on
the other.
No such parsing libraries for \gls{squall} were found to exist at the time.

\subsubsection{Adaptation of the Linker}
\label{s:adaptation_of_the_linker}

The linker as originally implemented \cite{omarUniversalQuestionAnsweringPlatform2023} requires some additional 
steps before \gls{sparql} queries are obtained.
Their generative model has a somewhat different target than the presented domain expert,
although placeholders are also used,
a list of triples instead of \gls{squall} is generated.
No other additional information is generated, more specifically, the structure of the query is not part of generation.
They address this through heuristics, for example, an \mintinline{sparql}{ASK} query is assumed if a question starts with ``is'',
which is then built after linking the generated triples.

The linker presented here is more streamlined since the need for query construction is entirely eliminated,
along with its resulting limitations.
Instead, the query structure emerges directly from the generation process, removing the necessity for heuristics.
Consequently, the coverage of the set of all possible \gls{sparql} queries depends on the models
(i.e., the \gls{squall} and domain expert)
and their training data ($\mathscr{D}_S$ and $\mathscr{D}_T$).
It targets \gls{squall}, which encompasses the full scope of SPARQL 1.1 \cite{ferreSQUALLExpressivenessSPARQL2014},
as opposed to the more restricted subset achievable through manual query construction.
Therefore, practical coverage is constrained by the used datasets (see \Cref{s:dataset}).
By leveraging an \gls{llm} for generating query structures,
this approach has the potential to achieve broader coverage, limited by data collection and engineering
rather than the a priori limits of heuristic methods.

\subsubsection{Differing Retrieval and Linking Strategies}

While retrieval and linking share similarities
--- both aim to relate text to a domain
--- their differences remain significant.  
Retrieval, as an \gls{ir} task, focuses on finding relevant information from the domain given an input string,
whereas linking maps an input string to a specific instance within the domain.

For instance, the outcome of retrieval (or post-retrieval processing) provides valuable information,
i.e., vertices and edges of the \gls{kg},
which augment the generation process leading to the \gls{pgp} with its placeholders.  
A natural question arises: why is this retrieved information not reused during linking?  
The challenge lies in disambiguation
--- selecting the correct vertex from the retrieved data is far from trivial.

% Recap
As outlined in the following section (\Cref{s:contributions}),
the primary technical focus of this work is to address the data scarcity issue.  
Moreover, both retrieval and linking have been extensively explored in the literature.  
For example, ReFinED \cite{ayoolaReFinEDEfficientZeroshotcapable2022} is a specialized entity linker.  
Consequently, further investigation into harmonizing the retrieval and linking components was deemed out-of-scope for
this work.

\subsection{Contributions}
\label{s:contributions}

%\TODO{
%	How to describe transfer learning approach?
%	Maybe feature-representation-transfer \cite{panSurveyTransferLearning2010}?
%	Other work that uses \gls{rag} to achieve TL	\cite{siriwardhanaImprovingDomainAdaptation2023} might already 
%	have some terminology.
%}

% Intro
Now that the architecture of the implementation has been elaborated,
the technical contributions of this work can be summarized appropriately.
The non-technical contributions outlined in the introduction (\Cref{s:introduction})
include the model reporting paradigm and Mo-Lab, a proof-of-concept implementation of this paradigm,
which leverages the reusable domain expert model.
Those related to the \gls{kgqa} pipeline and text-to-\gls{sparql} task now follow.

% Contributions
Domain adaptation is central to this work and its contributions.
Here, the adaptation is from the source domain $\mathscr{D}_S$ of the \gls{squall} expert
(see \Cref{s:source_domain_data})
to the target domain $\mathscr{D}_T$ of the domain expert
(see \Cref{s:gt_retriever_data} and \Cref{s:syn_retriever_data} for the ground truth and synthetic case, respectively).
The learning task is to generate question-equivalent \gls{squall}.
Conditions for the target domain are further relaxed \cite{panSurveyTransferLearning2010},
currently $\mathscr{D}_T$ consists of a sequence of questions $\mathcal{S}$
and a limited labeled support set $\mathscr{D}_{\text{sup}}$ for the target domain.

% Novelty
While domain adaptation through \gls{rag} is not a novel concept \cite{siriwardhanaImprovingDomainAdaptation2023},
no prior work has been identified in the literature that applies this approach to the query generation task.
In this regard, the combination of components introduced in the proposed architecture is,
to the best of the author's knowledge, unique.
Although some reliance on labeled data remains, this is deemed acceptable given the minimal amount required and the
relative ease of acquisition compared to alternatives.
For instance, engineering a complete question-query dataset for a specialized domain from scratch
\cite{auerSciQAScientificQuestion2023}, demands significantly more effort.

% Summary
The key technical contribution of this work is identified as domain adaptation,
and represents a significant step forward in integrating contemporary \gls{nlp} techniques into \gls{mbse}.
It is leveraged to realize a \gls{kgqa} pipeline in limited data scenarios.
Essential insights were the usage of placeholders (\Cref{s:why_placeholders})
and the sequential training approach (\Cref{s:training_finetuning}).

\section{Implementation Details}

This section explains
how the datasets used for the approach were created,
how the retriever component (\Cref{s:retrieval}) was initialized and evaluated,
and finally how fine-tuning and training was done.

\subsection{Datasets}
\label{s:dataset}

%An \gls{llm} is used for the generation, which is initially fine-tuned using a dataset consisting of \gls{nlq} and
%\gls{kg}-agnostic \gls{squall} pairs, i.e., there are no \glspl{uri} specific to any \gls{kg}, instead textual placeholders are used.
%The construction of the dataset and fine-tuning are detailed in \Cref{s:dataset} and \Cref{s:training_finetuning}, respectively.
%The base model is later tailored through a specific \gls{kg} through training.
%Here this is both done using a ground truth labeled dataset, for evaluation, and a synthetic one, for the data-scarce scenario.
%It was considered out of scope to pursue more complex generation methods that generate a more varied synthetic dataset.

This section describes the creation process for the datasets that were employed.
The first is used to fine-tune the \gls{llm} of the generator on the text-to-\gls{squall} task,
the second for the training of the retrieval,
and the final is a replication of the latter through synthetic data generation.
Common steps include replacing \glspl{uri} with placeholders,
which results in a \gls{pgp} given a \gls{gp},
and \gls{sparql}-to-\gls{squall} mapping.
A dataset $\mathscr{D}$ is formalized as a set of samples:
$ \mathscr{D} = \Set{ d_i | i \in \mathcal{I} } $, over some index set $\mathcal{I}$,
where each sample is a sequence:
$ d = (d_n)_{n \in \mathbb{N}} $.

\subsubsection{Source Domain Data}
\label{s:source_domain_data}

% LC-QuAD Intro
The source domain dataset $\mathscr{D}_S$ used for fine-tuning the \gls{squall} expert
(see \Cref{s:fineTuningSquallExpert}) includes natural language questions and \gls{squall} expressions.

It was constructed, starting from LC-QuAD 2.0 \cite{dubeyLCQuAD20Large2019},
which contains \num{30000} natural language questions for both the Wikidata and DBpedia 2018,
along with the corresponding queries.
The dataset contains 27 question and 33 query templates \cite{dialloComprehensiveEvaluationNeural2024}.
LC-QuAD 2.0 itself was created by starting from these templates, 
selecting vertices and appropriate edges, and generating \gls{sparql}.
Afterwards, these queries are transformed to \glsentryfullpl{nnqt},
situated between the original formal language and target natural language,
and containing textual placeholders in place of the original vertices and edges surrounded by brackets.
Through human verbalization they are turned into true natural language questions.
To increase variety, a final human paraphrasing step is performed on the verbalized questions,
with the explicit intention of combating over-fitting of downstream \gls{kgqa} models
on a particular question syntax \cite{dubeyLCQuAD20Large2019}.

% Process
The following steps are required to construct $\mathscr{D}_S$ from LC-QuAD 2.0.
Consider a sample with this question:
\begin{minted}{text}
What periodical literature does Delta Air Lines use as a mouthpiece?
\end{minted}

\begin{enumerate}[leftmargin=*, labelindent=\parindent]

	\item From the sample's \glsentryfull{nnqt} the placeholders between curly brackets are extracted,
		see \Cref{lst:nnqtToPlaceholders}.
		
	\begin{listing}[H]
		\inputminted{text}{src/listings/dataset-transformations/source-domain/extracting-placeholders.txt}
		\caption{Source domain: extracting placeholders from a \glsentrylong{nnqt}.}
		\label{lst:nnqtToPlaceholders}
	\end{listing}

	\item Then, from the sample's \gls{sparql} query the \glspl{uri} are extracted,
	see \Cref{lst:sourceDomainSparqlToUri}.

	\begin{listing}[H]
		\mint{text}{SPARQL:}
		\inputminted{sparql}{src/listings/dataset-transformations/source-domain/query.sparql}
		\mint{text}{}	
		\inputminted{text}{src/listings/dataset-transformations/source-domain/uris.txt}
		\caption{Source domain: extracting \glsentryshortpl{uri} from a \glsentryshort{sparql} query.}
		\label{lst:sourceDomainSparqlToUri}
	\end{listing}

	\item Using the sample's query-template, i.e., \mintinline{text}{<S P ?O ; ?O instanceOf Type>},
		\glspl{uri} can be associated with placeholders.
		E.g., \mintinline{text}{Type} is the \emph{object} of the second triple (\mintinline{text}{?O instanceOf Type})
		corresponding to the \mintinline{text}{wd:Q1002697} identifier of the query.
		For every sample from LC-QuAD that follows the same template,
		this \mintinline{text}{Type} will correspond to the first placeholder of the \gls{nnqt}.
		Since this is the case for all placeholders, a \gls{uri}-to-placeholder mapping can be constructed,
		see \Cref{lst:sourceDomainMapping}.

		\begin{listing}[H]
			\inputminted{json}{src/listings/dataset-transformations/source-domain/mapping.json}
			\caption{Source domain: \glsentryshort{uri} to placeholder mapping.}
			\label{lst:sourceDomainMapping}
		\end{listing}

	\item Afterwards, the \gls{sparql} query can be transformed to a \gls{pgp} using the mapping,
		see \Cref{lst:sourceDomainPGP}.

		\begin{listing}[H]
			\inputminted{sparql}{src/listings/dataset-transformations/source-domain/pgp.sparql}
			\caption{Source domain: \glsentrylong{pgp} mapped from \glsentryshort{sparql}.}
			\label{lst:sourceDomainPGP}
		\end{listing}

	\item Finally, using the \glsentrylong{s2s} translator, the \gls{pgp} is converted to a \gls{squall} expression:
		
		\mint{text}{What is a <mouthpiece> of <Delta_Air_Lines> and has <instance_of> <periodical_literature>?}

\end{enumerate}

% Practical
Limited data engineering was needed to correct some queries,
while the implementation of the numerous template mappings used to create the \glspl{pgp} was more involved.
See \Cref{appendix:sciqa_sparqls} for an overview of the different templates in SciQa and their corresponding
\glspl{pgp}.

\subsubsection{Ground Truth Target Domain Data}
\label{s:gt_retriever_data}

% Intro
The ground truth target domain dataset ($\mathscr{D}_{\text{gt}}$) used for training the domain expert
(see \Cref{s:trainingDomainExpert}) includes natural language questions and \gls{squall} expressions.

SciQA dataset \cite{auerSciQAScientificQuestion2023} served as the foundation for $\mathscr{D}_{\text{gt}}$,
and is publicly available on Zenodo \cite{auerSciQABenchmarkDataset2023}.
The dataset comprises \num{2565} question-query pairs tailored to the \glsentrylong{orkg},
organized into eight distinct query templates.
The creation of SciQA involves a two-phase workflow.
In the initial phase,
a core set of 100 questions is manually crafted and translated into corresponding \gls{sparql} queries.
This set forms the basis for generating the remaining \num{2465} questions.
The second phase involves creating question and \gls{sparql} query templates with placeholders,
which are populated using all possible entities from the \gls{orkg}.
This structured approach allows for efficient query generation while ensuring variety and relevance.
A key distinction between SciQA and LC-QuAD 2 lies in their question generation methods:
while SciQA primarily includes questions generated by an \gls{llm},
LC-QuAD 2 relies on explicit human verbalization for its questions.

% Process
The process to create $\mathscr{D}_{\text{gt}}$ from SciQA is similar to the process of
creating $\mathscr{D}_S$ from LC-QuAD,
although an additional subgraph construction step is required,
and the placeholders cannot be retrieved from a \gls{nnqt}.
Consider a sample with this question:
\begin{minted}{text}
Provide a list of papers that have utilized the Depth DDPPO model and include the links to their code?
\end{minted}

\begin{enumerate}[leftmargin=*, labelindent=\parindent]

	\item From the sample's \gls{sparql} query the \glspl{uri} are extracted,
		see \Cref{lst:targetDomainSparqlToUri}.

		\begin{listing}[H]
			\mint{text}{SPARQL:}
			\inputminted{sparql}{src/listings/dataset-transformations/target-domain/query.sparql}
			\mint{text}{}	
			\inputminted{text}{src/listings/dataset-transformations/target-domain/uris.txt}
			\caption{Target domain: extracting \glsentryshortpl{uri} from a \glsentryshort{sparql} query.}
			\label{lst:targetDomainSparqlToUri}
		\end{listing}

	\item For each \gls{uri}, an \mintinline{sparql}{rdfs:label} is queried from the \gls{kg} such that a
		\gls{uri}-to-placeholder mapping can be constructed, see \Cref{lst:targetDomainMapping}.

		\begin{listing}[H]
			\inputminted{json}{src/listings/dataset-transformations/target-domain/mapping.json}
			\caption{Target domain: \glsentryshort{uri} to placeholder mapping.}
			\label{lst:targetDomainMapping}
		\end{listing}

	\item Then, the \gls{sparql} query can be transformed to a \gls{pgp} using the mapping,
		see \Cref{lst:targetDomainPGP}.

		\begin{listing}[H]
			\inputminted{sparql}{src/listings/dataset-transformations/target-domain/pgp.sparql}
			\caption{Target domain: \glsentrylong{pgp} mapped from \gls{sparql}.}
			\label{lst:targetDomainPGP}
		\end{listing}

	\item Finally, using the \gls{s2s} translator, the \gls{pgp} is converted to a \gls{squall} expression:

		\mint{text}{What is a <has_source_code> of a thing that has <has_model> a <Model> X and has a <has_benchmark> that has a <has_dataset> and X has a rdfs:label whose string matches 'Depth DDPPO'?}

	\item Construction of the question-relevant subgraph is done using the retrieval and post-retrieval components,
	as explained in \Cref{s:retrieval} and \Cref{s:postRetrieval}, respectively.

\end{enumerate}

For an overview of the different types of queries present in SciQA,
\Cref{appendix:sciqa_sparqls} can be consulted;
the running example specifically is listed under \Cref{lst:sciqa_7}.

% Practical
Due to the complexity of many queries in the SciQA dataset and current limitations of the translator tool
--- the inverse mapping from \gls{sparql} to \gls{squall} is not yet complete
--- many could not be converted to \gls{squall} automatically.
Therefore, the equivalent \gls{squall} templates for these queries were created manually.
To facilitate the process, some projection variables for certain templates,
whose inclusion was deemed superfluous, were removed
--- e.g., a second projection variable that represented the \mintinline{sparql}{rdfs:label} of the first.

% Formal
The dataset is formalized as:
\[
	\mathscr{D}_{\text{gt}} = \Set{ (s_i, H_i, Q_i, \hat{Q}_i, S_i) | i \in \mathcal{I}, s_i \in \Sigma^*, H_i \in V_{\mathscr{C}(G)} },
\]
where $\mathcal{I}$ is a countable index set.
Thus, each sample consists of a question, a constructed subgraph, a \gls{gp}, a \gls{pgp},
and a \gls{squall} expression.

\subsubsection{Synthetic Target Domain Data}
\label{s:syn_retriever_data}

% Intro
The synthetic dataset $\mathscr{D}_{\text{syn}}$ is constructed in the same way as $\mathscr{D}_{\text{gt}}$,
except that its \gls{squall} queries are not deterministically obtained through the procedure laid out in
\Cref{s:gt_retriever_data}.
Instead, they are generated using a \gls{pllm},
which is prompted to generate equivalent \gls{squall} given a question.
All else being equal,
only the method to obtain synthetic \gls{squall} is elaborated further upon in this section.
An example of the resulting prompt template is given by \Cref{lst:synthetic_data_generation_prompt}.

\begin{listing}[!ht]

	\inputminted{text}{src/listings/synthetic-data-generation-prompt.txt}

	\caption{
		Instantiation of the generation task used for synthetic data generation.
		One in-context demonstration is included.
	}
	\label{lst:synthetic_data_generation_prompt}
\end{listing}


The remainder of this section formalizes the approach in accordance with a recent survey on \gls{llm}-driven synthetic
dataset generation \cite{longLLMsDrivenSyntheticData2024}.
While this formalization does not offer substantial additional practical utility,
it serves an important role in aligning with a theoretical framework that promotes academic rigor and,
crucially, ensures consistency.  

% Formalization  
The synthetic data generation task is formulated as follows.
Seed samples from the ground truth dataset are provided as supporting information
consisting of questions and their respective \gls{squall} mappings:
\[
	\mathscr{D}_{\text{sup}} = \Set{ (s_i, S_i) | i \in \mathcal{I_{\text{sup}}} \subseteq \mathcal{I} },
\]
The generation task is then noted as:
\[
	\mathscr{S}_{\text{syn}} \leftarrow \mathcal{M}_p(\mathcal{S}, \mathscr{D}_{\text{sup}}),
\]
where $\mathcal{M}$, $p$ and $\mathcal{S}$ are the \gls{pllm}, prompt function and sequence of questions from
$\mathscr{D}_{\text{gt}}$, respectively.
Ideally $\mathscr{S}_{\text{syn}}$ is as similar to $\mathscr{S}_{\text{gt}}$ as possible.
The data generation workflow is straightforward and restricted to prompt engineering,
whose usual key elements are instantiated as:
\begin{itemize}

	\item \textbf{Task Specification}:
		$e_{\text{task}} =$ \mintinline{text}{Given a Question, generate equivalent SQUALL}.

	\item \textbf{Generation Condition}:
		$e_{\text{condition}} = $ \mintinline{text}{Follow the requirement to only return SQUALL}.

	\item \textbf{In-Context Demonstrations}:
		$e_{\text{demo}} = $
		$ {\underset{(s, S) \in \mathscr{D}_{\text{sup}}}{\bigodot} }$
		\mintinline{text}{Question: } $ s \quad $ \mintinline{text}{SQUALL: } $ S $,

\end{itemize}
where $\bigodot$ is the large operator for string concatenation. 
They are wrapped together through a prompt template function:
\[
	E: (\Sigma^*)^3 \rightarrow \Sigma^*, (s_1, s_2, s_3) \mapsto E(s_1, s_2, s_3).
\]
The prompt function is noted as:
\[
	p(s, \mathscr{D}_{\text{sup}}) \leftarrow E(e_{\text{task}}, e_{\text{condition}}, e_{\text{demo}}),
	\quad s \in \mathcal{S},
\]
which given a sample question $s$ and the support set, maps to an instantiation of the generation task,
i.e., the prompt specific to $s$ used to generate its, ideally, equivalent \gls{squall}.
The prompt previously given was an example of such an instantiation (see \Cref{lst:synthetic_data_generation_prompt}).


Supports set are generally included to increase the faithfulness of the generated data,
since they can be effectively used by the \gls{pllm} due to its \gls{icl} capabilities
\cite{longLLMsDrivenSyntheticData2024}.

\subsection{Retriever Initialization and Evaluation}
\label{s:retriever_init_eval}

An ideal retriever, i.e., perfect \gls{pcst} parameters were implicitly assumed in \Cref{s:retrieval}, but in practice,
it's important to define what constitutes good performance for the retriever,
and to establish a method for evaluating this performance.
This allows the parameters to be adjusted for better results.

The problem of subgraph retrieval has been previously discussed by \cite{zhangSubgraphRetrievalEnhanced2022}
in the context of multi-hop \gls{kgqa}.
Certain results were empirically shown that still hold despite \gls{squall} generation being a distinct task.
First, reasoning over a question-relevant subgraph instead of the entire \gls{kb} is more performant
due to limiting the reasoning space.
Second, a subgraph that is too small for \gls{qa} may fail to contain the answer,
whereas an overly large graph introduces excessive noise, which can degrade performance.
The informativeness of retrieved subgraphs was assessed using the answer coverage rate,
an evaluation function that measures the proportion of retrieved content containing the correct answer.
In essence, this metric provides a specific way of evaluating recall.
\Cref{s:recall} elaborates on recall in the context of the text-to-\gls{sparql} task.

The initialization of the \gls{pcst} parameters in the context \gls{kg}-subgraph construction,
cannot be decoupled from the \gls{kg} itself.
Therefore, their impact is empirically evaluated, see \Cref{s:retrievalAnalysis}.

\subsection{Learning}
\label{s:training_finetuning}

This section provides a detailed explanation of the sequential training strategy,
thoroughly outlining its two distinct steps.

% Training strategy
\subsubsection{Sequential Training}

The sequential training strategy consists of two key steps:
fine-tuning the \gls{squall} expert and training the domain expert,
illustrated in \Cref{fig:domain_expert}.
In the first step, the \gls{squall} expert
--- an \gls{llm}
--- is fine-tuned to generate \glspl{pgp} for questions within a source domain.
This tailored fine-tuning equips the model with specialized knowledge for generating queries in \gls{squall}.
Subsequently, the fine-tuned \gls{squall} expert is integrated into the domain expert
--- a \gls{rag} model
--- as the generator component.
The domain expert is then trained to perform the same task, but in a different target domain.
Crucially, during this second training phase, the \gls{squall} expert remains frozen,
preserving its acquired knowledge of \gls{squall} and avoiding catastrophic forgetting.

The ensuing sections provide details into the specifics of each training step.

\subsubsection{Fine-Tuning of the SQUALL Expert}
\label{s:fineTuningSquallExpert}

The LC-QuAD 2.0 dataset was used as the basis for the source domain dataset ($\mathscr{D}_S$).
First, a validation set was created by splitting 20\% from its training set using stratified sampling based on query
templates to maintain a template distribution consistent with the original training set.
The resulting split was: \num{15434}, \num{3859} and \num{4781}.

% Data preprocessing
Then, the template distribution was optimized by grouping samples with similar query templates,
such as the nearly identical templates:

\begin{minted}{text}
?E is_a Type. ?E pred Obj. ?E-secondClause value. MIN (value)
?E is_a Type. ?E pred Obj. ?E-secondClause value. MAX (value)
\end{minted}

The data was then resampled to achieve a more balanced distribution of query templates.
This step aimed to mitigate overfitting to specific query types and enhance the model's adaptability to different domains.
By balancing the dataset, the model is less likely to overfit to query structures it encounters frequently during training.

This approach is supported by recent findings \cite{dialloComprehensiveEvaluationNeural2024},
which highlight that \glspl{llm} often associate specific query structures with types of question structures during training.
When confronted with a similar question structure, \glspl{llm} tend to generate the query structure they have most often
encountered with that input.
Balancing the query template distribution reduces this bias, promoting a more generalized and flexible generation process.

% Prompt
The instruction passed to the \gls{llm} during fine-tuning follows the  Alpaca Prompt Template,
see \Cref{lst:finetune_prompt}.\footnote{
	Alpaca Prompt Template from \url{https://github.com/tatsu-lab/stanford_alpaca}.
}

\begin{listing}[!ht]

	\inputminted{text}{src/listings/finetune-prompt.txt}

	\caption{Fine-tuning example following the Alpaca Prompt Template.}
	\label{lst:finetune_prompt}
\end{listing}

\subsubsection{Training of the Domain Expert}
\label{s:trainingDomainExpert}

The SciQA dataset was used as the basis for the target domain dataset ($\mathscr{D}_T$).
Its train, validation and test splits contain \num{1795}, 257, and 513 samples, respectively.
These splits were consistently applied across all experiments.
Where necessary and feasible, minor data engineering was carried out to rectify errors;
otherwise, problematic samples were withheld.
Finally, the queries from the SciQA dataset were executed on a local endpoint of the associated \gls{kg}
(see \Cref{s:kg_version}), and those without response were removed.
After cleaning SciQA, the distribution of templates in its test set is
approximately identical to the train and validation template distributions,
see \Cref{table:sciqaDistribution}.

\begin{table}[ht]
	\centering
	\begin{tabular}{l|S[table-format=2.2]S[table-format=2.2]S[table-format=2.2]S[table-format=2.2]S[table-format=2.2]S[table-format=2.2]S[table-format=2.2]S[table-format=2.2]}
		\textbf{Template}              & \textbf{1} & \textbf{2} & \textbf{3} & \textbf{4} & \textbf{5} & \textbf{6} & \textbf{7} & \textbf{8} \\ 
		\hline
		\textbf{Percentage of Samples} & 13.01 & 13.21 & 14.84 & 16.06 & 19.31 & 3.05 & 20.33 & 0.19 \\ 
	\end{tabular}
	\caption{SciQA test split query template distribution.}
	\label{table:sciqaDistribution}
\end{table}

% Data Scarcity
In order to realize the training of the graph embedder and projector under the assumption of data scarcity,
inspiration was taken from related work \cite{sachanQuestionsAreAll2023},
that proposes training a retriever solely relying on questions.\footnote{
	Questions that in turn might be generated themselves.
}
While distinct in approach, this work demonstrates the potential of starting with just a set of questions.
It served as the motivation to generate the synthetic dataset $\mathscr{D}_{\text{syn}}$.

Finally, training of the domain expert was based on previous open-sourced code
\cite{heGRetrieverRetrievalAugmentedGeneration2024}.
Changes include the use of the fine-tuned \gls{squall} expert and the use of one graph for all question,
i.e., the \glsentryfull{orkg}.

\section{Evaluation}

Validating the system requirements (\Cref{s:validation}) and answering the research question (\Cref{s:introduction})
require critical evaluation of the approach, both are first restated here.

\paragraph{System Requirements}

\begin{enumerate}[label=SR\arabic*.]
	\item Minimize data requirements.
	\item Maintain simplicity.
	\item Facilitate comprehensive report creation.
	\item Enable human oversight.
\end{enumerate}

\paragraph{Research Questions}

\begin{enumerate}[label=RQ\arabic*.]
	\item How does the performance of the proposed domain expert compare to other approaches across various data
		availability scenarios?
	\item Does the domain expert effectively utilize the retrieved information with which it is soft-prompted,
		and what factors influence the effectiveness of these prompts?
	\item What is the impact of using synthetic data, as opposed to ground truth data,
		on domain adaptation performance during the second learning phase of sequential training?
\end{enumerate}

Given that the last three system requirements are of a boolean nature,
and that their satisfaction is dependent on the text-to-\gls{sparql} performance of the system,
emphasis is further put on the primary system requirement and the research questions.
They can be quantitatively evaluated using four typical performance indicators:
\gls{bleu}-score, execution (or query) accuracy and $F_1$ score \cite{dialloComprehensiveEvaluationNeural2024}
--- the former originating from \gls{nmt} and the latter two from \gls{qa}
--- as well as recall
--- a classic \gls{ir} statistic.
Two additional evaluation functions were considered but are not discussed further due to their unsuitability.

\gls{bleu} was employed for most evaluations, including assessing the quality of $\mathscr{D}_{\text{syn}}$,
the performance of the domain expert, the overall text-to-\gls{sparql} component, and comparisons to related work.
Execution accuracy and $F_1$ score were used exclusively to evaluate the performance of the text-to-\gls{sparql}
component and to facilitate comparisons with related work.
Lastly, recall was solely utilized to analyze the retriever's performance.

\subsection{BLEU}

The corpus-level \gls{bleu} score from NLTK is used to evaluate the quality of machine
translations between natural languages,
though it is also commonly applied to assess translations from natural to formal languages.\footnote{
	NLTK: \url{nltk.org}.
}
In this context, \gls{bleu} serves to evaluate the generated queries against a baseline prior to the linking step.
Since linking is inherently imperfect, isolating the system's performance up to that stage provides valuable insights.

\subsection{Execution Accuracy}

Execution accuracy \cite{zhongSeq2SQLGeneratingStructured2017} is the ratio of
queries with correct results when executed $N_c$ to the total number of queries $N$, i.e., $N_c/N$.
The relevancy of this ratio deems from the no-code perspective introduced in \Cref{s:introduction}.
A shortcoming is the possibility of false positives: a set of queries might return identical results.
To address this, the $F_1$-score is also looked at,
it offers a more nuanced evaluation by considering both precision and recall.

\subsection{$F_1$ Score}

\gls{sparql} queries regularly return multiple results.
Precision $p$ is defined as the ratio of relevant retrieved results to the total number of retrieved instances,
while recall $r$ is the ratio of relevant retrieved results to the total number of relevant instances.
In this context, retrieved instances are those returned by the generated \gls{sparql} query,
and relevant instances are those present in the ground truth.
The $F_1$ score, calculated as:
\[
	F_1 = \frac{2pr}{p+r},
\]
is computed for each question and then averaged across the entire dataset,
resulting in the average $F_1$ score \cite{yihValueSemanticParse2016}.

\subsection{Recall}
\label{s:recall}

Recall was employed to provide insight into the constructed subgraphs,
it essentially evaluates how well a retrieved subgraph corresponds to its query.
In general \gls{ir} terms it is the ratio of relevant retrieved instances to all relevant instances.
Here, the instances are vertices and edges, and they are separately evaluated.
The ratios are denoted by vertex and edge recall,
and respectively measure the fraction of vertices and edges from a \gls{sparql} query that appear in the subgraph.
This evaluation confirms the intuitive notion that more generous \gls{pcst} parameter settings lead to more comprehensive
subgraphs (i.e., those that cover a larger portion of the graph's vertices and edges).
However, while this approach can indicate the extent of coverage,
it does not guarantee increased informativeness due to the growing size of subgraphs
and the inherent limitations of \glspl{llm}.

\begin{definition}
	\textit{Vertex Recall} and \textit{Edge Recall} are defined as:

	\begin{align}
		R_V(Q, H) &= \frac{ | \Set{ v \mid v \in V_Q \land v \in V_H } | }{ |V_Q| }, \\
		R_E(Q, H) &= \frac{ | \Set{ e \mid e \in E_Q \land e \in E_H } | }{ |E_Q| },
	\end{align}

	where $Q$ is a \gls{gp} and $H \in V_{\mathscr{C}(G)}$ with $G = (V_G, E_G)$ the \gls{kg}.
	Given a dataset $\mathscr{D}$ consisting of samples that at least include a \gls{pgp} and subgraph,
	the average recall is:

	\begin{align}
		\wideoverbar{R}_{V, \mathscr{D}}
			&= \frac{1}{|\mathscr{D}|} \sum_{(Q, H) \in \mathscr{D}} R_V(Q, H), \label{eq:avgVertexRecall} \\
		\wideoverbar{R}_{E, \mathscr{D}}
			&= \frac{1}{|\mathscr{D}|} \sum_{(Q, H) \in \mathscr{D}} R_E(Q, H). \label{eq:avgEdgeRecall}
	\end{align}

\end{definition}

\subsection{Graph Similarity}

While vertex and edge recall provide an initial measure of the quality of retrieved information,
they are limited in scope.
Specifically, these evaluation functions do not consider the size of the constructed subgraphs,
nor do they address reasoning and prompt size limitations inherent to \glspl{llm}.
Intuitively, semantically accurate queries should match with the \glsentrylong{kg}
making graph similarity a more comprehensive evaluation approach.\footnote{
	Indeed, graph databases perform \glsentryfull{gpm}
	--- the task of matching a \glsentrylong{gp} against a graph
	--- to query, for example, \gls{rdf} graphs \cite{thakkarIntegratedGraphAlgebra2019}.
}
The relationship between queries and \gls{kg} subgraphs has been previously studied
\cite{zouGStoreAnsweringSPARQL2011, zouGraphBasedRDFData2017, hanKeywordSearchRDF2017}.
This section examines the \gls{ged} and \gls{mcs} graph similarity functions;
however, experimental results revealed significant limitations, rendering them unsuitable for this task.

\paragraph{Graph Edit Distance}

The \glsentryfull{ged} is defined as the minimum cost required to transform a graph $G_1$ into an isomorphic graph $G_2$
through a series of vertex and edge edit operations.
This similarity function accounts for the size difference between the query graph and the subgraph,
operates as a non-learning algorithm,
and allows for configurable edit operation costs.
However, the size of the constructed subgraphs, ranging from approximately 16 to \num{9000} vertices,
and the significant size discrepancy compared to the query-derived graphs (i.e., \glsentrylongpl{gp}),
which consist of only 9 to 17 vertices (refer to the \glspl{gp} in \Cref{appendix:sciqa_sparqls}),
make \gls{ged} computationally infeasible.
For comparison, experiments with the reference implementation in the NetworkX library
\cite{abu-aishehExactGraphEdit2015} were limited to graphs with an average of 5 to 25 vertices.
\footnote{NetworkX: \url{networkx.org}}
NetworkX's approximate \gls{ged} implementation also proved insufficient for addressing the scalability challenges.

\paragraph{Maximum Common Subgraph}

\glsentryfull{mcs} identifies the largest subgraph common to two graphs
It does not account for differences in graph size.
Consequently, it exhibits a tendency to favor the quantity of information,
similar to recall metrics.
For instance, the most similar subgraph under \gls{mcs} will invariably be the entire \gls{kg} itself,
regardless of query specificity.
Furthermore, the computational intractability of exact \gls{mcs} \cite{royMaximumCommonSubgraph2022}
further limits its applicability, particularly for large-scale graphs.

\section{Conclusion}

This methodology chapter has outlined the development and validation process of the proposed model reporting paradigm.
This involved a detailed examination of the system architecture, design decisions,
and the evaluation strategy tailored to the openCAESAR framework.
The technical depth provided serves as the foundation for comprehensively assessing the paradigm's effectiveness and 
applicability to \gls{mbse}.

The upcoming experimental chapter provides a thorough empirical evaluation of the proposed methodology,
offering critical insights into its validity and practical applicability.
This analysis not only assesses the methodology's contributions
but also contextualizes them within the broader research landscape.


\chapter{Experiments}
\label{s:experiments}

This chapter presents the experiments conducted to validate the requirements outlined in \Cref{s:validation}.
It begins with a detailed explanation of the experimental setup,
followed by five key experiments and their results.
The approach is then benchmarked against related work from the literature.
The chapter concludes with a discussion of the findings and their broader implications.

\section{Experimental Setup}

This section elaborates on the configuration of the \glsentrylong{kg},
the specific models utilized along with their respective settings,
and the software and hardware resources employed during the experiments.

\subsection{Knowledge Graph and Benchmark}
\label{s:knowledge_graph_and_benchmark}

\gls{orkg} was chosen to evaluate the proposed approach,
it is a scientific knowledge graph consisting of papers and authors.\footnote{ORKG: \url{orkg.org}.}
As of writing the \gls{kg} consists of \num{904212} unique entities,
although for this work the choice was made to work with an older version
--- because the SciQA benchmark, introduced in \Cref{s:gt_retriever_data}, that targets \gls{orkg} is outdated
--- consisting of \num{180184} and \num{6888} unique entities and relations, respectively.

\subsubsection{Motivation}

The subsequent paragraphs explain why the choice was made to work with this \gls{kg} and its associated benchmark.

\paragraph{Domain}

\gls{orkg} and SciQA are specialized, i.e., tailored to a specific instead of a general domain.
This is in line with the ontologies that occur in \gls{mbse}.
A \gls{kg} with no particular focus would be less appropriate.

\paragraph{Semantics}

As discussed in \Cref{s:training_finetuning}, the \gls{squall} expert was fine-tuned using the LC-QuAD 2.0 dataset,
which is designed for the Wikidata \gls{kg}.  
Despite substituting all \glspl{uri} with placeholders 
--- mapping \glspl{gp} to \glspl{pgp}
--- before training, the semantic information of the original \gls{kg} remains embedded in the queries.
This persistence occurs because the original queries are inherently structured to align with the \gls{kg};
otherwise, they would yield no results upon execution.\footnote{
	\gls{sparql} is a graph query language where queries are represented as \glsentryfullpl{bgp} or \glsentryfullpl{cgp}
	--- both of which are ultimately graphs (see \Cref{fig:sparqlBGP}).
	Executing a query involves matching a \gls{gp} against a \gls{kg},
	a process known as \glsentryfull{gpm} or subgraph matching.
	Consequently, any dataset of queries designed for a specific \glsentrylong{kg} will include \glspl{gp}
	with topologies that align with the structure of that graph.
}
Overlap between datasets can occur; for instance,
both DBpedia and Wikidata include the \mintinline{text}{marriage} relation between two \mintinline{text}{human} entities.
To mitigate such similarities, the approach is evaluated on a specialized domain.
If evaluation were instead conducted on DBpedia while the \gls{squall} expert was trained on Wikidata,
the results could appear overly optimistic.

\paragraph{Version}
\label{s:kg_version}

SciQA is out of date with the current version of \gls{orkg}: only 10 out of \num{2565} queries returned results.
Hence, the version of February 14, 2023 was used
--- \gls{rdf} dump made available at \cite{auerSciQABenchmarkDataset2023}
--- which returns query results in \approx 92\% of cases,
and contains \num{181147} entities and \num{6888} properties.
It was made accessible through a local \gls{sparql} endpoint using Apache Jena Fuseki.

\paragraph{Size}
\label{s:size_kgb}

The survey of ontology engineering projects mentioned in the introduction of \Cref{c:proposedSolution}
also gives insight into the size of real-world ontologies \cite{simperlAchievingMaturityState2009},
where ontology size is taken to be the number of unique entities.
The size of the chosen \gls{kg} is considered to be sufficiently large to be representative,
see \Cref{fig:ontologySizes}.

\begin{figure}[h!]
    \centering
    \includegraphics[width=\textwidth]{ontology-size-distribution.pdf}
	 \caption{
	 	Distribution of ontology sizes (i.e., number of entities)
		from a survey \cite{simperlAchievingMaturityState2009}.
	}
    \label{fig:ontologySizes}
\end{figure}

\subsubsection{Templates}
\label{s:templates}

% Data example
To better understand the structure of queries in the SciQA dataset,
\Cref{fig:combinedSciQAListings} provides two example \gls{sparql} queries,
and the equivalent \gls{squall} expressions.
Note that while the questions and \gls{sparql} queries are from SciQA,
the \gls{squall} expressions are constructed as part of this work.

\begin{listing}[ht!]

	\mintinline{text}{SPARQL:}
	\inputminted{sparql}{src/listings/running-example/pgp.sparql}
	\mintinline{text}{}

	\mintinline{text}{SQUALL:}
	\inputminted{text}{src/listings/running-example/squall.txt}
	\mintinline{text}{}

	\mintinline{text}{===}
	\mintinline{text}{}

	\mintinline{text}{SPARQL:}
	\inputminted{sparql}{src/listings/sciqa-example-b/pgp.sparql}
	\mintinline{text}{}

	\mintinline{text}{SQUALL:}
	\inputminted{text}{src/listings/sciqa-example-b/squall.txt}
	\mintinline{text}{}

	\caption{Example question-query pairs from SciQA with equivalent \glsentryshort{squall} expression.}
	\label{fig:combinedSciQAListings}
\end{listing}

One notable difference between the two examples is the presence of an \mintinline{sparql}{OPTIONAL} clause
in the bottom one.
This difference is also reflected in the corresponding \gls{squall} expression,
where it is represented by \mintinline{text}{if defined}.

Clearly, these two queries have distinct structures.
However, due to the way the SciQA dataset is created, see \Cref{s:gt_retriever_data},
there are only a limited quantity of such unique query structures, called templates, present, namely eight.
Note that this is typically the case for these types of datasets, e.g., LC-QuAD.
Although queries will vary, for example, in terms of which literals are present, the structure they have will always
belong to one of eight templates.

These examples highlight the distinct structures of the queries.
However, the SciQA dataset, as explained in \Cref{s:gt_retriever_data},
is constructed with a limited number of unique query structures, referred to as templates.
Specifically, there are eight such templates in total.
This is a common characteristic of datasets in this domain, such as LC-QuAD.
While the specifics (e.g., literals) of queries in the dataset may vary,
their underlying structure always conforms to one of these templates.

The query templates in SciQA are denoted by the set $T = \Set{ 1, \ldots, 8 }$.
When a specific template $x \in T $ is excluded, this is represented as $ T \setminus \{x\} $.

For an example of each template, the appendix can be consulted, see \Cref{appendix:sciqa_sparqls}.

\subsection{Model Configurations and Settings}

In the paragraphs ahead practical choices related to the implementation and experimentation are summarized.

\paragraph{Retrieval}
\label{s:settings_retrieval}

Before training the domain expert, the \gls{kg} index ($\mathscr{I}_{\text{\gls{kg}}}$) is first built using
Sentence-BERT \cite{reimersSentenceBERTSentenceEmbeddings2019} as embedding model ($\lambda$)
with the dimension set to \num{1024}.
The similarity function ($\sigma$) used for indexing was the cosine similarity.

\paragraph{Augmentation}

Following \cite{heGRetrieverRetrievalAugmentedGeneration2024}, the specific graph embedder used for embedding the
subgraphs was the \glsentrylong{gat} \cite{velickovicGraphAttentionNetworks2018},
with 4 layers, a hidden dimension ($d_g$) of \num{1024}, 4 heads and a dropout of 0.
The projector is implemented as a \glsentrylong{mlp} with 2 layers, a hidden dimension of \num{2048} and
the output dimension ($d_t$) is \num{4096}, aligned with the hidden dimension of the \gls{llm} used in the generator.
Some experimentation was done where the entire \gls{kg} served as graph token,
which required reducing the dimensions of the graph embedder, though this line of investigation was not further pursued.

\paragraph{Generation}

The \gls{llm} used for the generator component was Llama2-7b \cite{touvronLlama2Open2023}.
It served as the \gls{squall} expert that was frozen after being fine-tuned in all the experiments.
More specifically, a LLaMA-Adapter \cite{zhangLLaMAAdapterEfficientFinetuning2023} is fine-tuned.
The \gls{peft} technique that was used was \gls{qlora} \cite{dettmersQLoRAEfficientFinetuning2023}.
Parameter settings were informed by standard fine-tuning approaches for LLaMA v2.\footnote{
	The fine-tuning procedure was guided by the tutorial available at
	\url{https://www.kdnuggets.com/fine-tuning-llamav2-with-qlora-on-google-colab-for-free}.
}

\paragraph{Domain Expert}

Observing diminishing returns from initial experiments, patience was set to 3,
with all other parameters following the settings described by \cite{heGRetrieverRetrievalAugmentedGeneration2024}.

\paragraph{Linking}
\label{s:settings_linking}

The similarity function ($\sigma$) used during linking was kept from the earlier implementation
\cite{omarUniversalQuestionAnsweringPlatform2023}, 
it is based on the \glsentryfull{cf} \cite{castrofernandezSeepingSemanticsLinking2018}.
Given a placeholder string $s$ and an \mintinline{sparql}{rdfs:label} of a vertex or edge $x$,
they are represented as sequences of words $t$:
\[
	t = (w_i)_{i \in I}. \\
\]
The sequences are embedded using an operator $T$, here defined as:
\[
	T: t \mapsto \Set{ W_i^{\phi} =
		\begin{cases}
			\lambda(w_i) & w_i \in \text{dom}(\lambda) \land \phi = \lambda, \\
			\delta(w_i)  & w_i \notin \text{dom}(\lambda) \land \phi = \delta,
		\end{cases} \ | i \in I
	}.
\]

%\TODO{Fix citation}
In practice $\lambda$ and $\delta$ are the fastText \cite{bojanowskiEnrichingWordVectors2017}
and chars2vec model, respectively.\footnote{
	Chars2vec available at \url{https://github.com/IntuitionEngineeringTeam/chars2vec}.
}
The latter serves as a fallback option if the former fails to recognize the input.
The similarity is then defined as:
\[
	\sigma(s, x) = f_{\glsentryshort{cf}}(T(s), T(x)).
\]
Let $ \mathcal{S} = T(s) $ and $ \mathcal{X} = T(x) $,
then the \gls{cf} between sets of embeddings is defined as follows:
\[
	f_{\gls{cf}}(\mathcal{S}, \mathcal{X})
	= \frac{1}{|\mathcal{S}||\mathcal{X}|}
	\sum_{\left(S^{\phi_s}, X^{\phi_x}\right) \in \mathcal{S} \times \mathcal{X}}
	\sigma_{*}\left(S^{\phi_s}, X^{\phi_x}\right),
\]
where 
\[
	\sigma_{*}\left(S^{\phi_s}, X^{\phi_x}\right)
	= \mathbb{1} _ { \Set{ \phi_s = \phi_x } }
	\left(\sigma_{\text{cos}}\left(S^{\phi_s}, X^{\phi_x}\right)\right),
\]
using the indicator function to define embeddings originating from different models as perpendicular.
Linking back to the similarity function used for indexing, they are evidently quite distinct.
The works that proposed them in their original context either did not explore or
report any significant performance impact using alternatives.
As a comprehensive evaluation of different similarity functions was deemed out of scope,
both functions were retained as originally implemented.

\paragraph{Pre-Trained Large Language Model}

The OpenAI \gls{api} was used to create the synthetic datasets.

\paragraph{Software and Hardware}

The implementation uses a server designed to process incoming questions,
interact with the \gls{sparql} endpoint of the \gls{kg}, and return results.
This server is constructed using a lightweight Python Flask
application and is hosted on Apache with an \gls{aws} environment.
Both the server and the query endpoint are deployed on the same
\gls{aws} EC2 instance.

\subsection{PCST Parameter Triple}

The subsequent sections make use of a shorthand notation to indicate the parameters of the
\glsentryfull{pcst} algorithm.
The algorithm is used to construct relevant graphs for a given input question,
and it is implemented as the retrieval and post-retrieval components.

The parameter triple \((k_\text{v}, k_\text{e}, C)\) specifies
the top-\(k\) retrieved vertices (\(k_\text{v}\)),
the top-\(k\) retrieved edges (\(k_\text{e}\)),
and the cost per edge (\(C\)),
as defined in \Cref{eq:topkVertices}, \Cref{eq:topkEdges}, and \Cref{eq:graphCost}.

In practice, $k_\text{v}$ and $k_\text{e}$ are always equal and are collectively referred to as $k$.

\subsection{Notebook and KGQA Pipeline Connection}

The interaction between the notebook environment and the \gls{kgqa} pipeline is facilitated through a custom Python
package that introduces a user-friendly magic command for posing questions.
Developed by François Goybet as part of his bachelor's project in tandem with this master's thesis,
this package enhances usability by enabling seamless communication with the \gls{kgqa} pipeline hosted on an \gls{aws}
server.\footnote{
	Python package available at \url{https://pypi.org/project/molab-ext}.
}

Once the custom magic is loaded in a notebook cell, users can input a natural language question.
Upon execution, the question is sent to the pipeline, which automatically handles all intermediate steps
--- ranging from question parsing to result retrieval.
This process eliminates the need for stakeholders to interact directly with the underlying pipeline,
providing an intuitive interface for querying the system.

This extension, tailored to the requirements of this work,
was independently developed but designed to integrate seamlessly into the research setup.
The connection allows users to leverage the pipeline without requiring extensive technical knowledge.
This integration not only simplifies stakeholder interaction with the \gls{kgqa} pipeline
but also significantly enhances the accessibility and practicality of the system.

%%%%%%%%%%%%%%%

%%%% Experiments

% Hypothesis: what
% Hypothesis: why
% Actual results
% If unexpected: insight why this might be

\section{Retrieval and Post-Retrieval Analysis}
% Parameter exploration

% Intro
The first experiment investigates the retrieval and post-retrieval components applied to \gls{orkg},
introduced in \Cref{s:retrieval} and \Cref{s:postRetrieval}, respectively.

% Recap
These two components aim to construct a relevant subgraph of the \gls{kg} for a given question 
that can be used to improve the performance of the domain expert.
%
The running example from \Cref{c:methodology} is reintroduced here, see \Cref{fig:runningExamplePostRetrieval}.
Consider the following question:
\mintinline[breaklines]{text}{Provide a list of papers that have utilized the Depth DDPPO model and include the links to their code?}
When inputted to the retriever it will return relevant vertices and edges of the \gls{kg},
for example, there might be a vertex \mintinline{text}{Model} and an edge \mintinline{text}{HAS_SOURCE_CODE}.
The post-retrieval component is then responsible for creating a connected graph out of these looses vertices and edges.

% Why
The primary objective of retrieval is to be as informative as possible.
However, two inherent limitations of any \gls{rag} paradigm are poor and over-retrieval,
see \Cref{s:why_placeholders}, both negatively impacting generation.
%
This implies that a poor parameter choice likely reduces the performance of the domain expert.
%
Ideally, the subgraphs at the end of post-retrieval capture all the necessary vertices and edges needed for the 
domain expert to correctly generate the expected \gls{squall}, while remaining as compact as possible.
Hence, it is important to have some notion of how the parameters of the retrieval and post-retrieval components
influence the constructed subgraphs.

% What
What is being evaluated is essentially how informative the subgraphs are in function of the
\gls{pcst} parameters.

\subsection{Hypothesis}

% Hypothesis: what
Increasing the top-$k$ parameter is expected to result in larger subgraphs with higher recall for both entities and
relations.
The cost per edge is anticipated to provide fine control over subgraph size,
for example, increasing the cost per edge will likely contribute to noise reduction by steering the algorithm
towards smaller subgraphs.

% Hypothesis: why
The retriever's parameters are presumed to have a direct impact on the quality of the constructed subgraphs.
Larger subgraphs are likely to capture more of the relevant vertices and edges.
However, excessive size may introduce unnecessary noise.
A higher cost per edge is expected to mitigate noise,
focusing the subgraphs on the most informative components while maintaining high recall.

\subsection{Setup}
\label{s:retrievalAnalysis}

% How
Evaluation is performed by varying the \gls{pcst} parameters and investigating how it impacts the subgraphs,
outputted by the post-retrieval component,
using average vertex/edge recall and average number of vertices/edges,
defined in \Cref{eq:avgVertexRecall} and \Cref{eq:avgEdgeRecall}, respectively.

\subsection{Results}
\label{s:retriever_parameters_results}

% Actual results
The results, see \Cref{table:retrieverRecall},
indicate that both the size of the constructed graphs and the recall increase with top-$k$,
confirming the straightforward intuition that more lenience in terms of which vertices and edges are considered 
relevant, results in bigger constructed subgraphs containing at least as many of the expected vertices and edges.
Evidently, if the entire graph would be returned from the retriever, recall would be equal to 1.
As for the cost per edge, increasing it reliably results in a reduced subgraph size, all else being equal.
The parameter seems especially valuable for higher top-$k$ values, since the loss in recall reduces (to zero),
while resulting in about half the amount of vertices/edges.

\begin{table}[t]

	\centering
	\setlength{\tabcolsep}{3pt} % Default is 6pt

	\begin{tabular}{l|SS|SS}
		\textbf{Parameters}
		& \textbf{Average \# Vertices}
		& \textbf{Average \# Edges}
		& \textbf{Average Vertex Recall}
		& \textbf{Average Edge Recall} \\
		\hline
		(5, 5, 0.1)       & 26   & 28    & 0.00  & 3.57 \\
		(5, 5, 0.5)       & 16   & 15    & 0.00  & 3.57 \\
		(10, 10, 0.1)     & 49   & 56    & 15.98 & 26.36 \\
		(10, 10, 0.5)     & 35   & 37    & 3.57  & 16.07 \\
		(30, 30, 0.1)     & 197  & 255   & 10.71 & 42.86 \\
		(30, 30, 0.5)     & 104  & 115   & 27.82 & 44.19 \\
		(100, 100, 0.1)   & 739  & 1150  & 21.43 & 78.57 \\
		(100, 100, 0.5)   & 386  & 477   & 42.72 & 65.08 \\
		(1000, 1000, 0.1) & 7316 & 11341 & 90.41 & 95.63 \\
		(1000, 1000, 0.5) & 3126 & 5166  & 88.75 & 95.23 \\

	\end{tabular}

	\caption{
		The average number of vertices and edges,
		and the average vertex and edge recall (expressed as percentages),
		in function of \glsentryshort{pcst} parameters.
	}
	\label{table:retrieverRecall}

\end{table}


% If unexpected: insight why this might be
Notably, the execution time of the \gls{pcst} algorithm remains stable even as the top-$k$ vertices and edges increase
and the cost per edge decreases, with only a slight increase in running time.

\section{Synthetic Dataset Quality}

% Intro
This experiment gauges the quality of the synthetic dataset generated using the rudimentary strategy,
as explained in \Cref{s:syn_retriever_data}, and used to train the domain expert, see \Cref{s:trainingDomainExpert}.

% Recap
A high-quality training dataset containing question-query pairs for a specific domain is often unavailable
in practice due to data scarcity.
This work addresses this challenge primarily through \glsentryfull{da} and the use of the \glsentrylong{cnl} \gls{squall}.
However, synthetic data can also play a vital role in mitigating data scarcity.
To explore this, the performance of the domain expert is compared when trained on ground truth data versus synthetic data,
as discussed in \Cref{s:generalizationCapabilities}.

% Why
A basic synthetic data generation approach was deliberately chosen 
--- advanced strategies are considered out of scope
--- nevertheless, evaluating the quality of the generated synthetic data remains essential for subsequent experiments,
as it provides critical insights when interpreting the results.

% What
Assessing the quality of generated synthetic data is straightforward when ground truths are available,
as is the case here, direct evaluation can be performed
which boils down to evaluating how faithful the generated samples are to their references 
\cite{longLLMsDrivenSyntheticData2024}, ideally they are identical.
See \Cref{lst:syntheticDatasetQualityEval} for an example.

\begin{listing}[!ht]

\mintinline{text}{Ground Truth SQUALL:}
\inputminted{text}{src/listings/running-example/squall.txt}
\mintinline{text}{---}

\mintinline{text}{High quality synthetic SQUALL:}
\inputminted{text}{src/listings/running-example/squall-synthetic-high-quality.txt}
\begin{minted}{text}
> BLEU score: 90
> One placeholder not faithful to reference:
> <has code link> instead of <has source code>. 
\end{minted}
\mintinline{text}{---}

\mintinline{text}{Low quality synthetic SQUALL:}
\inputminted{text}{src/listings/running-example/squall-synthetic-low-quality.txt}
\begin{minted}{text}
> BLEU score: 23
> Wrong demonstration followed from prompt.
\end{minted}

\caption{
	Comparison of ground truth and synthetic \glsentryshort{squall} for a question,
	highlighting high-quality and low-quality examples,
	along with their \glsentryshort{bleu} scores.
}
\label{lst:syntheticDatasetQualityEval}
\end{listing}


\subsection{Hypothesis}

% Hypothesis: what
The higher the \gls{dpt} and the more templates are included in the prompt likely improves performance up to a certain
point.

% Hypothesis: why
Given that only eight templates are present in the SciQA benchmark, it seems improbable that the inclusion of a template
would confuse the \gls{pllm}.
Furthermore, increasing the \gls{dpt} gives more examples to the model, which should intuitively increase faithfulness.

\subsection{Setup}
\label{s:synthetic_dataset}

% How
Faithfulness between generated and reference \gls{squall} queries is evaluated using the \gls{bleu} score.
Elementary prompt engineering experiments are conducted to examine how variations in the support set
($\mathscr{D}_{\text{sup}}$) affect data quality.

Two engineering dimensions are explored:

\begin{itemize}

	\item \textbf{Template Variations}:

		As outlined in \Cref{s:templates}, SciQA queries belong to one of eight templates.
		This axis examines the effect of excluding certain templates from the support set.
		Three configurations are tested:

		\begin{itemize}
			\item All templates included: $T$.
			\item Excluding template 2: $T \setminus \{2\}$.
			\item Excluding template 6: $T \setminus \{6\}$.
		\end{itemize}

		These configurations enable evaluation of the faithfulness of generated samples when a template is absent from the
		prompt.

		Template 2 was excluded due to its significant dissimilarity to the other query templates,
		making it a strong candidate for assessing the impact of exclusion on quality.
		In contrast, templates 4 and 5 were the most similar to each other,
		it is thus likely that excluding one or the other would result in minimal deterioration of the synthetic data quality.
		Template 6 was excluded for a similar reason to template 2, as it also exhibited notable dissimilarity.

		The dissimilarity was calculated by grouping the queries from the ground truth dataset by template,
		extracting all tokens associated with each template, and computing TF-IDF scores.
		Pairwise cosine similarity was then calculated for all template combinations,
		with templates ranked from most similar to most dissimilar.
		Templates 6 and 2 appeared as the least similar templates in the ranking,
		being the penultimate and last templates, respectively, to form a pair.

		While the ideal scenario would involve testing one setup for each excluded template,
		practical constraints necessitated focusing on this subset.
		This selection was deemed most suitable for evaluating generalization to unseen and,
		likely, significantly (more) dissimilar templates in real-world applications.

	\item \textbf{Number of \glsentryfullpl{dpt}}:
		The second axis explores the impact of varying the number of \glspl{dpt}.
		Two setups are considered: 

		\begin{itemize}
			\item One \gls{dpt}.
			\item Three \glspl{dpt}.
		\end{itemize}

\end{itemize}

In total, these variations result in six setups with each a unique configuration of the support set
($\mathscr{D}_{\text{sup}}$).
For each configuration, corpus-level \gls{bleu} scores are computed across all samples and on a per-template basis.
The ground truth dataset ($\mathscr{D}_{\text{gt}}$) serves as the reference for evaluating all generated synthetic
datasets.

The only factor differentiating the six setups is the composition of $\mathscr{D}_{\text{sup}}$,
see \Cref{s:syn_retriever_data}.

\subsection{Results}
\label{s:synthetic_dataset_quality}

% Actual results
The results of the experiment are given in \Cref{table:synthetic_dataset_bleu}.

\begin{table}[t]
	\centering
	%\setlength{\tabcolsep}{6pt} % Default is 6pt
	\begin{tabular}{
		l|S[table-format=2.0] S[table-format=2.0] S[table-format=2.0]|S[table-format=2.0] S[table-format=2.0] S[table-format=2.0]
	}
		\textbf{Templates} 
					  & \multicolumn{3}{c|}{\textbf{1 Demonstration per Template}}
					  & \multicolumn{3}{c}{\textbf{3 Demonstrations per Template}} \\
					  
					  & {$T$} & {$T \setminus \{ 2 \}$} & {$T \setminus \{ 6 \}$} 
					  & {$T$} & {$T \setminus \{ 2 \}$} & {$T \setminus \{ 6 \}$} \\
					  \hline

		$T$ & 74 & 73             & 74             & 93  & 88             & 92 \\
		\hline
		1   & 65 & 71             & 69             & 98  & 99             & 97 \\
		2   & 81 & \underline{38}     & 81             & 94  & \underline{37} & 96 \\
		3   & 89 & 89             & 91             & 98  & 96             & 98 \\
		4   & 76 & 74             & 79             & 93  & 91             & 94 \\
		5   & 67 & 69             & 63             & 91  & 94             & 90 \\
		6   & 79 & 75             & \uline{51} & 97  & 93             & \underline{56} \\
		7   & 71 & 72             & 71             & 87  & 88             & 86 \\
		8   & 72 & 82             & 83             & 100 & 100            & 95 \\

	\end{tabular}

	\caption{
		Quality assessment of the synthetic dataset using \glsentryshort{bleu} scores expressed as percentages.
		This evaluation compares the generated \glsentryshort{squall} queries to their corresponding ground truths,
		including an analysis of queries generated with certain demonstrations excluded from the prompt's support set.
	}

	\label{table:synthetic_dataset_bleu}
\end{table}


What is being evaluated here is in essence the \gls{pllm}'s ability to annotate data \cite{zhuCanChatGPTReproduce2023}.
As expected, it is unequivocal that increasing the \gls{dpt} from one to three improved the faithfulness of the data,
although no further conclusion can be made with this limited setup.

Moreover, the removal of a template from the prompt clearly has a substantial impact on the data quality
for that specific label, but this adverse effect is not equal across all templates,
e.g., the quality of the generated samples following query template 2 in the $T \setminus \{2\}$ case
is notably worse than the quality of the generated samples following query template 6 in the $T \setminus \{6\}$ case.
This could be related to the fact that template 2 is more dissimilar to the remaining demonstrations than template 6 is,
see \Cref{s:synthetic_dataset}. 
Finally, the removal of one template from the prompt has no significant impact on the quality of generated samples
belonging to the remaining templates.

% If unexpected: insight why this might be
No further conclusions can be made from this limited setup.

\section{Text-to-SQUALL}

% Intro
This section introduces the first evaluation of the domain expert,
particularly its ability to adapt to new a domain that differs from the source domain used during fine-tuning of the 
\gls{squall} expert.

% Recap
The domain expert, see \Cref{fig:domain_expert}, is the \gls{rag} model that generates \gls{squall} given a question.
Note that these \gls{squall} expressions do not yet contain any \glspl{uri} of the target \gls{kg}.
In order to obtain an executable \gls{sparql} query, a \gls{squall} expression must still pass through the remainder
of the text-to-\gls{sparql} pipeline consisting of translation and linking,
see \Cref{fig:textToSparql}.

% Why
The objective of this experiment is to evaluate the performance of the domain expert in isolation,
separate from the rest of the pipeline.
This approach minimizes confounding factors, such as inaccuracies introduced by an imperfect linking procedure.
Additionally, the evaluation focuses solely on the domain expert's capabilities before incorporating synthetic data;
the model is trained exclusively on the ground truth dataset ($\mathscr{D}_{\text{gt}}$).

% What
The evaluation assesses the model's ability to generate \gls{squall} queries with placeholders,
comparing its outputs against ground truths for various \gls{pcst} parameter configurations.
While the first experiment (see \Cref{s:retrievalAnalysis}) provides preliminary insights into the influence of these
parameters,
this experiment extends those findings by directly examining their impact on the quality of the generated queries,
as discussed in \Cref{s:retriever_init_eval}.

\subsection{Hypothesis}

% Hypothesis: what
Intuitively, more informative subgraph should enhance performance,
particularly when the retrieval and post-retrieval components achieve higher average recall
and maintain smaller graph sizes on average.

% Hypothesis: why
This assumption is grounded in the expectation that
increasing relevant information and decreasing distracting noise improve the generation process.

\subsection{Setup}

% How
Evaluation is done using \gls{bleu} scores, comparing generated to reference \gls{squall}.

\subsubsection{Reducing the Number of PCST Parameter Configurations}

Some parameter configurations were discarded due to practical considerations, such as running time, 
as informed by the retrieval and post-retrieval analysis experiment (see \Cref{s:retriever_parameters_results})
and a heuristic assessment.

For each pair of \gls{pcst} parameter triples where \( k \) is identical but the cost-per-edge differs, 
one configuration was excluded from further investigation. 
Consider the following pair from \Cref{table:retrieverRecall}: (1000, 1000, 0.1) and (1000, 1000, 0.5). 
Experimental results indicate that while the former achieves slightly higher vertex and edge recall, 
it also produces subgraphs that are, on average, more than twice the size in terms of vertices and edges. 
This is a direct consequence of its lower cost-per-edge parameter, and demonstrates a case of diminishing returns.

% Formally
The heuristic for deciding which parameter triple from a pair ($p_1, p_2$) to keep, is defined as follows:

\[
	p^* =
	\begin{cases}
		p_1 & \text{if } \wideoverbar{R}_{V, \mathscr{D}_1} \gg \wideoverbar{R}_{V, \mathscr{D}_2}, \\
		p_2 & \text{if } \wideoverbar{R}_{V, \mathscr{D}_2} \gg \wideoverbar{R}_{V, \mathscr{D}_1}, \\
		p_1 & \text{if } \wideoverbar{V}_{\mathscr{D}_1}    <   \wideoverbar{V}_{\mathscr{D}_2}, \\
		p_2 & \text{if } \wideoverbar{V}_{\mathscr{D}_2}    \le \wideoverbar{V}_{\mathscr{D}_2}.
	\end{cases}
\]

Here, \( p^* \) denotes the retained parameter tuple. 
\( \wideoverbar{R}_{V, \mathscr{D}_i} \) and \( \wideoverbar{V}_{\mathscr{D}_i} \) represent
the average vertex recall, see \Cref{eq:avgVertexRecall}
and the average number of vertices, see \Cref{eq:avgVertices},
respectively, 
for the dataset resulting from retrieval and post-retrieval when the parameters from \( p_i \) were used.

In practical terms, a difference in recall was considered significant if it was at least 10. 
For the previous example, the difference in recall was only 1.66. 
As a result, the parameter triple (1000, 1000, 0.5) was selected, as it yields more compact subgraphs.

Finally, parameter triples with \( k = 5 \) were excluded, as they resulted in an average vertex recall of zero.
Instead, a baseline approach was employed.

\subsubsection{Baseline}

% Why?
The post-retrieval component outputs a graph,
that helps the domain expert during generation through the augmentation process,
see \Cref{fig:domain_expert}.
To assess the impact of different \gls{pcst} parameters on the generation process, a baseline is established.

% What?
The baseline was determined based on two key observations. 
First, the output of the post-retrieval component is a graph. 
Second, the largest and smallest graphs that the post-retrieval component could theoretically produce are
the entire \gls{kg} and the zero-order graph \( K_0 \), respectively. 
Here, \( K_0 \) refers to a graph containing neither vertices nor edges. 
While both extremes were tested, using the entire \gls{kg} proved unfeasible due to computational constraints.

% How?
This experiment used the baseline by providing \( K_0 \) as input to the augmentation process of the domain expert
during inference.

\subsubsection{Internal Evaluation}
\label{s:internalEvaluation}

It is important to emphasize that the \gls{bleu} scores obtained from evaluating the \gls{squall} expressions generated
by the domain expert are meaningful only within the scope of this work.
Typically, in \glsentrylong{sp}, \gls{bleu} scores are calculated between reference and generated \gls{sparql} queries.
However, the \gls{squall} expressions must still pass through the remainder of the text-to-\gls{sparql} pipeline before
becoming \gls{sparql} queries, as illustrated in \Cref{fig:textToSparql}.

Nevertheless, \gls{bleu} scores based on \gls{squall} expressions are employed as they enable an effective evaluation
of the domain expert.

Consequently, evaluations based on \gls{squall} expressions are consistently employed,
but strictly for internal evaluation purposes.

\subsection{Results}
\label{s:text_to_prelinked_squall}

% Actual results
Given that no previous results were found that detail the impact of input graph size or recall on graph prompting
multiple models were trained with differing \gls{pcst} parameters to test
their impact on the text-to-\gls{squall} task.
The results are shown in \Cref{table:text_to_prelinked_squall}.

\begin{table*}[t]
	\centering
	\begin{tabular}{l|S[table-format=2.0]S[table-format=2.0]S[table-format=2.0]S[table-format=2.0]S[table-format=2.0]}
		\textbf{Templates} & \textbf{Baseline ($\mathbf{K_0}$)} & $\textbf{(10, 10, 0.1)}$ & $\textbf{(30, 30, 0.5)}$ & $\textbf{(100, 100, 0.5)}$ & $\textbf{(1000, 1000, 0.5)}$ \\
		\hline
		$T$ & 98            & 94  & 82  & 91  & 95 \\
		\hline
		1  & 99             & 93  & 74  & 84  & 95 \\
		2  & 100            & 99  & 85  & 99  & 98 \\
		3  & 100            & 84  & 61  & 70  & 98 \\
		4  & 99             & 91  & 84  & 94  & 90 \\
		5  & 100            & 97  & 79  & 95  & 99 \\
		6  & \underline{40} & 53  & 56  & 87  & 50 \\
		7  & 100            & 100 & 99  & 99  & 94 \\
		8  & \text{N/A}     & \text{N/A} & \text{N/A} & \text{N/A} & \text{N/A} \\

	\end{tabular}
	\caption{
		Evaluation of the domain expert's performance on the text-to-\glsentryshort{squall} task using the \glsentryshort{bleu} score,
		expressed as precentages, in function of different \glsentryshort{pcst} parameters.
	}
	\label{table:text_to_prelinked_squall}
\end{table*}



% If unexpected: insight why this might be
Performance in the $K_0$ scenario is notably and unexpectedly high.
The fact that the best performance is achieved for $K_0$ 
is likely due to the relatively homogeneous nature of the SciQA dataset.
This homogeneity is characterized by recurring entities and relationships within the \gls{sparql} queries,
thus somewhat nullifying the potential benefit of retrieval.

To concretize this, SciQA is compared to three other benchmarks in terms of its vocabulary.
The vocabulary of a \gls{kb}, denoted by $K$,
includes \glspl{uri} and literals \cite{dialloComprehensiveEvaluationNeural2024}.
In the context of \gls{rdf} datasets: $ K = I \cup L$, see \Cref{def:rdfDataset}.
A comparison of the datasets and their vocabularies is given by \Cref{table:comparisonDatasets} and
\Cref{table:comparisonVocabularies}, respectively.

\begin{table*}[h]

	\centering

	\begin{tabular}{l|S[table-format=6.0]S[table-format=6.0]S[table-format=5.0]S[table-format=5.0]|r}

		\textbf{Dataset}     & \textbf{Total} & \textbf{Train} & \textbf{Val} & \textbf{Test} & \textbf{Source} \\
		\hline
		\textbf{LC-QuAD}     & 5000     & 4000     & 500    & 500     & \cite{dialloComprehensiveEvaluationNeural2024} \\
		\textbf{LC-QuAD 2.0} & 30225    & 21761    & 2418   & 6046    & \cite{dialloComprehensiveEvaluationNeural2024} \\
		\textbf{DBNQA}       & 382794   & 306236   & 38279  & 38279   & \cite{dialloComprehensiveEvaluationNeural2024} \\
		\textbf{SciQA}       & 2565     & 1795     & 257    & 513     & This work

	\end{tabular}

	\caption{
		Number of samples of four datasets, broken down into training, validation, and test splits,
		along with the overall totals.
	}

	\label{table:comparisonDatasets}
\end{table*}

\begin{table*}[h]

	\centering

	\begin{tabular}{l|S[table-format=6.0]S[table-format=6.0]S[table-format=5.0]S[table-format=5.0]S[table-format=4.0]|r}
		\textbf{Dataset}            & \textbf{Total} & \textbf{Train} & \textbf{Val} & \textbf{Test} & \textbf{\glsentryshort{oov}} & \textbf{Source} \\
		\hline
		\textbf{LC-QuAD}            & 4751     & 4150     & 1068   & 1065    & 318                    & \cite{dialloComprehensiveEvaluationNeural2024} \\
		\textbf{LC-QuAD 2.0}        & 34801    & 27949    & 5413   & 10993   & 7523                   & \cite{dialloComprehensiveEvaluationNeural2024} \\
		\textbf{DBNQA}              & 157672   & 142985   & 36845  & 37130   & 7998                   & \cite{dialloComprehensiveEvaluationNeural2024} \\
		\textbf{SciQA}              & 1347     & 1077     & 297    & 483     & 182                    & This work \\
		\textbf{SciQA, \glspl{uri}} & 188      & 135      & 50     & 69      & 37                     & This work

	\end{tabular}

	\caption{
		\glsentrylong{kb} vocabulary sizes.
	}

	\label{table:comparisonVocabularies}
\end{table*}

\begin{table*}[h]

	\centering

	\begin{tabular}{l|S[table-format=1.3]S[table-format=1.3]S[table-format=1.3]S[table-format=1.3]S[table-format=1.3]}

		\textbf{Dataset}            & \textbf{Total}  & \textbf{Train} & \textbf{Val}     & \textbf{Test}    & \textbf{\glsentryshort{oov}} \\
		\hline
		\textbf{LC-QuAD}            & 0.950          &  1.038        &  2.136           &  2.130            &  0.071 \\
		\textbf{LC-QuAD 2.0}        & 1.151          &  1.284        &  2.239           &  1.818          &  0.311 \\
		\textbf{DBNQA}              & 0.412          &  0.467        &  0.963           &  0.970            &  0.023 \\
		\textbf{SciQA}              & 0.525          &  0.600        &  1.156           &  0.942          &  0.089 \\
		\textbf{SciQA, \glspl{uri}} & 0.073          &  0.075        &  0.195           &  0.135          &  0.018 

	\end{tabular}

	\caption{
		\glsentrylong{kb} vocabulary size to number of dataset samples.
	}

	\label{table:comparisonRatios}
\end{table*}



Clearly, SciQA is a relatively small dataset with a correspondingly limited vocabulary.
However, when adjusting for dataset size by examining the ratio of vocabulary to dataset size
(as shown in \Cref{table:comparisonRatios}), a clearer picture emerges.
From this perspective, SciQA exhibits a relatively poor vocabulary density:
only the much larger DBNQA dataset has comparable ratios.
This characteristic of SciQA likely makes memorization more effective,
offering a partial explanation for the high performance of the $K_0$ baseline.

Secondly, while SciQA's \gls{oov} ratio is higher than two other datasets,
suggesting a more challenging test split, this observation requires further context.
The vocabularies reported in previous tables include both \glspl{uri} and literals.
In SciQA, however, literals significantly outweigh \glspl{uri} in terms of occurrence,
thus \Cref{table:comparisonVocabularies} and \Cref{table:comparisonRatios} also give the 
vocabulary size and its ratio to the number of samples when only \glspl{uri} are considered for SciQA.
The number of unique \glspl{uri} is therefore included in parentheses for clarity where applicable.

Moreover, literals frequently appear verbatim in the original questions, including those in the test set.
Specifically, literals occur in approximately 90\% of the samples.
Recalling the example from \Cref{s:gt_retriever_data}, the literal \mintinline{text}{Depth DDPPO}
appears exactly in the question, see \Cref{lst:verbatimLiteral}.

\begin{listing}[!ht]

\begin{minted}{text}
SQUALL:
Provide a list of papers that have utilized the Depth DDPPO model and include the links to their code?

\end{minted}

\mint{text}{SPARQL:}
\begin{minted}{sparql}
SELECT DISTINCT ?code WHERE {
	?model a orkgc:Model;
		rdfs:label ?model_lbl .
	FILTER(STR(?model_lbl) = 'Depth DDPPO')
	?benchmark orkgp:HAS_DATASET ?dataset .
	?cont orkgp:HAS_BENCHMARK ?benchmark .
	?cont orkgp:HAS_MODEL ?model;
		orkgp:HAS_SOURCE_CODE ?code .
}
\end{minted}

\caption{String literal present in both the question and its equivalent query.}
\label{lst:verbatimLiteral}
\end{listing}

This heavy reliance on literals further reduces the vocabulary's expressiveness
and may contribute to the observed performance patterns.

Despite these limitations,
incorporating more informative subgraphs during graph prompting
--- particularly those with higher vertex and edge recall
--- demonstrates a measurable improvement in performance.
While the (1000, 1000, 0.5) parameter configuration does not surpass the $K_0$ baseline,
it still outperforms all other tested configurations.
Consequently, this combination was selected for further evaluation of the model's generalization capabilities.

Finally, it is hypothesized that the proposed \gls{rag} paradigm would show greater benefits and that the baseline
would not remain the best-performing setup if a dataset with a richer and more diverse vocabulary were used.
However, the current findings remain inconclusive due to the limitations of the dataset.
These limitations neither confirm nor entirely refute the hypothesis, underscoring the need for further investigation.

In summary, while these results provide insights,
additional research with richer datasets and varied experimental setups is essential to validate the hypothesis
and fully realize the potential of the proposed approach, as outlined in \Cref{s:discussion}.

\section{Generalization Capabilities}
\label{s:generalizationCapabilities}

% Intro
The practical applicability of the domain expert is now assessed.

% Recap
In practice,
there are typically no ground truth datasets available for systems in \gls{mbse} comparable to the SciQA dataset.
However, a data generation strategy could be used to create a synthetic dataset instead.
Furthermore, no matter the dataset used to train the domain expert,
there will always question-query templates it has not seen during its training.
For example, a question that needs to be answered by a query with a structure totally different from the seen templates.

% Why
The aim of this experiment is to understand how well the domain expert generalizes to realistic, data-scarce,
scenarios.
Due to the scarcity of data in \gls{mbse}, it is crucial to evaluate whether the proposed domain expert performs effectively
when trained on synthetic data or when encountering unseen question-query templates.

\subsection{Hypothesis}

% Hypothesis: what
It was hypothesized that reducing data quality
--- either through the use of synthetic data or the exclusion of templates during training
--- would negatively impact performance.
While a basic synthetic data generation strategy was employed,
it was not expected to significantly degrade the domain expert.
In contrast, template exclusion was anticipated to have a pronounced impact,
as supported by prior work.

% Hypothesis: What
The hypothesis is that reducing data quality
--- either through the use of synthetic data or the exclusion of templates during training
--- would negatively impact performance.
While a basic synthetic data generation strategy was applied,
it was not anticipated to significantly degrade the domain expert’s performance.
In contrast, template exclusion was expected to have a more pronounced impact, as suggested by prior work.

% Hypothesis: why
These hypotheses rest on two key assumptions.

First, the \gls{squall} expert
--- used as the generator component of the domain expert, see \Cref{fig:domain_expert}
--- is presumed to possess sufficient knowledge of \gls{squall}.
Before training, the domain expert is generally expected to generate syntactically correct \gls{squall} in most cases,
due to the \gls{squall} expert.
The domain expert's training is primarily done to align the domain expert with the target domain
such that it can generate \gls{squall} that is also semantically correct.

Second, \glspl{llm} are known to struggle with generalizing to unseen question-query templates
\cite{reydAssessingGeneralizationCapabilities2023}.
Consequently, the proposed approach is not expected to be immune to this limitation.

In essence, training the domain expert means that it learns how to generate particular query structures from questions:
it learns the semantics of the domain.
It is unreasonable to think the domain expert would be able to generate a semantically correct query for a 
question if it has not seen that particular type of question-query structure before.

Given that the synthetic data was found to be of reasonable quality
--- when examples for all question-query templates were included in the prompt
--- its impact on performance was not expected to be substantial, see \Cref{table:synthetic_dataset_bleu}.

\subsection{Setup}

% How
Performance is assessed using \gls{bleu} scores, comparing generated \gls{squall} expressions against their references.
There are again two evaluation dimensions:

\begin{itemize}
	\item \textbf{Training Data Type}: Ground truth versus synthetic data.
	\item \textbf{Template Familiarity}:
		Performance on all templates when template is seen during training versus left out.
\end{itemize}

Combining these dimensions results in four distinct evaluation setups.

A note regarding the setup involving synthetic data where queries from template 2 were excluded from the
training data:
the prompt used to generate this synthetic data also omitted any examples from template 2.
If examples from template 2 had been included in the prompt, the \gls{pllm} could have inadvertently generated
\gls{squall} expressions conforming to the template.
This would have resulted in samples with template 2 leaking into the training data of the domain expert.

To reduce the number of experiments, the \gls{pcst} parameters for all setups are set to (1000, 1000, 0.5).
This parameter triple resulted in the best results besides the baseline, see \Cref{s:text_to_prelinked_squall}.

\subsection{Results}

% Actual results
The experimental results indicate performance degradation when reducing training data quality,
see \Cref{table:generalization_capabilities_bleu}.

\begin{table}[t]
	\centering
	\begin{tabular}{
			l|S[table-format=2.0] S[table-format=2.0]|S[table-format=2.0] S[table-format=2.0]
		}

		\textbf{Templates}  
							 & \multicolumn{2}{c|}{\textbf{Ground Truth Target Domain Data}}
							 & \multicolumn{2}{c} {\textbf{Synthetic Target Domain Data}} \\
							 & {$T$} & {$T \setminus \{2\}$} 
							 & {$T$} & {$T \setminus \{2\}$} \\
							 \hline
	$T$          & 95  & 90             & 78  & 77 \\
	\hline
	1            & 95  & 92             & 93  & 85 \\
	2            & 98  & \underline{40} & 42  & 36 \\
	3            & 98  & 91             & 85  & 85 \\
	4            & 90  & 96             & 47  & 65 \\
	5            & 99  & 99             & 89  & 93 \\
	6            & 50  & 76             & 35  & 40 \\
	7            & 94  & 95             & 81  & 64 \\
	8            & \text{N/A} & \text{N/A} & \text{N/A} & \text{N/A} \\

\end{tabular}

\caption{
	Evaluation of the domain expert's ability to generalize on the text-to-\glsentryshort{squall} task using the \glsentryshort{bleu} score,
	expressed as percentages, for different training datasets.
}
\label{table:generalization_capabilities_bleu}
\end{table}


For the ground truth case, when template 2 is unseen during training the domain expert expectedly does a much
poorer job,
underscoring the common challenge of generalizing to unseen question-query templates in text-to-query models.
The limited size of the training dataset may be insufficient for generalizing to new questions targeting
\gls{orkg} \cite{yinNeuralMachineTranslating2021}.
Despite extensive fine-tuning of the \gls{squall} expert,
%which enables adaptation to \gls{orkg} with fewer data samples compared to training from scratch,
it remains essential that the training dataset for the domain expert is representative of the target domain.

The performance of the domain expert when trained on generated data is evidently affected by the relatively lower
quality of these synthetic samples.

% If unexpected: insight why this might be
However, some observations are unexpected:

\begin{itemize}

	\item \textbf{Template 8}:  
		Only one test sample was available, which limits the ability to draw significant conclusions from its performance.

	\item \textbf{Template 6}:  
		The poor performance of template 6 can likely be attributed to the limited number of training samples following
		this template
		--- 48, which constitutes less than 3\% of the total training data.
		Given that models tend to favor the most frequent query templates in the training data
		\cite{dialloComprehensiveEvaluationNeural2024}, it is reasonable to assume that the infrequency of this template
		may result in its questions being incorrectly categorized as belonging to other, more common templates.

	\item \textbf{Template 4 for Synthetic Target Domain Data}: 
		The poor performance of template 4 remains unclear.
		Although the quality of its synthetic data is slightly below average, as shown in
		\Cref{table:synthetic_dataset_bleu}, and it scores somewhat worse than average for the ground truth case,
		it seems unlikely that these factors alone would account for such a significant decline in performance.
		It is possible that the combination of these issues confuses the domain expert in a manner similar to template 6.  
		To investigate this further, all questions were combined on a per-template basis into a single text,
		which was then transformed into a vector representation using TF-IDF.
		Subsequently, the templates were compared pairwise using cosine similarity.
		Interestingly, the most similar pair of templates was template 4 and template 5.
		Furthermore, template 5 contains slightly more questions than template 4
		--- approximately 19\% compared to 14\%.
		All these factors combined may explain why the model occasionally generates queries that follow template 5 when
		they should actually follow template 4.

	\item \textbf{Template 2 for Synthetic Target Domain Data on $T:$}
		The same analysis that was applied to template 4 did not reveal any clear reasons for template 2's poor
		performance.
		This suggests that a deeper investigation is required to identify the underlying cause.

\end{itemize}

%An important observation stated by \cite{reydAssessingGeneralizationCapabilities2023} is that it is not clear whether
%models generate queries based on the semantics of the input question,
%or merely learn how to map them to particular structures.
%The difference is profound.
%Assuming the latter is correct, 
%this would be the case for both the \gls{pllm} used to generate the synthetic data as well as the domain expert.

\section{Text-to-SPARQL}

% Intro
The penultimate experiment evaluates the text-to-\gls{sparql} component on the eponymous task.

% Recap
As illustrated in \Cref{fig:textToSparql}, this component takes text as input and outputs \gls{sparql}.
It is composed of three subcomponents:

\begin{enumerate}
	\item The domain expert, which generates an equivalent \gls{squall} expression of a given question.
	\item The translator, which transforms \gls{squall} to \gls{sparql}.
	\item The linker, which maps placeholders to \glspl{uri}.
\end{enumerate}

It is important to reiterate that the translator component produces \gls{sparql} queries that still contain placeholders.
These outputs previously referred to as \glsentrylongpl{pgp}.
Only after all placeholders are replaced with \glspl{uri} do these \glspl{pgp} become executable queries,
referred to as \glsentrylongpl{gp}.

% Why
Although evaluation based solely on the domain expert's output is sufficient for internal comparisons,
as discussed in \Cref{s:internalEvaluation}
--- for instance, it allows for assessing the performance of the domain expert under various parameter configurations,
such as in the text-to-\gls{squall} experiment detailed in \Cref{s:text_to_prelinked_squall}
--- translation and linking are vital in order to compare the performance of the proposed approach to that of existing
methods in the literature.

% What
The evaluation focuses on the text-to-\gls{sparql} component of the proposed \gls{kgqa} pipeline,
as illustrated in \Cref{fig:pocImplementation}.
Two configurations of the domain expert, identified as the best-performing in the text-to-\gls{squall} evaluation,
are included alongside the baseline (see \Cref{table:text_to_prelinked_squall}):
$K_0$, (100, 100, 0.5), and (1000, 1000, 0.5).
All models were trained on ground truth data ($\mathscr{D}_{\text{gt}}$).

\subsection{Hypothesis}

% Hypothesis: what
No significant differences with the results of the text-to-\gls{squall} experiment were anticipated.
Likely the same, decreasing, order in terms of performances:
$K_0$, (1000, 1000, 0.5), and (100, 100, 0.5).

% Hypothesis: why
The linking strategy was implemented from the literature,
where it was tested and showed competitive performance \cite{omarUniversalQuestionAnsweringPlatform2023},
hence, there was no reason to expect major differences.

\subsection{Setup}

% How
The text-to-\gls{sparql} component is evaluated using the \gls{bleu} score, execution accuracy and $F_1$ score.

\subsubsection{Addressing Discrepancies in BLEU Scoring}

An important consideration is the choice of references for calculating the \gls{bleu} score,
as illustrated in \Cref{lst:originalVersusFromSquallMappedSparql}.

This example compares the original \glsentryfull{pgp} of a SciQA question to an equivalent \gls{squall} expression.
When the \gls{squall} expression is translated back into \gls{sparql},
the resulting \gls{pgp} differs slightly from the original.
For instance, a \mintinline{sparql}{FILTER} statement may be replaced with an equivalent \mintinline{sparql}{BIND} clause.
While these differences do not alter the query results, they negatively impact the computed \gls{bleu} score.

In the provided example, the \gls{bleu} score between the original and the translated \gls{pgp} is only 73,
despite the functional equivalence of the queries.

\begin{listing}[!ht]

	\mint{text}{Original PGP:}
	\inputminted{sparql}{src/listings/structure-changed-through-mapping/original-pgp.sparql}
	\mint{text}{===}

	\mint{text}{Equivalent SQUALL:}
	\inputminted{text}{src/listings/structure-changed-through-mapping/equivalent-squall.txt}
	\mint{text}{---}

	\mint{text}{Translated PGP:}
	\inputminted{sparql}{src/listings/structure-changed-through-mapping/translated-pgp.sparql}
	\mint{text}{===}

	\mint{text}{Generated SQUALL:}
	\inputminted{text}{src/listings/structure-changed-through-mapping/generated-squall.txt}
	\mint{text}{---}

	\mint{text}{Generated PGP (i.e., mapped from generated SQUALL):}
	\inputminted{sparql}{src/listings/structure-changed-through-mapping/generated-pgp.sparql}

	\caption{
		An example question with its original \glsentrylong{pgp},
		as well as an equivalent \glsentryshort{squall} expression,
		and the \glsentrylong{gp} the \glsentryshort{squall} expression has been mapped to by the translator.
	}
	\label{lst:originalVersusFromSquallMappedSparql}
\end{listing}

\subsubsection{Mitigating the Scoring Discrepancy}

Since every output from the text-to-\gls{sparql} component is processed through the translator,
comparing it to the original \glspl{pgp} would systematically penalize the evaluation.
This issue is highlighted in \Cref{lst:originalVersusFromSquallMappedSparql},
which also provides an example of a generated \gls{squall} expression and its translated \gls{pgp}.
Depending on the reference used, the \gls{bleu} score can vary significantly
--- 54 when compared to the original \gls{pgp} versus 79 when compared to the translated \gls{pgp}.
Notably, the only discrepancy in this example is the generated \mintinline{text}{contribution} placeholder.

This issue persists even when \glspl{uri} are used instead of placeholders,
as linking does not resolve these discrepancies.
Therefore, to ensure a fair evaluation of the text-to-\gls{sparql} component, this effect must be avoided.

\subsubsection{Creating New References}

To address this issue, new reference queries are created using the following process:

\begin{enumerate}
	\item Create equivalent \gls{squall} expressions for each \gls{sparql} query.
	\item Translate the equivalent \gls{squall} expressions back to \gls{sparql} using the translator.
\end{enumerate}

This approach ensures that the reference queries have also passed through the translator,
just like the outputs of the text-to-\gls{sparql} component.
Consequently, the discrepancies described earlier are avoided.

The detailed process for creating these reference queries is discussed in \Cref{s:gt_retriever_data}.
This method provides a fair and consistent basis for evaluation.

\subsection{Results}

The results of the experiment are presented in \Cref{table:text_to_sparql}.
These results confirm the hypothesis: the baseline setup remains the best-performing configuration,
as discussed in \Cref{s:text_to_prelinked_squall},
followed closely by the setup using parameter triple (1000, 1000, 0.5).

\begin{table}[t]
	\centering
	\begin{tabular}{l|SSSS}
		\textbf{Approach} & \textbf{BLEU} & \textbf{Accuracy} & \textbf{F1} & \textbf{Failure} \\

		\hline

		Baseline ($K_0$)  & 94.34         & 86.52             & 86.52       &  6.83\% \\
		(100,  100, 0.5)  & 92.76         & 83.57             & 83.72       & 21.81\% \\
		(1000, 1000, 0.5) & 93.95         & 85.77             & 85.78       & 14.98\% \\

	\end{tabular}
	\caption{
		\glsentryshort{bleu}, execution accuracy and $F_1$ scores of different setups on the text-to-\glsentryshort{sparql} task.
		All domains experts were trained using the ground truth domain dataset.
	}
	\label{table:text_to_sparql}
\end{table}



Samples for which no \gls{sparql} query was ultimately obtained are excluded from the computation of scores.
Instead, these cases are represented by their failure percentages.
Failures arise due to several reasons:  

\begin{itemize}
	\item General \gls{llm} generation issues, such as repeating tokens excessively or producing incomplete outputs.
	\item Inability to extract \gls{squall} expressions from the generated output,
	 e.g., because the \gls{llm} does not follow the expected output format.
	\item Syntax errors in the extracted \gls{squall}, rendering it unmappable to \gls{sparql}.
	\item Partial or complete failure in linking placeholders to \glspl{uri},
	 e.g., not finding a \gls{uri} for the \mintinline{text}{contribution} placeholder.
	\item Semantic errors in the linked \gls{sparql} query, which can occur due to:
	\begin{itemize}
		\item Incorrect linking, i.e., mapping placeholders to erroneous \glspl{uri}.
		\item Incorrect semantics in the generated \gls{squall}, for example, instead of generating


			\mintinline[breaklines]{sparql}{?dataset rdf:type <Dataset>},


			the \gls{llm} might produce


			\mintinline[breaklines]{sparql}{<Dataset> rdf:type ?dataset}.
	\end{itemize}
\end{itemize}

Failures are not included in the computation of scores due to the stringent requirements of the translator,
which requires inputs to be syntactically correct, otherwise an error is elicited.
This contrasts with approaches that generate \gls{sparql} queries directly,
which may contain minor syntax errors,
but still be readily usable for evaluation.
Since the text-to-\gls{sparql} component produces no output in these cases,
alternatives were looked at for evaluation purposes,
for example, using a generic fallback query.
Ultimately, this avenue was not pursued further, because it was deemed to make the results less interpretable.

Consider the baseline, the results now indicate that
for over 90\% of the inputted questions the text-to-\gls{sparql} component can output a syntactically correct query.
Furthermore, among this successful subset of samples,
a \gls{bleu} score of 94.34 and an $F_1$ score of 86.52 are achieved.

%Extensive error analysis could provide deeper insights into these results.
%Such an analysis was considered beyond the scope of this work, as the focus lies on implementing a proof of concept to demonstrate the feasibility of model reporting.

\section{Comparative Analysis}

% Intro
The results of the proposed approach are compared against a relevant selection from the literature to contextualize its
performance.

% Recap
While previous experiments have demonstrated the effectiveness of the proposed approach,
a comparative evaluation is necessary to assess its relative performance.

% Why
Such comparisons are critical for validating the architecture's effectiveness under limited data requirements
and assessing how it performs compared to approaches operating under ideal data availability conditions.

% What
Two scenarios are considered for this analysis:

\paragraph{Realistic Scenario}

The proposed approach is compared to \gls{kgqan} \cite{omarUniversalQuestionAnsweringPlatform2023},
a \gls{kgqa} platform designed to operate without domain-specific training data.
This comparison is particularly relevant,
as it represents the most challenging version of the text-to-\gls{sparql} task.

\paragraph{Ideal Scenario}

The performance of the proposed approach is also contrasted with a recent benchmark on the SciQA dataset
\cite{lehmannLargeLanguageModels2024}.
This benchmark evaluated a range of \gls{llm} setups, including fine-tuning and prompting strategies.
This scenario reflects the easiest version of the text-to-\gls{sparql} task,
where models are specifically tailored for the domain using ground truth data.

This dual comparison highlights the strengths and limitations of the proposed approach across diverse conditions.
By contrasting it with \gls{kgqan} for the most challenging scenario
and with state-of-the-art \gls{llm} setups for the ideal case, a comprehensive evaluation is achieved.

\subsection{Hypothesis}

% Hypothesis 1: What?
First, the \gls{kgqan} platform was selected for its alignment with the study's objectives.
It is hypothesized that \gls{kgqan} would perform competitively on the benchmark,
providing a robust point of comparison.
However, \gls{kgqan} is expected to represent a lower bound on performance for this work,
as it does not utilize any domain-specific data.

% Hypothesis 1: Why?
This expectation is grounded in \gls{kgqan}'s proven adaptability across a wide range of domains,
including general-purpose domains and more specialized ones like DBLP \cite{omarUniversalQuestionAnsweringPlatform2023}.

% Hypothesis 2: What?
Second, the SciQA benchmark results \cite{lehmannLargeLanguageModels2024} provide a complementary perspective.
The best results were achieved by fine-tuned \glspl{llm},
which the proposed approach is hypothesized not to surpass.
These benchmarks are expected to establish an upper bound for performance on SciQA,
as they only use domain-specific data.

% Hypothesis 2: Why?
This hypothesis is grounded in the expectation that specialized training confers a significant advantage for the
text-to-\gls{sparql} task, particularly on a dataset like SciQA,
which is relatively homogeneous, even across splits (see \Cref{s:text_to_prelinked_squall}).
Additionally, it is possible that \glspl{llm} do not truly learn sentence semantics
but instead rely on mapping questions to learned query structures \cite{reydAssessingGeneralizationCapabilities2023}. 
Building on this premise, training an \gls{llm} on question-query structures it will not encounter during evaluation
could result in greater confusion and, consequently, degraded performance.

\subsection{Setup}

First the approaches are summarized and compared to this work, afterwards the setups are detailed.

\subsubsection{KGQAn}

% Recap approach
\gls{kgqan} \cite{omarUniversalQuestionAnsweringPlatform2023} was first introduced in \Cref{s:QASPDA},
because it seemed an interesting work to compare to
due to its similar goals as a framework to assist non-technical stakeholders access \gls{kg} information.
It aims to be universal, and capable of forming queries for any \gls{kg},
without \gls{kg} specific training data nor prior information.

It attempts to achieve this by splitting up the text-to-\gls{sparql} task in two steps:
a question understanding step and a linking step.
%
The first step is aimed at sentence semantics:
given an input question a list of semantic triples is generated 
--- e.g., \mintinline[breaklines]{text}{(model reporting, is, flexible)}
--- using a trained \gls{llm}.
The model was previously trained on a \gls{qa} dataset similar SciQA,
however, one made domain agnostic by replacing \glspl{uri} with placeholders.
%
The second step then attempts to construct a query from the generated list using heuristics,
for example, building an \mintinline{sparql}{ASK} query if the question starts with \mintinline{text}{is}.
Subsequently, the placeholders are mapped to \glspl{uri} from the target \gls{kg},
using a dynamic \glsentryfull{jit} linker detailed in the linking section (see \Cref{s:linking}).
Essentially, it queries the \gls{kg} using string matching based on the placeholders.

% Main Difference
A key distinction between \gls{kgqan} and the proposed approach lies in the method of query structure generation.
Unlike this work, which leverages an \gls{llm} to generate query structures
--- i.e., the domain expert
--- \gls{kgqan} employs heuristics to construct them from a list of semantic triples.
So, while both approaches utilize an intermediary representation, they are distinctly different,
namely \gls{squall} expressions in this work versus semantic triples in \gls{kgqan})
Furthermore, the assumptions about data availability differ.
\gls{kgqan} operates under stricter data constraints, as it does not utilize any domain-specific data,
in contrast to the data-scarce setting assumed for this study.

% Setup
Still, \gls{kgqan} remains a viable benchmark due to the significant overlap in the intended audience
and the relatively similar data constraints when compared to other methods in the literature.

The parameters for the linker in both the proposed architecture and \gls{kgqan} were configured as follows:

\begin{itemize}
	\item The maximum number of vertices and edges returned from a probing query was set to 50 and \num{1000}, respectively.
	\item After scoring, only the top 10 vertices or edges were retained.
	\item Only the most likely final answer was selected.
\end{itemize}

\subsubsection{Benchmarks}

% Recap of Approach
Recent work \cite{lehmannLargeLanguageModels2024} has established the standard for the text-to-\gls{sparql} task on the
SciQA dataset using a diverse array of \glspl{llm} (e.g., T5, GPT-3.5-turbo)
and strategies such as \gls{zsl}, \gls{fsl}, and fine-tuning.

% Key Differences
The primary aim of \cite{lehmannLargeLanguageModels2024} was to demonstrate how \gls{llm} fine-tuning and
\gls{fsl} approaches can achieve strong performance on the challenging SciQA benchmark.
However, this work was not conducted in the context of data scarcity or with a focus on generalizability.
The methodology employed a straightforward supervised training approach for fine-tuning and relied on prompt
optimization techniques for \gls{fsl}. 

In contrast, the approach in this study addresses data scarcity through techniques such as incorporating non-parametric
memory within the domain expert.
Additionally, generalizability is emphasized by fine-tuning the \gls{squall} expert specifically for the
text-to-\gls{sparql} task and utilizing it as a generator for the domain expert.

% Experimental Setup
Various prompt engineering strategies were explored, with the most effective being a similarity-based sampling approach.
This method involved selecting the seven most semantically similar questions as examples
and including them in the prompt alongside their corresponding \gls{sparql} queries.

\subsection{Results}

Finally, the results indicate that the proposed approach is competitive especially considering it is built for 
data-scarce conditions.

\subsubsection{KGQAn}
\label{s:kgqan}

% Actual results

Contrary to expectations, the platform significantly underperformed.
The results, presented here for completeness, indicate that 54\% of the test questions failed at generation,
i.e., \gls{qu} did not produce a valid list of triples.
For the remaining queries, the \gls{bleu} score was 4.42, with both accuracy and $F_1$ effectively zero.

% If unexpected: insight why this might be

To investigate these unexpected outcomes, the platform's source code and results were examined further.
When \gls{qu} succeeded in generating queries, the outputs were overly generic.
While some correct answers were occasionally retrieved during query execution,
they were vastly outnumbered by incorrect results.
Two critical features were identified as the root causes, severely limiting the platform's generalizability:

\begin{itemize}

	\item \textbf{\glsentrylong{qu}:}

		Generating triples instead of complete queries using an \gls{llm} does not circumvent the template problem
		\cite{reydAssessingGeneralizationCapabilities2023}.
		The challenge of producing the correct structure remains significant:
		outputting a list of triples rather than a complete query does not fundamentally resolve this issue.

	\item \textbf{Query Building:}

		It became clear that its query building approach based on heuristics does not generalize well to questions
		expected to be answered through \glspl{cgp} such as those from SciQA,
		a challenging benchmark \cite{lehmannLargeLanguageModels2024}.

\end{itemize}

\subsubsection{Benchmark}  
\label{s:benchmark}  

The main results are summarized in \Cref{table:comparativeAnalysis}.  
For each model, the strategy that yielded the best performance was selected.  

\begin{table}[t]
	\centering
	\begin{tabular}{l|S[table-format=2.2]}

		\textbf{Setup}                                      & \textbf{BLEU} \\
		\hline                                                                             
		T5-base, fine-tuned                                 & 96.31         \\
		GPT2-large, fine-tuned                              & 95.04         \\
		Dolly-v2-3b, similarity-based \glsentryshort{fsl}   & 80.15         \\
		GPT-3.5-turbo, similarity-based \glsentryshort{fsl} & 95.71         \\
                                                           
	\end{tabular}                                           
	\caption{
		\glsentryshort{bleu} scores set on the SciQA benchmark by different \glsentryshortpl{llm} \cite{lehmannLargeLanguageModels2024}.
	}
	\label{table:comparativeAnalysis}
\end{table}

  

\paragraph{Observations on Model Performance}  

The fine-tuned T5 model and the \gls{fsl} GPT-3.5-turbo configurations,
in particular, showcase outstanding performance, setting a high standard for comparison.
However, the approach proposed in this work demonstrates results that are only marginally lower than the benchmarks,
and are thus considered competitive, see \Cref{table:text_to_sparql}.

\paragraph{Critical Evaluation}  

However, an essential critique arises from \cite{reydAssessingGeneralizationCapabilities2023}:  
\begin{quote}  
    The handling of unknown question-query structures is particularly important because it is unclear if models really
	 generate queries based on sentence semantics, or if they simply map them to known query structures.
\end{quote}  
With enough data, it seems plausible that any sufficiently powerful \gls{llm} could generate correct queries for a
given question, provided it has encountered/encounters the corresponding question-query template during training/in its prompt. 

\paragraph{Additional Insights}  

The analysis in \cite{lehmannLargeLanguageModels2024} provides several key points:  
\begin{enumerate}  
    \item \textbf{Entity Hallucination}:  
		 Entity hallucination, likely influenced by pre-training data,
		 occurs when models hallucinate entities from external knowledge sources like Wikidata.  
		 This aligns with earlier identified concerns in this work (see \Cref{s:why_placeholders}).  
    
    \item \textbf{Syntactical Errors}:  
		 The fine-tuned T5 model exhibits a significantly higher rate of syntactical errors (40.0\%) compared to
		 \gls{fsl} GPT-3.5-turbo (5.2\%), indicating weaker knowledge of \gls{sparql}.  
		 Likely causes include T5's smaller size and differences in pre-training data and tasks.  
    
    \item \textbf{Semantic Errors}:  
		 Frequent errors include misspelled entities and incorrect predicates.  
\end{enumerate}  

\paragraph{Mitigation in the Proposed Approach}  

The proposed approach preemptively attempted to address these issues:  
\begin{enumerate}  
    \item By using placeholders and subsequently linking.  
    \item By employing the \gls{squall} expert as the generator for the domain expert,
		 and freezing it during the latter's training phase.  
    \item By leveraging \gls{rag}.  
\end{enumerate}  

\paragraph{Template Generalization and Prompting Strategies}  

Finally, one of the tested prompting strategies involved randomly sampling examples without regard to query templates.  
This universally resulted in poorer performance,
highlighting a recurring challenge \cite{dialloComprehensiveEvaluationNeural2024}:  
generalizing to unseen templates.
The efficacy of \gls{fsl} heavily depends on well-chosen examples.  
It appears unlikely that \gls{fsl} approaches would succeed with entirely unseen question-query structures 
\cite{reydAssessingGeneralizationCapabilities2023}.  

% Note: Final discussion not result discussion.
\section{Discussion}
\label{s:discussion}

This section provides a comprehensive analysis of the experimental results,
offering insights into their broader implications.
It connects the findings to existing literature, evaluates their alignment with the stated requirements,
and demonstrates how model reporting is implemented.

Throughout this work, numerous decisions were made to address data scarcity,
aiming to design an architecture suitable for \gls{mbse}.
The experiments were structured to explore the potential of the proposed approach in achieving effective model reporting.
Key design choices included
restating the text-to-\gls{sparql} task as text-to-\gls{squall} task,
fine-tuning a robust \gls{llm} to develop a general understanding of \gls{squall},
integrating this fine-tuned \gls{llm} into a trainable \gls{rag} architecture,
and investigating the model's adaptability to new domains using corresponding (synthetic) data.
These choices all had the common goal of realizing robust \glsentrylong{da} from a data rich source domain
to a specialized (\gls{mbse}) domain.

\subsection{Challenges in Generalization to New Questions}

Previous work \cite{reydAssessingGeneralizationCapabilities2023} shows that
\gls{nmt} approaches typically struggle with unseen question-query structures.
During training, \glspl{llm} are often exposed to questions that follow specific structures,
a pattern evident in the datasets that are widely used, such as LC-QuAD 2.0, and present in SciQA as well.
Consequently, the corresponding queries adhere to certain templates,
making it difficult for \glspl{llm} to generate correct queries for questions requiring novel query structures.

While the proposed approach is not without limitations, as discussed in \Cref{s:generalizationCapabilities},
the experiments suggest significant potential for further development.
\Cref{lst:domainAdaptation} illustrates the notable differences between queries in the original and target domains.
Enhancing the diversity of training data during the fine-tuning of the \gls{squall} expert is both feasible
and primarily a matter of data engineering.
Such improvements could bolster the domain expert's ability to generalize to question-query structures not encountered
during the second training phase.
Additionally, employing more advanced synthetic data generation techniques could further expand the variety and
representativeness of the training dataset, enhancing the model's robustness and adaptability.

\subsection{Adaptation to New Knowledge Graphs and Domains}

The \gls{squall} expert was fine-tuned on LC-QuAD 2.0,
then frozen and integrated into a \gls{rag} architecture (see \Cref{s:sysArch}),
resulting in the domain expert.
That model was subsequently trained on SciQA, a dataset tailored for \gls{orkg} rather than Wikidata.
This training strategy (see \Cref{s:training_finetuning}) on two distinct domains provides valuable insight into the
domain adaptation, or more broadly, generalization capabilities of the proposed implementation.
As stated above, despite the significant differences between the \gls{sparql} templates used in LC-QuAD 2.0 and SciQA
the domain expert exhibited the ability to adapt to the new \gls{kg}/domain
using relatively little synthetic data of a lower than ground truth quality.

These results further corroborate the notion that \gls{squall} can effectively reduce data requirements.
Catastrophic forgetting the \gls{squall} expert is likely avoided of by keeping it frozen.
The use of placeholders enables reusing the \gls{squall} expert for a new domain, it makes possible this domain adaptation approach  .
Although deep error analysis is necessary to get good concrete empirical evidence.

\subsection{Adaptation to New Knowledge Graphs and Domains}

The \gls{squall} expert was fine-tuned on LC-QuAD 2.0, subsequently frozen,
and integrated into a \gls{rag} architecture (see \Cref{s:sysArch}), resulting in the domain expert.
This model was then trained on SciQA, a dataset specifically designed for \gls{orkg} rather than Wikidata.
This dual-domain training strategy (see \Cref{s:training_finetuning})
offers valuable insights into the domain adaptation and generalization capabilities of the proposed implementation.

Despite the significant differences between the \gls{sparql} templates used in LC-QuAD 2.0 and SciQA,
the domain expert demonstrated the ability to adapt effectively to the new \gls{kg}/domain using relatively limited
synthetic data, even when the data was of lower quality than ground truth.
These results support the potential of \gls{squall} to reduce data requirements effectively.

Freezing the \gls{squall} expert likely mitigates catastrophic forgetting,
ensuring that the gained \gls{squall} knowledge is retained.
Moreover, the use of placeholders facilitates the reuse of the \gls{squall} expert across domains,
enabling the proposed approach to domain adaptation.
While these findings are promising,
a detailed error analysis is necessary to provide robust empirical evidence and uncover potential areas for refinement.

\subsection{Ambiguity Related to RAG}

The effectiveness of using \gls{rag} in the proposed framework remains inconclusive.
While some improvement was observed with higher recall of constructed subgraphs, the baseline approach
--- where no subgraph is provided during generation
--- still outperformed the other setups (see \Cref{table:text_to_prelinked_squall}).
This outcome is likely due to the homogeneous nature of the SciQA dataset (see \Cref{s:text_to_prelinked_squall}),
which may limit the potential advantages of \gls{rag}.
In such cases, the baseline approach likely leads to memorization
--- due to the retrieved data being highly similar across samples,
--- outperforming actual learning to utilize retrieved data, which the other setups do.

Further investigation is required to confirm or rule out the hypothesized benefits of \gls{rag}.
Future work could explore its performance across more diverse datasets to provide a definitive assessment of its
utility within this context.

\subsection{Feasibility of Using Synthetic Data}

It is noteworthy that the majority of SciQA's samples are synthetically generated \cite{auerSciQAScientificQuestion2023}.
Consequently, the experimental results
--- including those using ground truth data 
--- support the idea that data scarcity can be effectively addressed through the use of synthetic data.
Employing advanced techniques to collect and generate question-query templates tailored to specific domains could
be critical to model reporting.

\subsection{Potential for Model Reporting and Data Transformation}

SciQA queries are notably complex \cite{lehmannLargeLanguageModels2024},
often surpassing those in datasets like LC-QuAD in several aspects.
They frequently involve a high number of triples and true \glspl{cgp},
which are combinations of \glspl{bgp} extended with relational operations \cite{anglesFoundationsModernQuery2017},
such as filters and unions.
While LC-QuAD queries include features like filters, limits, and basic ordering,
they lack the advanced constructs found in SciQA, such as aggregation, grouping, optionals, and nested queries.

Despite this complexity, the domain expert has demonstrated the ability to effectively handle \glspl{cgp}.
Current transformations present in the \gls{sparql} queries include solution modifiers and grouping
(see \Cref{lst:sciqa_4}).
Importantly, \gls{squall}'s comprehensive coverage of the \gls{sparql} 1.1 standard,
including graph updates \cite{ferreSQUALLExpressivenessSPARQL2014},
offers significant potential for extending these capabilities.
This expansion could support and automate more phases of the model reporting pipeline,
including advanced transformations.

Achieving this advancement would require high-quality data for fine-tuning,
along with significantly more diverse (synthetic) training data.
Such improvements represent a critical step toward enabling more robust transformations within the model reporting
pipeline.
An overview of the complex patterns achievable by the current approach is provided in \Cref{appendix:sciqa_sparqls},
showcasing examples of optional clauses, filters, solution modifiers (e.g., max, limit, order by)
aggregation, and grouping.


\chapter*{Conclusion}
\chaptermark{Conclusion}
\addcontentsline{toc}{chapter}{Conclusion}  

\section{Introduction}

% Goal and research problem
The primary goal of this work was to present and realize a more flexible approach to reporting in \gls{mbse}.
To this end, a novel model reporting paradigm was conceptualized.
This approach was realized through the development of a proof-of-concept implementation, Mo-Lab,
which leverages an innovative architecture and training techniques, addressing the principal challenge,
namely data scarcity, that hinders the use of neural models for specialized domains such as those found in \gls{mbse}.
The insights from this work could guide future efforts to integrate contemporary \gls{nlp} techniques into \gls{mbse}
workflows.

\section{Research Questions and Answers}

The research questions outlined in \Cref{s:introduction} are reiterated and addressed below.

\textbf{RQ1:}
How does the performance of the proposed domain expert compare to other approaches
across various data availability scenarios?

In data-rich scenarios, the domain expert, in its baseline setup,
performs comparably to state-of-the-art models specifically trained on the benchmark,
with only slight performance differences.
In data-poor scenarios, the domain expert significantly outperforms \gls{kgqan}, a comparable approach.
Although direct results for text-to-\gls{sparql} using synthetic data are not available,
it can be inferred that the performance would surpass \gls{kgqan},
given the minimal performance degradation observed in the text-to-\gls{squall} task,
--- when going from ground truth to synthetic data,
--- and the linking step,
--- when associating placeholders with \glspl{uri}.

\textbf{RQ2:}
Does the domain expert effectively utilize the retrieved information with which it is soft-prompted,
and what factors influence the effectiveness of these prompts?

Due to benchmark limitations, it is inconclusive whether the domain expert fully leverages the soft prompts.
However, assuming their efficacy,
more relevant constructed graphs with higher vertex and edge recall are likely to enhance prompt effectiveness,
and improve generation.

\textbf{RQ3:}
What is the impact of using synthetic data, as opposed to ground truth data,
on domain adaptation performance during the second learning phase?

Synthetic data negatively impacts performance due to its lower quality and lack of representativeness.
For instance, performance is poor on templates not included in the synthetic data.
This challenge is consistent with similar approaches,
as performance on unseen templates remains an open problem in the field.
However, it should be noted that the ground truth data, albeit of a much higher quality than the synthetic data,
was essentially also generated, indicating the importance of the quality of the generated data used during the second
learning phase.

\section{Synthesis and Significance}

The central motivation of this research stems from the desire to make reporting in \gls{mbse} more flexible
than is currently possible due to the challenge of data scarcity,
which poses a significant barrier to the adoption of contemporary \gls{nlp} techniques in \gls{mbse}.
Addressing this challenge required overcoming the reliance on extensive,
high-quality datasets or the presupposition of domain familiarity by \glspl{pllm}.
This work contributes to the field by:

\begin{enumerate}
	\item Introducing a novel paradigm for model reporting in \gls{mbse}.
	\item Developing an implementation that addresses data scarcity through innovative architectural and training
		solutions.
	\item Presenting the domain expert, a \gls{rag} model whose architecture and training procedures could inspire future
		research in applying \gls{nlp} techniques to \gls{mbse}.
\end{enumerate}

These contributions highlight the potential of \gls{nlp} advancements in empowering practitioners in \gls{mbse},
demonstrating practical applications of cutting-edge technology.

\section{Limitations}

Despite its contributions, this work evidently has certain limitations:

\begin{itemize}
	\item The agent is scoped to a \glsentryfull{qa} pipeline, limiting its generality.
	\item Results are restricted to illustrative post-processing,
		such as natural language explanations, tabulations, and visualizations via an external engine.
	\item Poor performance on unseen question-query templates, a challenge shared by similar approaches,
		limits its practical applicability.
	\item No conclusive evidence was gathered on the efficacy of the \gls{rag} mechanism.
\end{itemize}

\Cref{c:futureWork} outlines how these limitations can be addressed in future research,
offering multiple interesting avenues.

\section{Closing Thoughts}

This work aspires to inspire practitioners in \gls{mbse} by showcasing the potential of recent advancements in the
\gls{nlp} field.
It is hoped that the insights, methodologies, and findings presented here will serve as a foundation for further
exploration, contributing to the evolution of reporting in \gls{mbse} towards more flexibility and interactivity 
through \gls{ai}.


\chapter*{Future Work}
\chaptermark{Future Work}
\addcontentsline{toc}{chapter}{Future Work}  
\label{c:futureWork}

This chapter outlines potential future research directions inspired by this work.
These avenues are categorized into five key areas:
\gls{ai} agent, \glsentrylong{da}, \glsentrylong{rag}, synthetic data generation, and linking.

\section{AI Agent}

Mo-Lab, as a proof-of-concept implementation of the proposed model reporting paradigm,
is currently limited in its capabilities.
Future research could explore incorporating the domain expert into a fully-fledged \gls{ai} agent,
enabling interactive features such as dynamic feedback through a dialog between the stakeholder and the agent.
This enhancement could facilitate iterative refinement of results, improving both usability and accuracy.
Additionally, such an agent could respond to stakeholder requests in terms of analysis preferences,
visualization formats, and other personalized requirements.

The proposed text-to-\gls{sparql} pipeline, which includes the domain expert,
could serve as a specialized subcomponent of the agent, focusing on text-to-\gls{sparql} tasks.
Meanwhile, the general capabilities of a \gls{pllm}, such as ChatGPT, could be retained to handle broader interactions,
ensuring a versatile and comprehensive \gls{ai} agent.

\section{Domain Adaptation}

A promising direction for future research is enriching the source domain data used to train the \gls{squall} expert.
Since \gls{da} can be applied to a mixed source domain \cite{mansourDomainAdaptationMultiple2008},
incorporating multiple \gls{qa} datasets (e.g., LC-QuAD 2.0, DBLP-QuAD, QALD-9-Plus, etc.)
could enhance generalizability, improve practical applicability,
and reduce training data requirements for the domain expert in the target domain.

\section{Retrieval Augmented Generation}

Evaluating the proposed approach on a diverse set of more varies and less homogeneous benchmarks is necessary,
as the current experimentation could not yield conclusive results.
For instance, replacing DBLP and DBLP-QuAD with \gls{orkg} and SciQA could provide valuable insights.
While DBLP was initially considered but abandoned due to its size,
revisiting it now can be justified given the promising results achieved.

\section{Synthetic Data Generation}

Further exploration of synthetic data generation techniques presents another exciting research avenue.
Investigating methods to produce data that is more faithful, extensive,
and varied across the domain could significantly enhance domain expert training.
Additionally, this work focused on generating queries while assuming the availability of questions.
Future research could explore techniques for simultaneously generating both queries and questions.
Developing a framework for directly generating data from an ontology or \gls{kg} of a domain could also be valuable,
particularly if it is practically applicable across real-world \gls{mbse} projects.

\section{Linking}

The linking component utilized in this work is a component separate from the remainder of the domain expert, 
and it is non-neural.
Exploring more advanced linkers that could leverage the constructed subgraph output during post-retrieval,
might offer significant improvements.
Neural approaches applicable in data-scarce scenarios, potentially through prompt engineering or \gls{da},
represent another promising direction for future work.


\renewcommand\bibname{References}
%\bibliography{references-biblatex.bib}
\printbibliography[heading=bibintoc]

%%%%%%%%%%%%%%%%%%%%%%%%%%%%%%
%    Dutch version           %
%%%%%%%%%%%%%%%%%%%%%%%%%%%%%%
% \renewcommand\bibname{Referenties}
% \bibliography{references.bib}

\begin{appendices}

\section*{Kepler16b Report}
\addcontentsline{toc}{section}{Kepler16b Report}
\label{appendix:kepler}

This appendix presents an example of a report made using openCAESAR.\footnote{
	Kepler16 demo project available at \url{https://github.com/opencaesar/kepler16b-example}.
}\footnote{
	Kepler16b demo report available at \url{https://www.opencaesar.io/kepler16b-example}.
}

\begin{figure}[H]
	\includegraphics[page=1,width=0.75\textwidth]{images/kepler-report.pdf}
	\caption{Kepler16b report example, page 1.}
	\label{fig:keplerReport1}
\end{figure}

\begin{figure}[H]
	\includegraphics[page=2,width=0.75\textwidth]{images/kepler-report.pdf}
	\caption{Kepler16b report example, page 2.}
	\label{fig:keplerReport2}
\end{figure}

\begin{figure}[H]
	\includegraphics[page=3,width=0.75\textwidth]{images/kepler-report.pdf}
	\caption{Kepler16b report example, page 3.}
	\label{fig:keplerReport3}
\end{figure}

\newpage

\section*{SciQA Templates}
\addcontentsline{toc}{section}{SciQA Templates}
\label{appendix:sciqa_sparqls}

This appendix presents an example query (more specifically, a \gls{pgp}) for each template available in the SciQA dataset.
The corresponding \gls{squall} expressions, which are (nearly) equivalent, are also provided,
along with details on the method of acquisition.

In some cases, a second projection variable
--- representing the \mintinline{sparql}{rdfs:label} of the first
--- was omitted to simplify the manual construction of the queries.
Prior to handcrafting the \gls{squall} queries,
the \gls{sparql} queries were further simplified while preserving their equivalence to the originals.
For example, nested queries were flattened, see \Cref{lst:sciqa_5}.

\begin{listing}[H]
	\mint{text}{SPARQL:}
	\inputminted{sparql}{src/listings/sciqa-reference-queries/t1.sparql}
	\mint{text}{---}
	\mint{text}{SQUALL:}
	\inputminted{text}{src/listings/sciqa-reference-queries/t1.txt}
	\mint{text}{---}
	\mint{text}{# Handcrafted SQUALL: 'optional' not supported by SPARQL2SQUALL.}
	\caption{SciQA example for template 1.}
	\label{lst:sciqa_1}
\end{listing}

\begin{listing}[H]
	\mint{text}{SPARQL:}
	\inputminted{sparql}{src/listings/sciqa-reference-queries/t2.sparql}
	\mint{text}{---}
	\mint{text}{SQUALL:}
	\inputminted{text}{src/listings/sciqa-reference-queries/t2.txt}
	\mint{text}{---}
	\mint{text}{/* Mapped SPARQL to SQUALL using SPARQL2SQUALL. */}
	\caption{SciQA example for template 2.}
	\label{lst:sciqa_2}
\end{listing}

\begin{listing}[H]
	\mint{text}{SPARQL:}
	\inputminted{sparql}{src/listings/sciqa-reference-queries/t3.sparql}
	\mint{text}{---}
	\mint{text}{SQUALL:}
	\inputminted{text}{src/listings/sciqa-reference-queries/t3.txt}
	\mint{text}{---}
	\mint{text}{# Handcrafted SQUALL: 'optional' not supported by SPARQL2SQUALL.}
	\caption{SciQA example for template 3.}
	\label{lst:sciqa_3}
\end{listing}

\begin{listing}[H]
	\mint{text}{SPARQL:}
	\inputminted{sparql}{src/listings/sciqa-reference-queries/t4.sparql}
	\mint{text}{---}
	\mint{text}{SQUALL:}
	\inputminted{text}{src/listings/sciqa-reference-queries/t4.txt}
	\mint{text}{---}
	\mint{text}{# Handcrafted SQUALL: Verbalization of aggregates not supported by SPARQL2SQUALL.}
	\caption{SciQA example for template 4.}
	\label{lst:sciqa_4}
\end{listing}

\begin{listing}[H]
	\mint{text}{SPARQL:}
	\inputminted{sparql}{src/listings/sciqa-reference-queries/t5.sparql}
	\mint{text}{---}
	\mint{text}{SQUALL:}
	\inputminted{text}{src/listings/sciqa-reference-queries/t5.txt}
	\mint{text}{---}
	\mint{text}{# Handcrafted SQUALL: Verbalization of sub-queries not supported by SPARQL2SQUALL.}
	\caption{SciQA example for template 5.}
	\label{lst:sciqa_5}
\end{listing}

\begin{listing}[H]
	\mint{text}{SPARQL:}
	\inputminted{sparql}{src/listings/sciqa-reference-queries/t6.sparql}
	\mint{text}{---}
	\mint{text}{SQUALL:}
	\inputminted{text}{src/listings/sciqa-reference-queries/t6.txt}
	\mint{text}{---}
	\mint{text}{/* Mapped SPARQL to SQUALL using SPARQL2SQUALL. */}
	\caption{SciQA example for template 6.}
	\label{lst:sciqa_6}
\end{listing}

\begin{listing}[H]
	\mint{text}{SPARQL:}
	\inputminted{sparql}{src/listings/sciqa-reference-queries/t7.sparql}
	\mint{text}{---}
	\mint{text}{SQUALL:}
	\inputminted{text}{src/listings/sciqa-reference-queries/t7.txt}
	\mint{text}{---}
	\mint{text}{# Handcrafted SQUALL: Verbalization cannot be made non-ambiguous without brackets by SPARQL2SQUALL.}
	\caption{SciQA example for template 7.}
	\label{lst:sciqa_7}
\end{listing}

\begin{listing}[H]
	\mint{text}{SPARQL:}
	\inputminted{sparql}{src/listings/sciqa-reference-queries/t8.sparql}
	\mint{text}{---}
	\mint{text}{SQUALL:}
	\inputminted{text}{src/listings/sciqa-reference-queries/t8.txt}
	\mint{text}{---}
	\mint{text}{/* Mapped SPARQL to SQUALL using SPARQL2SQUALL. */}
	\caption{SciQA example for template 8.}
	\label{lst:sciqa_8}
\end{listing}

\newpage

\section*{Responsible Use of Generative Artificial Intelligence}
\addcontentsline{toc}{section}{Responsible Use of Generative Artificial Intelligence}  
\label{appendix:genAI}

The use of generative \gls{ai} in this thesis has been carefully restricted to two specific use cases:
serving as a writing assistant and a productivity enhancer.
In particular, ChatGPT was utilized to refine the flow of pre-written passages,
summarize chapters for use in introductions and conclusions,
and expedite the creation of TikZ schematics and graphs.
At no point was \gls{ai} used to generate content from scratch.

\newpage

\section*{Societal Reflection}
\addcontentsline{toc}{section}{Societal Reflection}  

Upon reflection,
it became clear that certain Sustainable Development Goals were inherently aligned with the research process,
even though they were not consciously adopted as guiding principles from the outset.

Great care was taken to ensure responsible production throughout the methodology and experimentation phases.
This was demonstrated by the judicious use of computational resources,
highlighted by the selection of a comparatively smaller domain-specific model, i.e., the domain expert,
which consumed significantly less computational power than, for example, a ChatGPT-based prompting approach.

Transparency and accessibility have also been central to this work.
Every step of the research has been explicitly documented and explained to ensure that the findings are accessible and
reproducible, fostering inclusivity and openness in scientific research.

Finally, this thesis originated at Polytechnique Montréal and was further developed upon my return to Ghent University.
I hope that this work not only contributes to advancing academic knowledge
but also fosters stronger collaborative ties between these two institutions.
I will enthusiastically encourage and recommend similar exchange opportunities.

\newpage

\end{appendices}


\pagestyle{numberless} 
\pagestyle{empty}
%\include{chapters/12-appendices.tex}

\end{document}

