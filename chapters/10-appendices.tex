\begin{appendices}

\section*{Kepler16b Report}
\addcontentsline{toc}{section}{Kepler16b Report}
\label{appendix:kepler}

This appendix presents an example of a report made using openCAESAR.\footnote{
	Kepler16 demo project available at \url{https://github.com/opencaesar/kepler16b-example}.
}\footnote{
	Kepler16b demo report available at \url{https://www.opencaesar.io/kepler16b-example}.
}

\begin{figure}[H]
	\includegraphics[page=1,width=0.75\textwidth]{images/kepler-report.pdf}
	\caption{Kepler16b report example, page 1.}
	\label{fig:keplerReport1}
\end{figure}

\begin{figure}[H]
	\includegraphics[page=2,width=0.75\textwidth]{images/kepler-report.pdf}
	\caption{Kepler16b report example, page 2.}
	\label{fig:keplerReport2}
\end{figure}

\begin{figure}[H]
	\includegraphics[page=3,width=0.75\textwidth]{images/kepler-report.pdf}
	\caption{Kepler16b report example, page 3.}
	\label{fig:keplerReport3}
\end{figure}

\newpage

\section*{SciQA Templates}
\addcontentsline{toc}{section}{SciQA Templates}
\label{appendix:sciqa_sparqls}

This appendix presents an example query (more specifically, a \gls{pgp}) for each template available in the SciQA dataset.
The corresponding \gls{squall} expressions, which are (nearly) equivalent, are also provided,
along with details on the method of acquisition.

In some cases, a second projection variable
--- representing the \mintinline{sparql}{rdfs:label} of the first
--- was omitted to simplify the manual construction of the queries.
Prior to handcrafting the \gls{squall} queries,
the \gls{sparql} queries were further simplified while preserving their equivalence to the originals.
For example, nested queries were flattened, see \Cref{lst:sciqa_5}.

\begin{listing}[H]
	\mint{text}{SPARQL:}
	\inputminted{sparql}{src/listings/sciqa-reference-queries/t1.sparql}
	\mint{text}{---}
	\mint{text}{SQUALL:}
	\inputminted{text}{src/listings/sciqa-reference-queries/t1.txt}
	\mint{text}{---}
	\mint{text}{# Handcrafted SQUALL: 'optional' not supported by SPARQL2SQUALL.}
	\caption{SciQA example for template 1.}
	\label{lst:sciqa_1}
\end{listing}

\begin{listing}[H]
	\mint{text}{SPARQL:}
	\inputminted{sparql}{src/listings/sciqa-reference-queries/t2.sparql}
	\mint{text}{---}
	\mint{text}{SQUALL:}
	\inputminted{text}{src/listings/sciqa-reference-queries/t2.txt}
	\mint{text}{---}
	\mint{text}{/* Mapped SPARQL to SQUALL using SPARQL2SQUALL. */}
	\caption{SciQA example for template 2.}
	\label{lst:sciqa_2}
\end{listing}

\begin{listing}[H]
	\mint{text}{SPARQL:}
	\inputminted{sparql}{src/listings/sciqa-reference-queries/t3.sparql}
	\mint{text}{---}
	\mint{text}{SQUALL:}
	\inputminted{text}{src/listings/sciqa-reference-queries/t3.txt}
	\mint{text}{---}
	\mint{text}{# Handcrafted SQUALL: 'optional' not supported by SPARQL2SQUALL.}
	\caption{SciQA example for template 3.}
	\label{lst:sciqa_3}
\end{listing}

\begin{listing}[H]
	\mint{text}{SPARQL:}
	\inputminted{sparql}{src/listings/sciqa-reference-queries/t4.sparql}
	\mint{text}{---}
	\mint{text}{SQUALL:}
	\inputminted{text}{src/listings/sciqa-reference-queries/t4.txt}
	\mint{text}{---}
	\mint{text}{# Handcrafted SQUALL: Verbalization of aggregates not supported by SPARQL2SQUALL.}
	\caption{SciQA example for template 4.}
	\label{lst:sciqa_4}
\end{listing}

\begin{listing}[H]
	\mint{text}{SPARQL:}
	\inputminted{sparql}{src/listings/sciqa-reference-queries/t5.sparql}
	\mint{text}{---}
	\mint{text}{SQUALL:}
	\inputminted{text}{src/listings/sciqa-reference-queries/t5.txt}
	\mint{text}{---}
	\mint{text}{# Handcrafted SQUALL: Verbalization of sub-queries not supported by SPARQL2SQUALL.}
	\caption{SciQA example for template 5.}
	\label{lst:sciqa_5}
\end{listing}

\begin{listing}[H]
	\mint{text}{SPARQL:}
	\inputminted{sparql}{src/listings/sciqa-reference-queries/t6.sparql}
	\mint{text}{---}
	\mint{text}{SQUALL:}
	\inputminted{text}{src/listings/sciqa-reference-queries/t6.txt}
	\mint{text}{---}
	\mint{text}{/* Mapped SPARQL to SQUALL using SPARQL2SQUALL. */}
	\caption{SciQA example for template 6.}
	\label{lst:sciqa_6}
\end{listing}

\begin{listing}[H]
	\mint{text}{SPARQL:}
	\inputminted{sparql}{src/listings/sciqa-reference-queries/t7.sparql}
	\mint{text}{---}
	\mint{text}{SQUALL:}
	\inputminted{text}{src/listings/sciqa-reference-queries/t7.txt}
	\mint{text}{---}
	\mint{text}{# Handcrafted SQUALL: Verbalization cannot be made non-ambiguous without brackets by SPARQL2SQUALL.}
	\caption{SciQA example for template 7.}
	\label{lst:sciqa_7}
\end{listing}

\begin{listing}[H]
	\mint{text}{SPARQL:}
	\inputminted{sparql}{src/listings/sciqa-reference-queries/t8.sparql}
	\mint{text}{---}
	\mint{text}{SQUALL:}
	\inputminted{text}{src/listings/sciqa-reference-queries/t8.txt}
	\mint{text}{---}
	\mint{text}{/* Mapped SPARQL to SQUALL using SPARQL2SQUALL. */}
	\caption{SciQA example for template 8.}
	\label{lst:sciqa_8}
\end{listing}

\newpage

\section*{Responsible Use of Generative Artificial Intelligence}
\addcontentsline{toc}{section}{Responsible Use of Generative Artificial Intelligence}  
\label{appendix:genAI}

The use of generative \gls{ai} in this thesis has been carefully restricted to two specific use cases:
serving as a writing assistant and a productivity enhancer.
In particular, ChatGPT was utilized to refine the flow of pre-written passages,
summarize chapters for use in introductions and conclusions,
and expedite the creation of TikZ schematics and graphs.
At no point was \gls{ai} used to generate content from scratch.

\newpage

\section*{Societal Reflection}
\addcontentsline{toc}{section}{Societal Reflection}  

Upon reflection,
it became clear that certain Sustainable Development Goals were inherently aligned with the research process,
even though they were not consciously adopted as guiding principles from the outset.

Great care was taken to ensure responsible production throughout the methodology and experimentation phases.
This was demonstrated by the judicious use of computational resources,
highlighted by the selection of a comparatively smaller domain-specific model, i.e., the domain expert,
which consumed significantly less computational power than, for example, a ChatGPT-based prompting approach.

Transparency and accessibility have also been central to this work.
Every step of the research has been explicitly documented and explained to ensure that the findings are accessible and
reproducible, fostering inclusivity and openness in scientific research.

Finally, this thesis originated at Polytechnique Montréal and was further developed upon my return to Ghent University.
I hope that this work not only contributes to advancing academic knowledge
but also fosters stronger collaborative ties between these two institutions.
I will enthusiastically encourage and recommend similar exchange opportunities.

\newpage

\end{appendices}
