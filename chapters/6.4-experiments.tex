\chapter{Experiments}
\label{s:experiments}

This chapter presents the experiments conducted to validate the requirements outlined in \Cref{s:validation}.
It begins with a detailed explanation of the experimental setup,
followed by five key experiments and their results.
The approach is then benchmarked against related work from the literature.
The chapter concludes with a discussion of the findings and their broader implications.

\section{Experimental Setup}

This section elaborates on the configuration of the \glsentrylong{kg},
the specific models utilized along with their respective settings,
and the software and hardware resources employed during the experiments.

\subsection{Knowledge Graph and Benchmark}
\label{s:knowledge_graph_and_benchmark}

\gls{orkg} was chosen to evaluate the proposed approach,
it is a scientific knowledge graph consisting of papers and authors.\footnote{ORKG: \url{orkg.org}.}
As of writing the \gls{kg} consists of \num{904212} unique entities,
although for this work the choice was made to work with an older version
--- because the SciQA benchmark, introduced in \Cref{s:gt_retriever_data}, that targets \gls{orkg} is outdated
--- consisting of \num{180184} and \num{6888} unique entities and relations, respectively.

\subsubsection{Motivation}

The subsequent paragraphs explain why the choice was made to work with this \gls{kg} and its associated benchmark.

\paragraph{Domain}

\gls{orkg} and SciQA are specialized, i.e., tailored to a specific instead of a general domain.
This is in line with the ontologies that occur in \gls{mbse}.
A \gls{kg} with no particular focus would be less appropriate.

\paragraph{Semantics}

As discussed in \Cref{s:training_finetuning}, the \gls{squall} expert was fine-tuned using the LC-QuAD 2.0 dataset,
which is designed for the Wikidata \gls{kg}.  
Despite substituting all \glspl{uri} with placeholders 
--- mapping \glspl{gp} to \glspl{pgp}
--- before training, the semantic information of the original \gls{kg} remains embedded in the queries.
This persistence occurs because the original queries are inherently structured to align with the \gls{kg};
otherwise, they would yield no results upon execution.\footnote{
	\gls{sparql} is a graph query language where queries are represented as \glsentryfullpl{bgp} or \glsentryfullpl{cgp}
	--- both of which are ultimately graphs (see \Cref{fig:sparqlBGP}).
	Executing a query involves matching a \gls{gp} against a \gls{kg},
	a process known as \glsentryfull{gpm} or subgraph matching.
	Consequently, any dataset of queries designed for a specific \glsentrylong{kg} will include \glspl{gp}
	with topologies that align with the structure of that graph.
}
Overlap between datasets can occur; for instance,
both DBpedia and Wikidata include the \mintinline{text}{marriage} relation between two \mintinline{text}{human} entities.
To mitigate such similarities, the approach is evaluated on a specialized domain.
If evaluation were instead conducted on DBpedia while the \gls{squall} expert was trained on Wikidata,
the results could appear overly optimistic.

\paragraph{Version}
\label{s:kg_version}

SciQA is out of date with the current version of \gls{orkg}: only 10 out of \num{2565} queries returned results.
Hence, the version of February 14, 2023 was used
--- \gls{rdf} dump made available at \cite{auerSciQABenchmarkDataset2023}
--- which returns query results in \approx 92\% of cases,
and contains \num{181147} entities and \num{6888} properties.
It was made accessible through a local \gls{sparql} endpoint using Apache Jena Fuseki.

\paragraph{Size}
\label{s:size_kgb}

The survey of ontology engineering projects mentioned in the introduction of \Cref{c:proposedSolution}
also gives insight into the size of real-world ontologies \cite{simperlAchievingMaturityState2009},
where ontology size is taken to be the number of unique entities.
The size of the chosen \gls{kg} is considered to be sufficiently large to be representative,
see \Cref{fig:ontologySizes}.

\begin{figure}[h!]
    \centering
    \includegraphics[width=\textwidth]{ontology-size-distribution.pdf}
	 \caption{
	 	Distribution of ontology sizes (i.e., number of entities)
		from a survey \cite{simperlAchievingMaturityState2009}.
	}
    \label{fig:ontologySizes}
\end{figure}

\subsubsection{Templates}
\label{s:templates}

% Data example
To better understand the structure of queries in the SciQA dataset,
\Cref{fig:combinedSciQAListings} provides two example \gls{sparql} queries,
and the equivalent \gls{squall} expressions.
Note that while the questions and \gls{sparql} queries are from SciQA,
the \gls{squall} expressions are constructed as part of this work.

\begin{listing}[ht!]

	\mintinline{text}{SPARQL:}
	\inputminted{sparql}{src/listings/running-example/pgp.sparql}
	\mintinline{text}{}

	\mintinline{text}{SQUALL:}
	\inputminted{text}{src/listings/running-example/squall.txt}
	\mintinline{text}{}

	\mintinline{text}{===}
	\mintinline{text}{}

	\mintinline{text}{SPARQL:}
	\inputminted{sparql}{src/listings/sciqa-example-b/pgp.sparql}
	\mintinline{text}{}

	\mintinline{text}{SQUALL:}
	\inputminted{text}{src/listings/sciqa-example-b/squall.txt}
	\mintinline{text}{}

	\caption{Example question-query pairs from SciQA with equivalent \glsentryshort{squall} expression.}
	\label{fig:combinedSciQAListings}
\end{listing}

One notable difference between the two examples is the presence of an \mintinline{sparql}{OPTIONAL} clause
in the bottom one.
This difference is also reflected in the corresponding \gls{squall} expression,
where it is represented by \mintinline{text}{if defined}.

Clearly, these two queries have distinct structures.
However, due to the way the SciQA dataset is created, see \Cref{s:gt_retriever_data},
there are only a limited quantity of such unique query structures, called templates, present, namely eight.
Note that this is typically the case for these types of datasets, e.g., LC-QuAD.
Although queries will vary, for example, in terms of which literals are present, the structure they have will always
belong to one of eight templates.

These examples highlight the distinct structures of the queries.
However, the SciQA dataset, as explained in \Cref{s:gt_retriever_data},
is constructed with a limited number of unique query structures, referred to as templates.
Specifically, there are eight such templates in total.
This is a common characteristic of datasets in this domain, such as LC-QuAD.
While the specifics (e.g., literals) of queries in the dataset may vary,
their underlying structure always conforms to one of these templates.

The query templates in SciQA are denoted by the set $T = \Set{ 1, \ldots, 8 }$.
When a specific template $x \in T $ is excluded, this is represented as $ T \setminus \{x\} $.

For an example of each template, the appendix can be consulted, see \Cref{appendix:sciqa_sparqls}.

\subsection{Model Configurations and Settings}

In the paragraphs ahead practical choices related to the implementation and experimentation are summarized.

\paragraph{Retrieval}
\label{s:settings_retrieval}

Before training the domain expert, the \gls{kg} index ($\mathscr{I}_{\text{\gls{kg}}}$) is first built using
Sentence-BERT \cite{reimersSentenceBERTSentenceEmbeddings2019} as embedding model ($\lambda$)
with the dimension set to \num{1024}.
The similarity function ($\sigma$) used for indexing was the cosine similarity.

\paragraph{Augmentation}

Following \cite{heGRetrieverRetrievalAugmentedGeneration2024}, the specific graph embedder used for embedding the
subgraphs was the \glsentrylong{gat} \cite{velickovicGraphAttentionNetworks2018},
with 4 layers, a hidden dimension ($d_g$) of \num{1024}, 4 heads and a dropout of 0.
The projector is implemented as a \glsentrylong{mlp} with 2 layers, a hidden dimension of \num{2048} and
the output dimension ($d_t$) is \num{4096}, aligned with the hidden dimension of the \gls{llm} used in the generator.
Some experimentation was done where the entire \gls{kg} served as graph token,
which required reducing the dimensions of the graph embedder, though this line of investigation was not further pursued.

\paragraph{Generation}

The \gls{llm} used for the generator component was Llama2-7b \cite{touvronLlama2Open2023}.
It served as the \gls{squall} expert that was frozen after being fine-tuned in all the experiments.
More specifically, a LLaMA-Adapter \cite{zhangLLaMAAdapterEfficientFinetuning2023} is fine-tuned.
The \gls{peft} technique that was used was \gls{qlora} \cite{dettmersQLoRAEfficientFinetuning2023}.
Parameter settings were informed by standard fine-tuning approaches for LLaMA v2.\footnote{
	The fine-tuning procedure was guided by the tutorial available at
	\url{https://www.kdnuggets.com/fine-tuning-llamav2-with-qlora-on-google-colab-for-free}.
}

\paragraph{Domain Expert}

Observing diminishing returns from initial experiments, patience was set to 3,
with all other parameters following the settings described by \cite{heGRetrieverRetrievalAugmentedGeneration2024}.

\paragraph{Linking}
\label{s:settings_linking}

The similarity function ($\sigma$) used during linking was kept from the earlier implementation
\cite{omarUniversalQuestionAnsweringPlatform2023}, 
it is based on the \glsentryfull{cf} \cite{castrofernandezSeepingSemanticsLinking2018}.
Given a placeholder string $s$ and an \mintinline{sparql}{rdfs:label} of a vertex or edge $x$,
they are represented as sequences of words $t$:
\[
	t = (w_i)_{i \in I}. \\
\]
The sequences are embedded using an operator $T$, here defined as:
\[
	T: t \mapsto \Set{ W_i^{\phi} =
		\begin{cases}
			\lambda(w_i) & w_i \in \text{dom}(\lambda) \land \phi = \lambda, \\
			\delta(w_i)  & w_i \notin \text{dom}(\lambda) \land \phi = \delta,
		\end{cases} \ | i \in I
	}.
\]

%\TODO{Fix citation}
In practice $\lambda$ and $\delta$ are the fastText \cite{bojanowskiEnrichingWordVectors2017}
and chars2vec model, respectively.\footnote{
	Chars2vec available at \url{https://github.com/IntuitionEngineeringTeam/chars2vec}.
}
The latter serves as a fallback option if the former fails to recognize the input.
The similarity is then defined as:
\[
	\sigma(s, x) = f_{\glsentryshort{cf}}(T(s), T(x)).
\]
Let $ \mathcal{S} = T(s) $ and $ \mathcal{X} = T(x) $,
then the \gls{cf} between sets of embeddings is defined as follows:
\[
	f_{\gls{cf}}(\mathcal{S}, \mathcal{X})
	= \frac{1}{|\mathcal{S}||\mathcal{X}|}
	\sum_{\left(S^{\phi_s}, X^{\phi_x}\right) \in \mathcal{S} \times \mathcal{X}}
	\sigma_{*}\left(S^{\phi_s}, X^{\phi_x}\right),
\]
where 
\[
	\sigma_{*}\left(S^{\phi_s}, X^{\phi_x}\right)
	= \mathbb{1} _ { \Set{ \phi_s = \phi_x } }
	\left(\sigma_{\text{cos}}\left(S^{\phi_s}, X^{\phi_x}\right)\right),
\]
using the indicator function to define embeddings originating from different models as perpendicular.
Linking back to the similarity function used for indexing, they are evidently quite distinct.
The works that proposed them in their original context either did not explore or
report any significant performance impact using alternatives.
As a comprehensive evaluation of different similarity functions was deemed out of scope,
both functions were retained as originally implemented.

\paragraph{Pre-Trained Large Language Model}

The OpenAI \gls{api} was used to create the synthetic datasets.

\paragraph{Software and Hardware}

The implementation uses a server designed to process incoming questions,
interact with the \gls{sparql} endpoint of the \gls{kg}, and return results.
This server is constructed using a lightweight Python Flask
application and is hosted on Apache with an \gls{aws} environment.
Both the server and the query endpoint are deployed on the same
\gls{aws} EC2 instance.

\subsection{PCST Parameter Triple}

The subsequent sections make use of a shorthand notation to indicate the parameters of the
\glsentryfull{pcst} algorithm.
The algorithm is used to construct relevant graphs for a given input question,
and it is implemented as the retrieval and post-retrieval components.

The parameter triple \((k_\text{v}, k_\text{e}, C)\) specifies
the top-\(k\) retrieved vertices (\(k_\text{v}\)),
the top-\(k\) retrieved edges (\(k_\text{e}\)),
and the cost per edge (\(C\)),
as defined in \Cref{eq:topkVertices}, \Cref{eq:topkEdges}, and \Cref{eq:graphCost}.

In practice, $k_\text{v}$ and $k_\text{e}$ are always equal and are collectively referred to as $k$.

\subsection{Notebook and KGQA Pipeline Connection}

The interaction between the notebook environment and the \gls{kgqa} pipeline is facilitated through a custom Python
package that introduces a user-friendly magic command for posing questions.
Developed by François Goybet as part of his bachelor's project in tandem with this master's thesis,
this package enhances usability by enabling seamless communication with the \gls{kgqa} pipeline hosted on an \gls{aws}
server.\footnote{
	Python package available at \url{https://pypi.org/project/molab-ext}.
}

Once the custom magic is loaded in a notebook cell, users can input a natural language question.
Upon execution, the question is sent to the pipeline, which automatically handles all intermediate steps
--- ranging from question parsing to result retrieval.
This process eliminates the need for stakeholders to interact directly with the underlying pipeline,
providing an intuitive interface for querying the system.

This extension, tailored to the requirements of this work,
was independently developed but designed to integrate seamlessly into the research setup.
The connection allows users to leverage the pipeline without requiring extensive technical knowledge.
This integration not only simplifies stakeholder interaction with the \gls{kgqa} pipeline
but also significantly enhances the accessibility and practicality of the system.

%%%%%%%%%%%%%%%

%%%% Experiments

% Hypothesis: what
% Hypothesis: why
% Actual results
% If unexpected: insight why this might be

\section{Retrieval and Post-Retrieval Analysis}
% Parameter exploration

% Intro
The first experiment investigates the retrieval and post-retrieval components applied to \gls{orkg},
introduced in \Cref{s:retrieval} and \Cref{s:postRetrieval}, respectively.

% Recap
These two components aim to construct a relevant subgraph of the \gls{kg} for a given question 
that can be used to improve the performance of the domain expert.
%
The running example from \Cref{c:methodology} is reintroduced here, see \Cref{fig:runningExamplePostRetrieval}.
Consider the following question:
\mintinline[breaklines]{text}{Provide a list of papers that have utilized the Depth DDPPO model and include the links to their code?}
When inputted to the retriever it will return relevant vertices and edges of the \gls{kg},
for example, there might be a vertex \mintinline{text}{Model} and an edge \mintinline{text}{HAS_SOURCE_CODE}.
The post-retrieval component is then responsible for creating a connected graph out of these looses vertices and edges.

% Why
The primary objective of retrieval is to be as informative as possible.
However, two inherent limitations of any \gls{rag} paradigm are poor and over-retrieval,
see \Cref{s:why_placeholders}, both negatively impacting generation.
%
This implies that a poor parameter choice likely reduces the performance of the domain expert.
%
Ideally, the subgraphs at the end of post-retrieval capture all the necessary vertices and edges needed for the 
domain expert to correctly generate the expected \gls{squall}, while remaining as compact as possible.
Hence, it is important to have some notion of how the parameters of the retrieval and post-retrieval components
influence the constructed subgraphs.

% What
What is being evaluated is essentially how informative the subgraphs are in function of the
\gls{pcst} parameters.

\subsection{Hypothesis}

% Hypothesis: what
Increasing the top-$k$ parameter is expected to result in larger subgraphs with higher recall for both entities and
relations.
The cost per edge is anticipated to provide fine control over subgraph size,
for example, increasing the cost per edge will likely contribute to noise reduction by steering the algorithm
towards smaller subgraphs.

% Hypothesis: why
The retriever's parameters are presumed to have a direct impact on the quality of the constructed subgraphs.
Larger subgraphs are likely to capture more of the relevant vertices and edges.
However, excessive size may introduce unnecessary noise.
A higher cost per edge is expected to mitigate noise,
focusing the subgraphs on the most informative components while maintaining high recall.

\subsection{Setup}
\label{s:retrievalAnalysis}

% How
Evaluation is performed by varying the \gls{pcst} parameters and investigating how it impacts the subgraphs,
outputted by the post-retrieval component,
using average vertex/edge recall and average number of vertices/edges,
defined in \Cref{eq:avgVertexRecall} and \Cref{eq:avgEdgeRecall}, respectively.

\subsection{Results}
\label{s:retriever_parameters_results}

% Actual results
The results, see \Cref{table:retrieverRecall},
indicate that both the size of the constructed graphs and the recall increase with top-$k$,
confirming the straightforward intuition that more lenience in terms of which vertices and edges are considered 
relevant, results in bigger constructed subgraphs containing at least as many of the expected vertices and edges.
Evidently, if the entire graph would be returned from the retriever, recall would be equal to 1.
As for the cost per edge, increasing it reliably results in a reduced subgraph size, all else being equal.
The parameter seems especially valuable for higher top-$k$ values, since the loss in recall reduces (to zero),
while resulting in about half the amount of vertices/edges.

\begin{table}[t]

	\centering
	\setlength{\tabcolsep}{3pt} % Default is 6pt

	\begin{tabular}{l|SS|SS}
		\textbf{Parameters}
		& \textbf{Average \# Vertices}
		& \textbf{Average \# Edges}
		& \textbf{Average Vertex Recall}
		& \textbf{Average Edge Recall} \\
		\hline
		(5, 5, 0.1)       & 26   & 28    & 0.00  & 3.57 \\
		(5, 5, 0.5)       & 16   & 15    & 0.00  & 3.57 \\
		(10, 10, 0.1)     & 49   & 56    & 15.98 & 26.36 \\
		(10, 10, 0.5)     & 35   & 37    & 3.57  & 16.07 \\
		(30, 30, 0.1)     & 197  & 255   & 10.71 & 42.86 \\
		(30, 30, 0.5)     & 104  & 115   & 27.82 & 44.19 \\
		(100, 100, 0.1)   & 739  & 1150  & 21.43 & 78.57 \\
		(100, 100, 0.5)   & 386  & 477   & 42.72 & 65.08 \\
		(1000, 1000, 0.1) & 7316 & 11341 & 90.41 & 95.63 \\
		(1000, 1000, 0.5) & 3126 & 5166  & 88.75 & 95.23 \\

	\end{tabular}

	\caption{
		The average number of vertices and edges,
		and the average vertex and edge recall (expressed as percentages),
		in function of \glsentryshort{pcst} parameters.
	}
	\label{table:retrieverRecall}

\end{table}


% If unexpected: insight why this might be
Notably, the execution time of the \gls{pcst} algorithm remains stable even as the top-$k$ vertices and edges increase
and the cost per edge decreases, with only a slight increase in running time.

\section{Synthetic Dataset Quality}

% Intro
This experiment gauges the quality of the synthetic dataset generated using the rudimentary strategy,
as explained in \Cref{s:syn_retriever_data}, and used to train the domain expert, see \Cref{s:trainingDomainExpert}.

% Recap
A high-quality training dataset containing question-query pairs for a specific domain is often unavailable
in practice due to data scarcity.
This work addresses this challenge primarily through \glsentryfull{da} and the use of the \glsentrylong{cnl} \gls{squall}.
However, synthetic data can also play a vital role in mitigating data scarcity.
To explore this, the performance of the domain expert is compared when trained on ground truth data versus synthetic data,
as discussed in \Cref{s:generalizationCapabilities}.

% Why
A basic synthetic data generation approach was deliberately chosen 
--- advanced strategies are considered out of scope
--- nevertheless, evaluating the quality of the generated synthetic data remains essential for subsequent experiments,
as it provides critical insights when interpreting the results.

% What
Assessing the quality of generated synthetic data is straightforward when ground truths are available,
as is the case here, direct evaluation can be performed
which boils down to evaluating how faithful the generated samples are to their references 
\cite{longLLMsDrivenSyntheticData2024}, ideally they are identical.
See \Cref{lst:syntheticDatasetQualityEval} for an example.

\begin{listing}[!ht]

\mintinline{text}{Ground Truth SQUALL:}
\inputminted{text}{src/listings/running-example/squall.txt}
\mintinline{text}{---}

\mintinline{text}{High quality synthetic SQUALL:}
\inputminted{text}{src/listings/running-example/squall-synthetic-high-quality.txt}
\begin{minted}{text}
> BLEU score: 90
> One placeholder not faithful to reference:
> <has code link> instead of <has source code>. 
\end{minted}
\mintinline{text}{---}

\mintinline{text}{Low quality synthetic SQUALL:}
\inputminted{text}{src/listings/running-example/squall-synthetic-low-quality.txt}
\begin{minted}{text}
> BLEU score: 23
> Wrong demonstration followed from prompt.
\end{minted}

\caption{
	Comparison of ground truth and synthetic \glsentryshort{squall} for a question,
	highlighting high-quality and low-quality examples,
	along with their \glsentryshort{bleu} scores.
}
\label{lst:syntheticDatasetQualityEval}
\end{listing}


\subsection{Hypothesis}

% Hypothesis: what
The higher the \gls{dpt} and the more templates are included in the prompt likely improves performance up to a certain
point.

% Hypothesis: why
Given that only eight templates are present in the SciQA benchmark, it seems improbable that the inclusion of a template
would confuse the \gls{pllm}.
Furthermore, increasing the \gls{dpt} gives more examples to the model, which should intuitively increase faithfulness.

\subsection{Setup}
\label{s:synthetic_dataset}

% How
Faithfulness between generated and reference \gls{squall} queries is evaluated using the \gls{bleu} score.
Elementary prompt engineering experiments are conducted to examine how variations in the support set
($\mathscr{D}_{\text{sup}}$) affect data quality.

Two engineering dimensions are explored:

\begin{itemize}

	\item \textbf{Template Variations}:

		As outlined in \Cref{s:templates}, SciQA queries belong to one of eight templates.
		This axis examines the effect of excluding certain templates from the support set.
		Three configurations are tested:

		\begin{itemize}
			\item All templates included: $T$.
			\item Excluding template 2: $T \setminus \{2\}$.
			\item Excluding template 6: $T \setminus \{6\}$.
		\end{itemize}

		These configurations enable evaluation of the faithfulness of generated samples when a template is absent from the
		prompt.

		Template 2 was excluded due to its significant dissimilarity to the other query templates,
		making it a strong candidate for assessing the impact of exclusion on quality.
		In contrast, templates 4 and 5 were the most similar to each other,
		it is thus likely that excluding one or the other would result in minimal deterioration of the synthetic data quality.
		Template 6 was excluded for a similar reason to template 2, as it also exhibited notable dissimilarity.

		The dissimilarity was calculated by grouping the queries from the ground truth dataset by template,
		extracting all tokens associated with each template, and computing TF-IDF scores.
		Pairwise cosine similarity was then calculated for all template combinations,
		with templates ranked from most similar to most dissimilar.
		Templates 6 and 2 appeared as the least similar templates in the ranking,
		being the penultimate and last templates, respectively, to form a pair.

		While the ideal scenario would involve testing one setup for each excluded template,
		practical constraints necessitated focusing on this subset.
		This selection was deemed most suitable for evaluating generalization to unseen and,
		likely, significantly (more) dissimilar templates in real-world applications.

	\item \textbf{Number of \glsentryfullpl{dpt}}:
		The second axis explores the impact of varying the number of \glspl{dpt}.
		Two setups are considered: 

		\begin{itemize}
			\item One \gls{dpt}.
			\item Three \glspl{dpt}.
		\end{itemize}

\end{itemize}

In total, these variations result in six setups with each a unique configuration of the support set
($\mathscr{D}_{\text{sup}}$).
For each configuration, corpus-level \gls{bleu} scores are computed across all samples and on a per-template basis.
The ground truth dataset ($\mathscr{D}_{\text{gt}}$) serves as the reference for evaluating all generated synthetic
datasets.

The only factor differentiating the six setups is the composition of $\mathscr{D}_{\text{sup}}$,
see \Cref{s:syn_retriever_data}.

\subsection{Results}
\label{s:synthetic_dataset_quality}

% Actual results
The results of the experiment are given in \Cref{table:synthetic_dataset_bleu}.

\begin{table}[t]
	\centering
	%\setlength{\tabcolsep}{6pt} % Default is 6pt
	\begin{tabular}{
		l|S[table-format=2.0] S[table-format=2.0] S[table-format=2.0]|S[table-format=2.0] S[table-format=2.0] S[table-format=2.0]
	}
		\textbf{Templates} 
					  & \multicolumn{3}{c|}{\textbf{1 Demonstration per Template}}
					  & \multicolumn{3}{c}{\textbf{3 Demonstrations per Template}} \\
					  
					  & {$T$} & {$T \setminus \{ 2 \}$} & {$T \setminus \{ 6 \}$} 
					  & {$T$} & {$T \setminus \{ 2 \}$} & {$T \setminus \{ 6 \}$} \\
					  \hline

		$T$ & 74 & 73             & 74             & 93  & 88             & 92 \\
		\hline
		1   & 65 & 71             & 69             & 98  & 99             & 97 \\
		2   & 81 & \underline{38}     & 81             & 94  & \underline{37} & 96 \\
		3   & 89 & 89             & 91             & 98  & 96             & 98 \\
		4   & 76 & 74             & 79             & 93  & 91             & 94 \\
		5   & 67 & 69             & 63             & 91  & 94             & 90 \\
		6   & 79 & 75             & \uline{51} & 97  & 93             & \underline{56} \\
		7   & 71 & 72             & 71             & 87  & 88             & 86 \\
		8   & 72 & 82             & 83             & 100 & 100            & 95 \\

	\end{tabular}

	\caption{
		Quality assessment of the synthetic dataset using \glsentryshort{bleu} scores expressed as percentages.
		This evaluation compares the generated \glsentryshort{squall} queries to their corresponding ground truths,
		including an analysis of queries generated with certain demonstrations excluded from the prompt's support set.
	}

	\label{table:synthetic_dataset_bleu}
\end{table}


What is being evaluated here is in essence the \gls{pllm}'s ability to annotate data \cite{zhuCanChatGPTReproduce2023}.
As expected, it is unequivocal that increasing the \gls{dpt} from one to three improved the faithfulness of the data,
although no further conclusion can be made with this limited setup.

Moreover, the removal of a template from the prompt clearly has a substantial impact on the data quality
for that specific label, but this adverse effect is not equal across all templates,
e.g., the quality of the generated samples following query template 2 in the $T \setminus \{2\}$ case
is notably worse than the quality of the generated samples following query template 6 in the $T \setminus \{6\}$ case.
This could be related to the fact that template 2 is more dissimilar to the remaining demonstrations than template 6 is,
see \Cref{s:synthetic_dataset}. 
Finally, the removal of one template from the prompt has no significant impact on the quality of generated samples
belonging to the remaining templates.

% If unexpected: insight why this might be
No further conclusions can be made from this limited setup.

\section{Text-to-SQUALL}

% Intro
This section introduces the first evaluation of the domain expert,
particularly its ability to adapt to new a domain that differs from the source domain used during fine-tuning of the 
\gls{squall} expert.

% Recap
The domain expert, see \Cref{fig:domain_expert}, is the \gls{rag} model that generates \gls{squall} given a question.
Note that these \gls{squall} expressions do not yet contain any \glspl{uri} of the target \gls{kg}.
In order to obtain an executable \gls{sparql} query, a \gls{squall} expression must still pass through the remainder
of the text-to-\gls{sparql} pipeline consisting of translation and linking,
see \Cref{fig:textToSparql}.

% Why
The objective of this experiment is to evaluate the performance of the domain expert in isolation,
separate from the rest of the pipeline.
This approach minimizes confounding factors, such as inaccuracies introduced by an imperfect linking procedure.
Additionally, the evaluation focuses solely on the domain expert's capabilities before incorporating synthetic data;
the model is trained exclusively on the ground truth dataset ($\mathscr{D}_{\text{gt}}$).

% What
The evaluation assesses the model's ability to generate \gls{squall} queries with placeholders,
comparing its outputs against ground truths for various \gls{pcst} parameter configurations.
While the first experiment (see \Cref{s:retrievalAnalysis}) provides preliminary insights into the influence of these
parameters,
this experiment extends those findings by directly examining their impact on the quality of the generated queries,
as discussed in \Cref{s:retriever_init_eval}.

\subsection{Hypothesis}

% Hypothesis: what
Intuitively, more informative subgraph should enhance performance,
particularly when the retrieval and post-retrieval components achieve higher average recall
and maintain smaller graph sizes on average.

% Hypothesis: why
This assumption is grounded in the expectation that
increasing relevant information and decreasing distracting noise improve the generation process.

\subsection{Setup}

% How
Evaluation is done using \gls{bleu} scores, comparing generated to reference \gls{squall}.

\subsubsection{Reducing the Number of PCST Parameter Configurations}

Some parameter configurations were discarded due to practical considerations, such as running time, 
as informed by the retrieval and post-retrieval analysis experiment (see \Cref{s:retriever_parameters_results})
and a heuristic assessment.

For each pair of \gls{pcst} parameter triples where \( k \) is identical but the cost-per-edge differs, 
one configuration was excluded from further investigation. 
Consider the following pair from \Cref{table:retrieverRecall}: (1000, 1000, 0.1) and (1000, 1000, 0.5). 
Experimental results indicate that while the former achieves slightly higher vertex and edge recall, 
it also produces subgraphs that are, on average, more than twice the size in terms of vertices and edges. 
This is a direct consequence of its lower cost-per-edge parameter, and demonstrates a case of diminishing returns.

% Formally
The heuristic for deciding which parameter triple from a pair ($p_1, p_2$) to keep, is defined as follows:

\[
	p^* =
	\begin{cases}
		p_1 & \text{if } \wideoverbar{R}_{V, \mathscr{D}_1} \gg \wideoverbar{R}_{V, \mathscr{D}_2}, \\
		p_2 & \text{if } \wideoverbar{R}_{V, \mathscr{D}_2} \gg \wideoverbar{R}_{V, \mathscr{D}_1}, \\
		p_1 & \text{if } \wideoverbar{V}_{\mathscr{D}_1}    <   \wideoverbar{V}_{\mathscr{D}_2}, \\
		p_2 & \text{if } \wideoverbar{V}_{\mathscr{D}_2}    \le \wideoverbar{V}_{\mathscr{D}_2}.
	\end{cases}
\]

Here, \( p^* \) denotes the retained parameter tuple. 
\( \wideoverbar{R}_{V, \mathscr{D}_i} \) and \( \wideoverbar{V}_{\mathscr{D}_i} \) represent
the average vertex recall, see \Cref{eq:avgVertexRecall}
and the average number of vertices, see \Cref{eq:avgVertices},
respectively, 
for the dataset resulting from retrieval and post-retrieval when the parameters from \( p_i \) were used.

In practical terms, a difference in recall was considered significant if it was at least 10. 
For the previous example, the difference in recall was only 1.66. 
As a result, the parameter triple (1000, 1000, 0.5) was selected, as it yields more compact subgraphs.

Finally, parameter triples with \( k = 5 \) were excluded, as they resulted in an average vertex recall of zero.
Instead, a baseline approach was employed.

\subsubsection{Baseline}

% Why?
The post-retrieval component outputs a graph,
that helps the domain expert during generation through the augmentation process,
see \Cref{fig:domain_expert}.
To assess the impact of different \gls{pcst} parameters on the generation process, a baseline is established.

% What?
The baseline was determined based on two key observations. 
First, the output of the post-retrieval component is a graph. 
Second, the largest and smallest graphs that the post-retrieval component could theoretically produce are
the entire \gls{kg} and the zero-order graph \( K_0 \), respectively. 
Here, \( K_0 \) refers to a graph containing neither vertices nor edges. 
While both extremes were tested, using the entire \gls{kg} proved unfeasible due to computational constraints.

% How?
This experiment used the baseline by providing \( K_0 \) as input to the augmentation process of the domain expert
during inference.

\subsubsection{Internal Evaluation}
\label{s:internalEvaluation}

It is important to emphasize that the \gls{bleu} scores obtained from evaluating the \gls{squall} expressions generated
by the domain expert are meaningful only within the scope of this work.
Typically, in \glsentrylong{sp}, \gls{bleu} scores are calculated between reference and generated \gls{sparql} queries.
However, the \gls{squall} expressions must still pass through the remainder of the text-to-\gls{sparql} pipeline before
becoming \gls{sparql} queries, as illustrated in \Cref{fig:textToSparql}.

Nevertheless, \gls{bleu} scores based on \gls{squall} expressions are employed as they enable an effective evaluation
of the domain expert.

Consequently, evaluations based on \gls{squall} expressions are consistently employed,
but strictly for internal evaluation purposes.

\subsection{Results}
\label{s:text_to_prelinked_squall}

% Actual results
Given that no previous results were found that detail the impact of input graph size or recall on graph prompting
multiple models were trained with differing \gls{pcst} parameters to test
their impact on the text-to-\gls{squall} task.
The results are shown in \Cref{table:text_to_prelinked_squall}.

\begin{table*}[t]
	\centering
	\begin{tabular}{l|S[table-format=2.0]S[table-format=2.0]S[table-format=2.0]S[table-format=2.0]S[table-format=2.0]}
		\textbf{Templates} & \textbf{Baseline ($\mathbf{K_0}$)} & $\textbf{(10, 10, 0.1)}$ & $\textbf{(30, 30, 0.5)}$ & $\textbf{(100, 100, 0.5)}$ & $\textbf{(1000, 1000, 0.5)}$ \\
		\hline
		$T$ & 98            & 94  & 82  & 91  & 95 \\
		\hline
		1  & 99             & 93  & 74  & 84  & 95 \\
		2  & 100            & 99  & 85  & 99  & 98 \\
		3  & 100            & 84  & 61  & 70  & 98 \\
		4  & 99             & 91  & 84  & 94  & 90 \\
		5  & 100            & 97  & 79  & 95  & 99 \\
		6  & \underline{40} & 53  & 56  & 87  & 50 \\
		7  & 100            & 100 & 99  & 99  & 94 \\
		8  & \text{N/A}     & \text{N/A} & \text{N/A} & \text{N/A} & \text{N/A} \\

	\end{tabular}
	\caption{
		Evaluation of the domain expert's performance on the text-to-\glsentryshort{squall} task using the \glsentryshort{bleu} score,
		expressed as precentages, in function of different \glsentryshort{pcst} parameters.
	}
	\label{table:text_to_prelinked_squall}
\end{table*}



% If unexpected: insight why this might be
Performance in the $K_0$ scenario is notably and unexpectedly high.
The fact that the best performance is achieved for $K_0$ 
is likely due to the relatively homogeneous nature of the SciQA dataset.
This homogeneity is characterized by recurring entities and relationships within the \gls{sparql} queries,
thus somewhat nullifying the potential benefit of retrieval.

To concretize this, SciQA is compared to three other benchmarks in terms of its vocabulary.
The vocabulary of a \gls{kb}, denoted by $K$,
includes \glspl{uri} and literals \cite{dialloComprehensiveEvaluationNeural2024}.
In the context of \gls{rdf} datasets: $ K = I \cup L$, see \Cref{def:rdfDataset}.
A comparison of the datasets and their vocabularies is given by \Cref{table:comparisonDatasets} and
\Cref{table:comparisonVocabularies}, respectively.

\begin{table*}[h]

	\centering

	\begin{tabular}{l|S[table-format=6.0]S[table-format=6.0]S[table-format=5.0]S[table-format=5.0]|r}

		\textbf{Dataset}     & \textbf{Total} & \textbf{Train} & \textbf{Val} & \textbf{Test} & \textbf{Source} \\
		\hline
		\textbf{LC-QuAD}     & 5000     & 4000     & 500    & 500     & \cite{dialloComprehensiveEvaluationNeural2024} \\
		\textbf{LC-QuAD 2.0} & 30225    & 21761    & 2418   & 6046    & \cite{dialloComprehensiveEvaluationNeural2024} \\
		\textbf{DBNQA}       & 382794   & 306236   & 38279  & 38279   & \cite{dialloComprehensiveEvaluationNeural2024} \\
		\textbf{SciQA}       & 2565     & 1795     & 257    & 513     & This work

	\end{tabular}

	\caption{
		Number of samples of four datasets, broken down into training, validation, and test splits,
		along with the overall totals.
	}

	\label{table:comparisonDatasets}
\end{table*}

\begin{table*}[h]

	\centering

	\begin{tabular}{l|S[table-format=6.0]S[table-format=6.0]S[table-format=5.0]S[table-format=5.0]S[table-format=4.0]|r}
		\textbf{Dataset}            & \textbf{Total} & \textbf{Train} & \textbf{Val} & \textbf{Test} & \textbf{\glsentryshort{oov}} & \textbf{Source} \\
		\hline
		\textbf{LC-QuAD}            & 4751     & 4150     & 1068   & 1065    & 318                    & \cite{dialloComprehensiveEvaluationNeural2024} \\
		\textbf{LC-QuAD 2.0}        & 34801    & 27949    & 5413   & 10993   & 7523                   & \cite{dialloComprehensiveEvaluationNeural2024} \\
		\textbf{DBNQA}              & 157672   & 142985   & 36845  & 37130   & 7998                   & \cite{dialloComprehensiveEvaluationNeural2024} \\
		\textbf{SciQA}              & 1347     & 1077     & 297    & 483     & 182                    & This work \\
		\textbf{SciQA, \glspl{uri}} & 188      & 135      & 50     & 69      & 37                     & This work

	\end{tabular}

	\caption{
		\glsentrylong{kb} vocabulary sizes.
	}

	\label{table:comparisonVocabularies}
\end{table*}

\begin{table*}[h]

	\centering

	\begin{tabular}{l|S[table-format=1.3]S[table-format=1.3]S[table-format=1.3]S[table-format=1.3]S[table-format=1.3]}

		\textbf{Dataset}            & \textbf{Total}  & \textbf{Train} & \textbf{Val}     & \textbf{Test}    & \textbf{\glsentryshort{oov}} \\
		\hline
		\textbf{LC-QuAD}            & 0.950          &  1.038        &  2.136           &  2.130            &  0.071 \\
		\textbf{LC-QuAD 2.0}        & 1.151          &  1.284        &  2.239           &  1.818          &  0.311 \\
		\textbf{DBNQA}              & 0.412          &  0.467        &  0.963           &  0.970            &  0.023 \\
		\textbf{SciQA}              & 0.525          &  0.600        &  1.156           &  0.942          &  0.089 \\
		\textbf{SciQA, \glspl{uri}} & 0.073          &  0.075        &  0.195           &  0.135          &  0.018 

	\end{tabular}

	\caption{
		\glsentrylong{kb} vocabulary size to number of dataset samples.
	}

	\label{table:comparisonRatios}
\end{table*}



Clearly, SciQA is a relatively small dataset with a correspondingly limited vocabulary.
However, when adjusting for dataset size by examining the ratio of vocabulary to dataset size
(as shown in \Cref{table:comparisonRatios}), a clearer picture emerges.
From this perspective, SciQA exhibits a relatively poor vocabulary density:
only the much larger DBNQA dataset has comparable ratios.
This characteristic of SciQA likely makes memorization more effective,
offering a partial explanation for the high performance of the $K_0$ baseline.

Secondly, while SciQA's \gls{oov} ratio is higher than two other datasets,
suggesting a more challenging test split, this observation requires further context.
The vocabularies reported in previous tables include both \glspl{uri} and literals.
In SciQA, however, literals significantly outweigh \glspl{uri} in terms of occurrence,
thus \Cref{table:comparisonVocabularies} and \Cref{table:comparisonRatios} also give the 
vocabulary size and its ratio to the number of samples when only \glspl{uri} are considered for SciQA.
The number of unique \glspl{uri} is therefore included in parentheses for clarity where applicable.

Moreover, literals frequently appear verbatim in the original questions, including those in the test set.
Specifically, literals occur in approximately 90\% of the samples.
Recalling the example from \Cref{s:gt_retriever_data}, the literal \mintinline{text}{Depth DDPPO}
appears exactly in the question, see \Cref{lst:verbatimLiteral}.

\begin{listing}[!ht]

\begin{minted}{text}
SQUALL:
Provide a list of papers that have utilized the Depth DDPPO model and include the links to their code?

\end{minted}

\mint{text}{SPARQL:}
\begin{minted}{sparql}
SELECT DISTINCT ?code WHERE {
	?model a orkgc:Model;
		rdfs:label ?model_lbl .
	FILTER(STR(?model_lbl) = 'Depth DDPPO')
	?benchmark orkgp:HAS_DATASET ?dataset .
	?cont orkgp:HAS_BENCHMARK ?benchmark .
	?cont orkgp:HAS_MODEL ?model;
		orkgp:HAS_SOURCE_CODE ?code .
}
\end{minted}

\caption{String literal present in both the question and its equivalent query.}
\label{lst:verbatimLiteral}
\end{listing}

This heavy reliance on literals further reduces the vocabulary's expressiveness
and may contribute to the observed performance patterns.

Despite these limitations,
incorporating more informative subgraphs during graph prompting
--- particularly those with higher vertex and edge recall
--- demonstrates a measurable improvement in performance.
While the (1000, 1000, 0.5) parameter configuration does not surpass the $K_0$ baseline,
it still outperforms all other tested configurations.
Consequently, this combination was selected for further evaluation of the model's generalization capabilities.

Finally, it is hypothesized that the proposed \gls{rag} paradigm would show greater benefits and that the baseline
would not remain the best-performing setup if a dataset with a richer and more diverse vocabulary were used.
However, the current findings remain inconclusive due to the limitations of the dataset.
These limitations neither confirm nor entirely refute the hypothesis, underscoring the need for further investigation.

In summary, while these results provide insights,
additional research with richer datasets and varied experimental setups is essential to validate the hypothesis
and fully realize the potential of the proposed approach, as outlined in \Cref{s:discussion}.

\section{Generalization Capabilities}
\label{s:generalizationCapabilities}

% Intro
The practical applicability of the domain expert is now assessed.

% Recap
In practice,
there are typically no ground truth datasets available for systems in \gls{mbse} comparable to the SciQA dataset.
However, a data generation strategy could be used to create a synthetic dataset instead.
Furthermore, no matter the dataset used to train the domain expert,
there will always question-query templates it has not seen during its training.
For example, a question that needs to be answered by a query with a structure totally different from the seen templates.

% Why
The aim of this experiment is to understand how well the domain expert generalizes to realistic, data-scarce,
scenarios.
Due to the scarcity of data in \gls{mbse}, it is crucial to evaluate whether the proposed domain expert performs effectively
when trained on synthetic data or when encountering unseen question-query templates.

\subsection{Hypothesis}

% Hypothesis: what
It was hypothesized that reducing data quality
--- either through the use of synthetic data or the exclusion of templates during training
--- would negatively impact performance.
While a basic synthetic data generation strategy was employed,
it was not expected to significantly degrade the domain expert.
In contrast, template exclusion was anticipated to have a pronounced impact,
as supported by prior work.

% Hypothesis: What
The hypothesis is that reducing data quality
--- either through the use of synthetic data or the exclusion of templates during training
--- would negatively impact performance.
While a basic synthetic data generation strategy was applied,
it was not anticipated to significantly degrade the domain expert’s performance.
In contrast, template exclusion was expected to have a more pronounced impact, as suggested by prior work.

% Hypothesis: why
These hypotheses rest on two key assumptions.

First, the \gls{squall} expert
--- used as the generator component of the domain expert, see \Cref{fig:domain_expert}
--- is presumed to possess sufficient knowledge of \gls{squall}.
Before training, the domain expert is generally expected to generate syntactically correct \gls{squall} in most cases,
due to the \gls{squall} expert.
The domain expert's training is primarily done to align the domain expert with the target domain
such that it can generate \gls{squall} that is also semantically correct.

Second, \glspl{llm} are known to struggle with generalizing to unseen question-query templates
\cite{reydAssessingGeneralizationCapabilities2023}.
Consequently, the proposed approach is not expected to be immune to this limitation.

In essence, training the domain expert means that it learns how to generate particular query structures from questions:
it learns the semantics of the domain.
It is unreasonable to think the domain expert would be able to generate a semantically correct query for a 
question if it has not seen that particular type of question-query structure before.

Given that the synthetic data was found to be of reasonable quality
--- when examples for all question-query templates were included in the prompt
--- its impact on performance was not expected to be substantial, see \Cref{table:synthetic_dataset_bleu}.

\subsection{Setup}

% How
Performance is assessed using \gls{bleu} scores, comparing generated \gls{squall} expressions against their references.
There are again two evaluation dimensions:

\begin{itemize}
	\item \textbf{Training Data Type}: Ground truth versus synthetic data.
	\item \textbf{Template Familiarity}:
		Performance on all templates when template is seen during training versus left out.
\end{itemize}

Combining these dimensions results in four distinct evaluation setups.

A note regarding the setup involving synthetic data where queries from template 2 were excluded from the
training data:
the prompt used to generate this synthetic data also omitted any examples from template 2.
If examples from template 2 had been included in the prompt, the \gls{pllm} could have inadvertently generated
\gls{squall} expressions conforming to the template.
This would have resulted in samples with template 2 leaking into the training data of the domain expert.

To reduce the number of experiments, the \gls{pcst} parameters for all setups are set to (1000, 1000, 0.5).
This parameter triple resulted in the best results besides the baseline, see \Cref{s:text_to_prelinked_squall}.

\subsection{Results}

% Actual results
The experimental results indicate performance degradation when reducing training data quality,
see \Cref{table:generalization_capabilities_bleu}.

\begin{table}[t]
	\centering
	\begin{tabular}{
			l|S[table-format=2.0] S[table-format=2.0]|S[table-format=2.0] S[table-format=2.0]
		}

		\textbf{Templates}  
							 & \multicolumn{2}{c|}{\textbf{Ground Truth Target Domain Data}}
							 & \multicolumn{2}{c} {\textbf{Synthetic Target Domain Data}} \\
							 & {$T$} & {$T \setminus \{2\}$} 
							 & {$T$} & {$T \setminus \{2\}$} \\
							 \hline
	$T$          & 95  & 90             & 78  & 77 \\
	\hline
	1            & 95  & 92             & 93  & 85 \\
	2            & 98  & \underline{40} & 42  & 36 \\
	3            & 98  & 91             & 85  & 85 \\
	4            & 90  & 96             & 47  & 65 \\
	5            & 99  & 99             & 89  & 93 \\
	6            & 50  & 76             & 35  & 40 \\
	7            & 94  & 95             & 81  & 64 \\
	8            & \text{N/A} & \text{N/A} & \text{N/A} & \text{N/A} \\

\end{tabular}

\caption{
	Evaluation of the domain expert's ability to generalize on the text-to-\glsentryshort{squall} task using the \glsentryshort{bleu} score,
	expressed as percentages, for different training datasets.
}
\label{table:generalization_capabilities_bleu}
\end{table}


For the ground truth case, when template 2 is unseen during training the domain expert expectedly does a much
poorer job,
underscoring the common challenge of generalizing to unseen question-query templates in text-to-query models.
The limited size of the training dataset may be insufficient for generalizing to new questions targeting
\gls{orkg} \cite{yinNeuralMachineTranslating2021}.
Despite extensive fine-tuning of the \gls{squall} expert,
%which enables adaptation to \gls{orkg} with fewer data samples compared to training from scratch,
it remains essential that the training dataset for the domain expert is representative of the target domain.

The performance of the domain expert when trained on generated data is evidently affected by the relatively lower
quality of these synthetic samples.

% If unexpected: insight why this might be
However, some observations are unexpected:

\begin{itemize}

	\item \textbf{Template 8}:  
		Only one test sample was available, which limits the ability to draw significant conclusions from its performance.

	\item \textbf{Template 6}:  
		The poor performance of template 6 can likely be attributed to the limited number of training samples following
		this template
		--- 48, which constitutes less than 3\% of the total training data.
		Given that models tend to favor the most frequent query templates in the training data
		\cite{dialloComprehensiveEvaluationNeural2024}, it is reasonable to assume that the infrequency of this template
		may result in its questions being incorrectly categorized as belonging to other, more common templates.

	\item \textbf{Template 4 for Synthetic Target Domain Data}: 
		The poor performance of template 4 remains unclear.
		Although the quality of its synthetic data is slightly below average, as shown in
		\Cref{table:synthetic_dataset_bleu}, and it scores somewhat worse than average for the ground truth case,
		it seems unlikely that these factors alone would account for such a significant decline in performance.
		It is possible that the combination of these issues confuses the domain expert in a manner similar to template 6.  
		To investigate this further, all questions were combined on a per-template basis into a single text,
		which was then transformed into a vector representation using TF-IDF.
		Subsequently, the templates were compared pairwise using cosine similarity.
		Interestingly, the most similar pair of templates was template 4 and template 5.
		Furthermore, template 5 contains slightly more questions than template 4
		--- approximately 19\% compared to 14\%.
		All these factors combined may explain why the model occasionally generates queries that follow template 5 when
		they should actually follow template 4.

	\item \textbf{Template 2 for Synthetic Target Domain Data on $T:$}
		The same analysis that was applied to template 4 did not reveal any clear reasons for template 2's poor
		performance.
		This suggests that a deeper investigation is required to identify the underlying cause.

\end{itemize}

%An important observation stated by \cite{reydAssessingGeneralizationCapabilities2023} is that it is not clear whether
%models generate queries based on the semantics of the input question,
%or merely learn how to map them to particular structures.
%The difference is profound.
%Assuming the latter is correct, 
%this would be the case for both the \gls{pllm} used to generate the synthetic data as well as the domain expert.

\section{Text-to-SPARQL}

% Intro
The penultimate experiment evaluates the text-to-\gls{sparql} component on the eponymous task.

% Recap
As illustrated in \Cref{fig:textToSparql}, this component takes text as input and outputs \gls{sparql}.
It is composed of three subcomponents:

\begin{enumerate}
	\item The domain expert, which generates an equivalent \gls{squall} expression of a given question.
	\item The translator, which transforms \gls{squall} to \gls{sparql}.
	\item The linker, which maps placeholders to \glspl{uri}.
\end{enumerate}

It is important to reiterate that the translator component produces \gls{sparql} queries that still contain placeholders.
These outputs previously referred to as \glsentrylongpl{pgp}.
Only after all placeholders are replaced with \glspl{uri} do these \glspl{pgp} become executable queries,
referred to as \glsentrylongpl{gp}.

% Why
Although evaluation based solely on the domain expert's output is sufficient for internal comparisons,
as discussed in \Cref{s:internalEvaluation}
--- for instance, it allows for assessing the performance of the domain expert under various parameter configurations,
such as in the text-to-\gls{squall} experiment detailed in \Cref{s:text_to_prelinked_squall}
--- translation and linking are vital in order to compare the performance of the proposed approach to that of existing
methods in the literature.

% What
The evaluation focuses on the text-to-\gls{sparql} component of the proposed \gls{kgqa} pipeline,
as illustrated in \Cref{fig:pocImplementation}.
Two configurations of the domain expert, identified as the best-performing in the text-to-\gls{squall} evaluation,
are included alongside the baseline (see \Cref{table:text_to_prelinked_squall}):
$K_0$, (100, 100, 0.5), and (1000, 1000, 0.5).
All models were trained on ground truth data ($\mathscr{D}_{\text{gt}}$).

\subsection{Hypothesis}

% Hypothesis: what
No significant differences with the results of the text-to-\gls{squall} experiment were anticipated.
Likely the same, decreasing, order in terms of performances:
$K_0$, (1000, 1000, 0.5), and (100, 100, 0.5).

% Hypothesis: why
The linking strategy was implemented from the literature,
where it was tested and showed competitive performance \cite{omarUniversalQuestionAnsweringPlatform2023},
hence, there was no reason to expect major differences.

\subsection{Setup}

% How
The text-to-\gls{sparql} component is evaluated using the \gls{bleu} score, execution accuracy and $F_1$ score.

\subsubsection{Addressing Discrepancies in BLEU Scoring}

An important consideration is the choice of references for calculating the \gls{bleu} score,
as illustrated in \Cref{lst:originalVersusFromSquallMappedSparql}.

This example compares the original \glsentryfull{pgp} of a SciQA question to an equivalent \gls{squall} expression.
When the \gls{squall} expression is translated back into \gls{sparql},
the resulting \gls{pgp} differs slightly from the original.
For instance, a \mintinline{sparql}{FILTER} statement may be replaced with an equivalent \mintinline{sparql}{BIND} clause.
While these differences do not alter the query results, they negatively impact the computed \gls{bleu} score.

In the provided example, the \gls{bleu} score between the original and the translated \gls{pgp} is only 73,
despite the functional equivalence of the queries.

\begin{listing}[!ht]

	\mint{text}{Original PGP:}
	\inputminted{sparql}{src/listings/structure-changed-through-mapping/original-pgp.sparql}
	\mint{text}{===}

	\mint{text}{Equivalent SQUALL:}
	\inputminted{text}{src/listings/structure-changed-through-mapping/equivalent-squall.txt}
	\mint{text}{---}

	\mint{text}{Translated PGP:}
	\inputminted{sparql}{src/listings/structure-changed-through-mapping/translated-pgp.sparql}
	\mint{text}{===}

	\mint{text}{Generated SQUALL:}
	\inputminted{text}{src/listings/structure-changed-through-mapping/generated-squall.txt}
	\mint{text}{---}

	\mint{text}{Generated PGP (i.e., mapped from generated SQUALL):}
	\inputminted{sparql}{src/listings/structure-changed-through-mapping/generated-pgp.sparql}

	\caption{
		An example question with its original \glsentrylong{pgp},
		as well as an equivalent \glsentryshort{squall} expression,
		and the \glsentrylong{gp} the \glsentryshort{squall} expression has been mapped to by the translator.
	}
	\label{lst:originalVersusFromSquallMappedSparql}
\end{listing}

\subsubsection{Mitigating the Scoring Discrepancy}

Since every output from the text-to-\gls{sparql} component is processed through the translator,
comparing it to the original \glspl{pgp} would systematically penalize the evaluation.
This issue is highlighted in \Cref{lst:originalVersusFromSquallMappedSparql},
which also provides an example of a generated \gls{squall} expression and its translated \gls{pgp}.
Depending on the reference used, the \gls{bleu} score can vary significantly
--- 54 when compared to the original \gls{pgp} versus 79 when compared to the translated \gls{pgp}.
Notably, the only discrepancy in this example is the generated \mintinline{text}{contribution} placeholder.

This issue persists even when \glspl{uri} are used instead of placeholders,
as linking does not resolve these discrepancies.
Therefore, to ensure a fair evaluation of the text-to-\gls{sparql} component, this effect must be avoided.

\subsubsection{Creating New References}

To address this issue, new reference queries are created using the following process:

\begin{enumerate}
	\item Create equivalent \gls{squall} expressions for each \gls{sparql} query.
	\item Translate the equivalent \gls{squall} expressions back to \gls{sparql} using the translator.
\end{enumerate}

This approach ensures that the reference queries have also passed through the translator,
just like the outputs of the text-to-\gls{sparql} component.
Consequently, the discrepancies described earlier are avoided.

The detailed process for creating these reference queries is discussed in \Cref{s:gt_retriever_data}.
This method provides a fair and consistent basis for evaluation.

\subsection{Results}

The results of the experiment are presented in \Cref{table:text_to_sparql}.
These results confirm the hypothesis: the baseline setup remains the best-performing configuration,
as discussed in \Cref{s:text_to_prelinked_squall},
followed closely by the setup using parameter triple (1000, 1000, 0.5).

\begin{table}[t]
	\centering
	\begin{tabular}{l|SSSS}
		\textbf{Approach} & \textbf{BLEU} & \textbf{Accuracy} & \textbf{F1} & \textbf{Failure} \\

		\hline

		Baseline ($K_0$)  & 94.34         & 86.52             & 86.52       &  6.83\% \\
		(100,  100, 0.5)  & 92.76         & 83.57             & 83.72       & 21.81\% \\
		(1000, 1000, 0.5) & 93.95         & 85.77             & 85.78       & 14.98\% \\

	\end{tabular}
	\caption{
		\glsentryshort{bleu}, execution accuracy and $F_1$ scores of different setups on the text-to-\glsentryshort{sparql} task.
		All domains experts were trained using the ground truth domain dataset.
	}
	\label{table:text_to_sparql}
\end{table}



Samples for which no \gls{sparql} query was ultimately obtained are excluded from the computation of scores.
Instead, these cases are represented by their failure percentages.
Failures arise due to several reasons:  

\begin{itemize}
	\item General \gls{llm} generation issues, such as repeating tokens excessively or producing incomplete outputs.
	\item Inability to extract \gls{squall} expressions from the generated output,
	 e.g., because the \gls{llm} does not follow the expected output format.
	\item Syntax errors in the extracted \gls{squall}, rendering it unmappable to \gls{sparql}.
	\item Partial or complete failure in linking placeholders to \glspl{uri},
	 e.g., not finding a \gls{uri} for the \mintinline{text}{contribution} placeholder.
	\item Semantic errors in the linked \gls{sparql} query, which can occur due to:
	\begin{itemize}
		\item Incorrect linking, i.e., mapping placeholders to erroneous \glspl{uri}.
		\item Incorrect semantics in the generated \gls{squall}, for example, instead of generating


			\mintinline[breaklines]{sparql}{?dataset rdf:type <Dataset>},


			the \gls{llm} might produce


			\mintinline[breaklines]{sparql}{<Dataset> rdf:type ?dataset}.
	\end{itemize}
\end{itemize}

Failures are not included in the computation of scores due to the stringent requirements of the translator,
which requires inputs to be syntactically correct, otherwise an error is elicited.
This contrasts with approaches that generate \gls{sparql} queries directly,
which may contain minor syntax errors,
but still be readily usable for evaluation.
Since the text-to-\gls{sparql} component produces no output in these cases,
alternatives were looked at for evaluation purposes,
for example, using a generic fallback query.
Ultimately, this avenue was not pursued further, because it was deemed to make the results less interpretable.

Consider the baseline, the results now indicate that
for over 90\% of the inputted questions the text-to-\gls{sparql} component can output a syntactically correct query.
Furthermore, among this successful subset of samples,
a \gls{bleu} score of 94.34 and an $F_1$ score of 86.52 are achieved.

%Extensive error analysis could provide deeper insights into these results.
%Such an analysis was considered beyond the scope of this work, as the focus lies on implementing a proof of concept to demonstrate the feasibility of model reporting.

\section{Comparative Analysis}

% Intro
The results of the proposed approach are compared against a relevant selection from the literature to contextualize its
performance.

% Recap
While previous experiments have demonstrated the effectiveness of the proposed approach,
a comparative evaluation is necessary to assess its relative performance.

% Why
Such comparisons are critical for validating the architecture's effectiveness under limited data requirements
and assessing how it performs compared to approaches operating under ideal data availability conditions.

% What
Two scenarios are considered for this analysis:

\paragraph{Realistic Scenario}

The proposed approach is compared to \gls{kgqan} \cite{omarUniversalQuestionAnsweringPlatform2023},
a \gls{kgqa} platform designed to operate without domain-specific training data.
This comparison is particularly relevant,
as it represents the most challenging version of the text-to-\gls{sparql} task.

\paragraph{Ideal Scenario}

The performance of the proposed approach is also contrasted with a recent benchmark on the SciQA dataset
\cite{lehmannLargeLanguageModels2024}.
This benchmark evaluated a range of \gls{llm} setups, including fine-tuning and prompting strategies.
This scenario reflects the easiest version of the text-to-\gls{sparql} task,
where models are specifically tailored for the domain using ground truth data.

This dual comparison highlights the strengths and limitations of the proposed approach across diverse conditions.
By contrasting it with \gls{kgqan} for the most challenging scenario
and with state-of-the-art \gls{llm} setups for the ideal case, a comprehensive evaluation is achieved.

\subsection{Hypothesis}

% Hypothesis 1: What?
First, the \gls{kgqan} platform was selected for its alignment with the study's objectives.
It is hypothesized that \gls{kgqan} would perform competitively on the benchmark,
providing a robust point of comparison.
However, \gls{kgqan} is expected to represent a lower bound on performance for this work,
as it does not utilize any domain-specific data.

% Hypothesis 1: Why?
This expectation is grounded in \gls{kgqan}'s proven adaptability across a wide range of domains,
including general-purpose domains and more specialized ones like DBLP \cite{omarUniversalQuestionAnsweringPlatform2023}.

% Hypothesis 2: What?
Second, the SciQA benchmark results \cite{lehmannLargeLanguageModels2024} provide a complementary perspective.
The best results were achieved by fine-tuned \glspl{llm},
which the proposed approach is hypothesized not to surpass.
These benchmarks are expected to establish an upper bound for performance on SciQA,
as they only use domain-specific data.

% Hypothesis 2: Why?
This hypothesis is grounded in the expectation that specialized training confers a significant advantage for the
text-to-\gls{sparql} task, particularly on a dataset like SciQA,
which is relatively homogeneous, even across splits (see \Cref{s:text_to_prelinked_squall}).
Additionally, it is possible that \glspl{llm} do not truly learn sentence semantics
but instead rely on mapping questions to learned query structures \cite{reydAssessingGeneralizationCapabilities2023}. 
Building on this premise, training an \gls{llm} on question-query structures it will not encounter during evaluation
could result in greater confusion and, consequently, degraded performance.

\subsection{Setup}

First the approaches are summarized and compared to this work, afterwards the setups are detailed.

\subsubsection{KGQAn}

% Recap approach
\gls{kgqan} \cite{omarUniversalQuestionAnsweringPlatform2023} was first introduced in \Cref{s:QASPDA},
because it seemed an interesting work to compare to
due to its similar goals as a framework to assist non-technical stakeholders access \gls{kg} information.
It aims to be universal, and capable of forming queries for any \gls{kg},
without \gls{kg} specific training data nor prior information.

It attempts to achieve this by splitting up the text-to-\gls{sparql} task in two steps:
a question understanding step and a linking step.
%
The first step is aimed at sentence semantics:
given an input question a list of semantic triples is generated 
--- e.g., \mintinline[breaklines]{text}{(model reporting, is, flexible)}
--- using a trained \gls{llm}.
The model was previously trained on a \gls{qa} dataset similar SciQA,
however, one made domain agnostic by replacing \glspl{uri} with placeholders.
%
The second step then attempts to construct a query from the generated list using heuristics,
for example, building an \mintinline{sparql}{ASK} query if the question starts with \mintinline{text}{is}.
Subsequently, the placeholders are mapped to \glspl{uri} from the target \gls{kg},
using a dynamic \glsentryfull{jit} linker detailed in the linking section (see \Cref{s:linking}).
Essentially, it queries the \gls{kg} using string matching based on the placeholders.

% Main Difference
A key distinction between \gls{kgqan} and the proposed approach lies in the method of query structure generation.
Unlike this work, which leverages an \gls{llm} to generate query structures
--- i.e., the domain expert
--- \gls{kgqan} employs heuristics to construct them from a list of semantic triples.
So, while both approaches utilize an intermediary representation, they are distinctly different,
namely \gls{squall} expressions in this work versus semantic triples in \gls{kgqan})
Furthermore, the assumptions about data availability differ.
\gls{kgqan} operates under stricter data constraints, as it does not utilize any domain-specific data,
in contrast to the data-scarce setting assumed for this study.

% Setup
Still, \gls{kgqan} remains a viable benchmark due to the significant overlap in the intended audience
and the relatively similar data constraints when compared to other methods in the literature.

The parameters for the linker in both the proposed architecture and \gls{kgqan} were configured as follows:

\begin{itemize}
	\item The maximum number of vertices and edges returned from a probing query was set to 50 and \num{1000}, respectively.
	\item After scoring, only the top 10 vertices or edges were retained.
	\item Only the most likely final answer was selected.
\end{itemize}

\subsubsection{Benchmarks}

% Recap of Approach
Recent work \cite{lehmannLargeLanguageModels2024} has established the standard for the text-to-\gls{sparql} task on the
SciQA dataset using a diverse array of \glspl{llm} (e.g., T5, GPT-3.5-turbo)
and strategies such as \gls{zsl}, \gls{fsl}, and fine-tuning.

% Key Differences
The primary aim of \cite{lehmannLargeLanguageModels2024} was to demonstrate how \gls{llm} fine-tuning and
\gls{fsl} approaches can achieve strong performance on the challenging SciQA benchmark.
However, this work was not conducted in the context of data scarcity or with a focus on generalizability.
The methodology employed a straightforward supervised training approach for fine-tuning and relied on prompt
optimization techniques for \gls{fsl}. 

In contrast, the approach in this study addresses data scarcity through techniques such as incorporating non-parametric
memory within the domain expert.
Additionally, generalizability is emphasized by fine-tuning the \gls{squall} expert specifically for the
text-to-\gls{sparql} task and utilizing it as a generator for the domain expert.

% Experimental Setup
Various prompt engineering strategies were explored, with the most effective being a similarity-based sampling approach.
This method involved selecting the seven most semantically similar questions as examples
and including them in the prompt alongside their corresponding \gls{sparql} queries.

\subsection{Results}

Finally, the results indicate that the proposed approach is competitive especially considering it is built for 
data-scarce conditions.

\subsubsection{KGQAn}
\label{s:kgqan}

% Actual results

Contrary to expectations, the platform significantly underperformed.
The results, presented here for completeness, indicate that 54\% of the test questions failed at generation,
i.e., \gls{qu} did not produce a valid list of triples.
For the remaining queries, the \gls{bleu} score was 4.42, with both accuracy and $F_1$ effectively zero.

% If unexpected: insight why this might be

To investigate these unexpected outcomes, the platform's source code and results were examined further.
When \gls{qu} succeeded in generating queries, the outputs were overly generic.
While some correct answers were occasionally retrieved during query execution,
they were vastly outnumbered by incorrect results.
Two critical features were identified as the root causes, severely limiting the platform's generalizability:

\begin{itemize}

	\item \textbf{\glsentrylong{qu}:}

		Generating triples instead of complete queries using an \gls{llm} does not circumvent the template problem
		\cite{reydAssessingGeneralizationCapabilities2023}.
		The challenge of producing the correct structure remains significant:
		outputting a list of triples rather than a complete query does not fundamentally resolve this issue.

	\item \textbf{Query Building:}

		It became clear that its query building approach based on heuristics does not generalize well to questions
		expected to be answered through \glspl{cgp} such as those from SciQA,
		a challenging benchmark \cite{lehmannLargeLanguageModels2024}.

\end{itemize}

\subsubsection{Benchmark}  
\label{s:benchmark}  

The main results are summarized in \Cref{table:comparativeAnalysis}.  
For each model, the strategy that yielded the best performance was selected.  

\begin{table}[t]
	\centering
	\begin{tabular}{l|S[table-format=2.2]}

		\textbf{Setup}                                      & \textbf{BLEU} \\
		\hline                                                                             
		T5-base, fine-tuned                                 & 96.31         \\
		GPT2-large, fine-tuned                              & 95.04         \\
		Dolly-v2-3b, similarity-based \glsentryshort{fsl}   & 80.15         \\
		GPT-3.5-turbo, similarity-based \glsentryshort{fsl} & 95.71         \\
                                                           
	\end{tabular}                                           
	\caption{
		\glsentryshort{bleu} scores set on the SciQA benchmark by different \glsentryshortpl{llm} \cite{lehmannLargeLanguageModels2024}.
	}
	\label{table:comparativeAnalysis}
\end{table}

  

\paragraph{Observations on Model Performance}  

The fine-tuned T5 model and the \gls{fsl} GPT-3.5-turbo configurations,
in particular, showcase outstanding performance, setting a high standard for comparison.
However, the approach proposed in this work demonstrates results that are only marginally lower than the benchmarks,
and are thus considered competitive, see \Cref{table:text_to_sparql}.

\paragraph{Critical Evaluation}  

However, an essential critique arises from \cite{reydAssessingGeneralizationCapabilities2023}:  
\begin{quote}  
    The handling of unknown question-query structures is particularly important because it is unclear if models really
	 generate queries based on sentence semantics, or if they simply map them to known query structures.
\end{quote}  
With enough data, it seems plausible that any sufficiently powerful \gls{llm} could generate correct queries for a
given question, provided it has encountered/encounters the corresponding question-query template during training/in its prompt. 

\paragraph{Additional Insights}  

The analysis in \cite{lehmannLargeLanguageModels2024} provides several key points:  
\begin{enumerate}  
    \item \textbf{Entity Hallucination}:  
		 Entity hallucination, likely influenced by pre-training data,
		 occurs when models hallucinate entities from external knowledge sources like Wikidata.  
		 This aligns with earlier identified concerns in this work (see \Cref{s:why_placeholders}).  
    
    \item \textbf{Syntactical Errors}:  
		 The fine-tuned T5 model exhibits a significantly higher rate of syntactical errors (40.0\%) compared to
		 \gls{fsl} GPT-3.5-turbo (5.2\%), indicating weaker knowledge of \gls{sparql}.  
		 Likely causes include T5's smaller size and differences in pre-training data and tasks.  
    
    \item \textbf{Semantic Errors}:  
		 Frequent errors include misspelled entities and incorrect predicates.  
\end{enumerate}  

\paragraph{Mitigation in the Proposed Approach}  

The proposed approach preemptively attempted to address these issues:  
\begin{enumerate}  
    \item By using placeholders and subsequently linking.  
    \item By employing the \gls{squall} expert as the generator for the domain expert,
		 and freezing it during the latter's training phase.  
    \item By leveraging \gls{rag}.  
\end{enumerate}  

\paragraph{Template Generalization and Prompting Strategies}  

Finally, one of the tested prompting strategies involved randomly sampling examples without regard to query templates.  
This universally resulted in poorer performance,
highlighting a recurring challenge \cite{dialloComprehensiveEvaluationNeural2024}:  
generalizing to unseen templates.
The efficacy of \gls{fsl} heavily depends on well-chosen examples.  
It appears unlikely that \gls{fsl} approaches would succeed with entirely unseen question-query structures 
\cite{reydAssessingGeneralizationCapabilities2023}.  

% Note: Final discussion not result discussion.
\section{Discussion}
\label{s:discussion}

This section provides a comprehensive analysis of the experimental results,
offering insights into their broader implications.
It connects the findings to existing literature, evaluates their alignment with the stated requirements,
and demonstrates how model reporting is implemented.

Throughout this work, numerous decisions were made to address data scarcity,
aiming to design an architecture suitable for \gls{mbse}.
The experiments were structured to explore the potential of the proposed approach in achieving effective model reporting.
Key design choices included
restating the text-to-\gls{sparql} task as text-to-\gls{squall} task,
fine-tuning a robust \gls{llm} to develop a general understanding of \gls{squall},
integrating this fine-tuned \gls{llm} into a trainable \gls{rag} architecture,
and investigating the model's adaptability to new domains using corresponding (synthetic) data.
These choices all had the common goal of realizing robust \glsentrylong{da} from a data rich source domain
to a specialized (\gls{mbse}) domain.

\subsection{Challenges in Generalization to New Questions}

Previous work \cite{reydAssessingGeneralizationCapabilities2023} shows that
\gls{nmt} approaches typically struggle with unseen question-query structures.
During training, \glspl{llm} are often exposed to questions that follow specific structures,
a pattern evident in the datasets that are widely used, such as LC-QuAD 2.0, and present in SciQA as well.
Consequently, the corresponding queries adhere to certain templates,
making it difficult for \glspl{llm} to generate correct queries for questions requiring novel query structures.

While the proposed approach is not without limitations, as discussed in \Cref{s:generalizationCapabilities},
the experiments suggest significant potential for further development.
\Cref{lst:domainAdaptation} illustrates the notable differences between queries in the original and target domains.
Enhancing the diversity of training data during the fine-tuning of the \gls{squall} expert is both feasible
and primarily a matter of data engineering.
Such improvements could bolster the domain expert's ability to generalize to question-query structures not encountered
during the second training phase.
Additionally, employing more advanced synthetic data generation techniques could further expand the variety and
representativeness of the training dataset, enhancing the model's robustness and adaptability.

\subsection{Adaptation to New Knowledge Graphs and Domains}

The \gls{squall} expert was fine-tuned on LC-QuAD 2.0,
then frozen and integrated into a \gls{rag} architecture (see \Cref{s:sysArch}),
resulting in the domain expert.
That model was subsequently trained on SciQA, a dataset tailored for \gls{orkg} rather than Wikidata.
This training strategy (see \Cref{s:training_finetuning}) on two distinct domains provides valuable insight into the
domain adaptation, or more broadly, generalization capabilities of the proposed implementation.
As stated above, despite the significant differences between the \gls{sparql} templates used in LC-QuAD 2.0 and SciQA
the domain expert exhibited the ability to adapt to the new \gls{kg}/domain
using relatively little synthetic data of a lower than ground truth quality.

These results further corroborate the notion that \gls{squall} can effectively reduce data requirements.
Catastrophic forgetting the \gls{squall} expert is likely avoided of by keeping it frozen.
The use of placeholders enables reusing the \gls{squall} expert for a new domain, it makes possible this domain adaptation approach  .
Although deep error analysis is necessary to get good concrete empirical evidence.

\subsection{Adaptation to New Knowledge Graphs and Domains}

The \gls{squall} expert was fine-tuned on LC-QuAD 2.0, subsequently frozen,
and integrated into a \gls{rag} architecture (see \Cref{s:sysArch}), resulting in the domain expert.
This model was then trained on SciQA, a dataset specifically designed for \gls{orkg} rather than Wikidata.
This dual-domain training strategy (see \Cref{s:training_finetuning})
offers valuable insights into the domain adaptation and generalization capabilities of the proposed implementation.

Despite the significant differences between the \gls{sparql} templates used in LC-QuAD 2.0 and SciQA,
the domain expert demonstrated the ability to adapt effectively to the new \gls{kg}/domain using relatively limited
synthetic data, even when the data was of lower quality than ground truth.
These results support the potential of \gls{squall} to reduce data requirements effectively.

Freezing the \gls{squall} expert likely mitigates catastrophic forgetting,
ensuring that the gained \gls{squall} knowledge is retained.
Moreover, the use of placeholders facilitates the reuse of the \gls{squall} expert across domains,
enabling the proposed approach to domain adaptation.
While these findings are promising,
a detailed error analysis is necessary to provide robust empirical evidence and uncover potential areas for refinement.

\subsection{Ambiguity Related to RAG}

The effectiveness of using \gls{rag} in the proposed framework remains inconclusive.
While some improvement was observed with higher recall of constructed subgraphs, the baseline approach
--- where no subgraph is provided during generation
--- still outperformed the other setups (see \Cref{table:text_to_prelinked_squall}).
This outcome is likely due to the homogeneous nature of the SciQA dataset (see \Cref{s:text_to_prelinked_squall}),
which may limit the potential advantages of \gls{rag}.
In such cases, the baseline approach likely leads to memorization
--- due to the retrieved data being highly similar across samples,
--- outperforming actual learning to utilize retrieved data, which the other setups do.

Further investigation is required to confirm or rule out the hypothesized benefits of \gls{rag}.
Future work could explore its performance across more diverse datasets to provide a definitive assessment of its
utility within this context.

\subsection{Feasibility of Using Synthetic Data}

It is noteworthy that the majority of SciQA's samples are synthetically generated \cite{auerSciQAScientificQuestion2023}.
Consequently, the experimental results
--- including those using ground truth data 
--- support the idea that data scarcity can be effectively addressed through the use of synthetic data.
Employing advanced techniques to collect and generate question-query templates tailored to specific domains could
be critical to model reporting.

\subsection{Potential for Model Reporting and Data Transformation}

SciQA queries are notably complex \cite{lehmannLargeLanguageModels2024},
often surpassing those in datasets like LC-QuAD in several aspects.
They frequently involve a high number of triples and true \glspl{cgp},
which are combinations of \glspl{bgp} extended with relational operations \cite{anglesFoundationsModernQuery2017},
such as filters and unions.
While LC-QuAD queries include features like filters, limits, and basic ordering,
they lack the advanced constructs found in SciQA, such as aggregation, grouping, optionals, and nested queries.

Despite this complexity, the domain expert has demonstrated the ability to effectively handle \glspl{cgp}.
Current transformations present in the \gls{sparql} queries include solution modifiers and grouping
(see \Cref{lst:sciqa_4}).
Importantly, \gls{squall}'s comprehensive coverage of the \gls{sparql} 1.1 standard,
including graph updates \cite{ferreSQUALLExpressivenessSPARQL2014},
offers significant potential for extending these capabilities.
This expansion could support and automate more phases of the model reporting pipeline,
including advanced transformations.

Achieving this advancement would require high-quality data for fine-tuning,
along with significantly more diverse (synthetic) training data.
Such improvements represent a critical step toward enabling more robust transformations within the model reporting
pipeline.
An overview of the complex patterns achievable by the current approach is provided in \Cref{appendix:sciqa_sparqls},
showcasing examples of optional clauses, filters, solution modifiers (e.g., max, limit, order by)
aggregation, and grouping.

