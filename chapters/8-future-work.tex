\chapter*{Future Work}
\chaptermark{Future Work}
\addcontentsline{toc}{chapter}{Future Work}  
\label{c:futureWork}

This chapter outlines potential future research directions inspired by this work.
These avenues are categorized into five key areas:
\gls{ai} agent, \glsentrylong{da}, \glsentrylong{rag}, synthetic data generation, and linking.

\section{AI Agent}

Mo-Lab, as a proof-of-concept implementation of the proposed model reporting paradigm,
is currently limited in its capabilities.
Future research could explore incorporating the domain expert into a fully-fledged \gls{ai} agent,
enabling interactive features such as dynamic feedback through a dialog between the stakeholder and the agent.
This enhancement could facilitate iterative refinement of results, improving both usability and accuracy.
Additionally, such an agent could respond to stakeholder requests in terms of analysis preferences,
visualization formats, and other personalized requirements.

The proposed text-to-\gls{sparql} pipeline, which includes the domain expert,
could serve as a specialized subcomponent of the agent, focusing on text-to-\gls{sparql} tasks.
Meanwhile, the general capabilities of a \gls{pllm}, such as ChatGPT, could be retained to handle broader interactions,
ensuring a versatile and comprehensive \gls{ai} agent.

\section{Domain Adaptation}

A promising direction for future research is enriching the source domain data used to train the \gls{squall} expert.
Since \gls{da} can be applied to a mixed source domain \cite{mansourDomainAdaptationMultiple2008},
incorporating multiple \gls{qa} datasets (e.g., LC-QuAD 2.0, DBLP-QuAD, QALD-9-Plus, etc.)
could enhance generalizability, improve practical applicability,
and reduce training data requirements for the domain expert in the target domain.

\section{Retrieval Augmented Generation}

Evaluating the proposed approach on a diverse set of more varies and less homogeneous benchmarks is necessary,
as the current experimentation could not yield conclusive results.
For instance, replacing DBLP and DBLP-QuAD with \gls{orkg} and SciQA could provide valuable insights.
While DBLP was initially considered but abandoned due to its size,
revisiting it now can be justified given the promising results achieved.

\section{Synthetic Data Generation}

Further exploration of synthetic data generation techniques presents another exciting research avenue.
Investigating methods to produce data that is more faithful, extensive,
and varied across the domain could significantly enhance domain expert training.
Additionally, this work focused on generating queries while assuming the availability of questions.
Future research could explore techniques for simultaneously generating both queries and questions.
Developing a framework for directly generating data from an ontology or \gls{kg} of a domain could also be valuable,
particularly if it is practically applicable across real-world \gls{mbse} projects.

\section{Linking}

The linking component utilized in this work is a component separate from the remainder of the domain expert, 
and it is non-neural.
Exploring more advanced linkers that could leverage the constructed subgraph output during post-retrieval,
might offer significant improvements.
Neural approaches applicable in data-scarce scenarios, potentially through prompt engineering or \gls{da},
represent another promising direction for future work.

